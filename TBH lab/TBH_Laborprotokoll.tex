\newcommand{\institut}{}
\newcommand{\fachgebiet}{Halbleiterbauelemente}
\newcommand{\veranstaltung}{Praktikum Technologie und Bauelemente der Halbleitertechnik}
\newcommand{\pdfautor}{Dirk Barbendererde (321 836), Thomas Kapa (), Alona
Siebert (), Özgü Dogan (326048)}
\newcommand{\autor}{Dirk Barbendererde (321 836)\\ Thomas Kapa ()\\ Alona
Siebert ()\\ Özgü Dogan (326 048)}
\newcommand{\pdftitle}{Praktikum\ Technologie und Bauelemente der
Halbleitertechnik}
\newcommand{\prototitle}{Praktikum Technologie und Bauelemente der Halbleitertechnik}
\newcommand{\aufgabe}{}

\newcommand{\gruppe}{Gruppe 1}
\newcommand{\betreuer}{Betreuer:\\ Clemens Helfmeier\\ Philipp Scholz}



\input{../../packages/tu_header_9}

\setcaptionwidth{7.5cm}

\begin{document}


%     \lstinputlisting{./praktikum6.sce}



%---------------------------------------------------------------------
%---------------------------------------------------------------------
%---------------------------------------------------------------------

\section{Kennlinie}
\begin{quote}
    
    Benennung der Dateien:\\
    Kennlinie_{Wavernummer}_Die_[{Zeile},{Spalte}]_Mess_{messung}.mat\\
    
    100 µA Strombegrenzung\\
    Welche Diode?
    Welcher Manipulator
    
    SMU1 = GND
    SMU2 = GND
    SMU3 = Var1
    
    Unterdiffusion
    
    unterspannung
    0.0 & 2.4
    0.5 & 3.8
    0.8 & 4.5
    1.0 & 5.0
    1.3 & 5.5
    1.8 & 6.7
    2.5 & 8.2
    3.0 & 9.2
    
    \TODO{müssen wir die Dies durchnummerieren?}
\end{quote} %sec Kennlinie

%--------------------------------------------------------------------
%--------------------------------------------------------------------

\section{Schaltverhalten}
\begin{quote}
    
\end{quote} %sec Schaltverhalten

%--------------------------------------------------------------------
%--------------------------------------------------------------------

\section{Emissionsmessung}
\begin{quote}
    
    \TODO{Einleitung, Theorie, Verknüpfung, Messung, Ergebnisse, Auswertung}
    
    Eine Emissionsmessung ist in der Halbleitertechnologie insofern interessant,
    weil sie Erkenntnisse über charakteristische Eigenschaften des Halbleiters,
    in unserem Fall die selbst hergestellte Diode, liefert. Dazu gehören die
    Ladungsträgerlebensdauer $\tau_{n,p}$ und die Diffusionslänge $L_{n,p}$. Auf
    die Lebensdauer lässt sich mithilfe der Diffusionslänge und des
    Diffusionskoeffizienten schließen. Dieser ist ein materialabhängiger Wert,
    welcher von uns nicht weiter beachtet wird. Der folgende Zusammenhang hilft
    bei der Berechnung von $\tau_{n,p}$:
    
    \begin{equation*}
        \begin{split}
            L_{n,p} = \sqrt{\tau_{n,p} \cdot D_{n,p}} 
        \end{split}
    \end{equation*}
    
    Ziel unserer Messung aber, war die Bestimmung der Diffusionslänge in unserer
    Diode. Dieser wurde über die realisierte Intensitätsmessung anhand der
    Emissionen in der Diode ermittelt. Dabei wurden folgende
    Proportionalitätsverhätnisse verwendet:
    
    \begin{equation*}
        \begin{split}
            I(x) \sim \ \Delta n \sim \ exp(-\frac{x}{L_n}) 
        \end{split}
    \end{equation*}
    
    Hierbei werden Strahlungsintensität ins Verhältnis mit der\\
    Minoritätsüberschussladungträgerkonzentration, hier frei bewegliche
    Elektronen im p-Gebiet, und dieser wiederum ins Verhältnis mit einem
    Exponentialtherm gesetzt. Daher kann man die Intensitätsmessung direkt mit
    diesem Therm in Verbindung setzen, welcher in seinem Argument die gesuchte
    Diffusionslänge beinhaltet. Weiteres zur Berechnung des $L_n$ steht in der
    Auswertung der Messung.\\
    
    Um aber die Intensitätsmessung verstehen zu können müssen einige
    grundlegende Theorien der Halbleiter bezüglich ihrer Typen und ihrer
    Rekombinationsarten behandelt und nachvollzogen werden.\\ 
    Daher gibt es vor der Versuchsdurchführung und der Auswertung zunächst eine
    kleine Exkursion in den theoretischen Bereich.
    
        \subsection{Direkter und indirekter Halbleiter }
        \begin{quote}
        
        Es gibt zwei Arten von Halbleitern, die direkten und die indirekten
        Halbleiter. Diese unterscheiden sich darin, dass die Rekombination eines
        Ladungsträgers aus dem Leitungs- in das Valenzband unterschiedliche
        Vorraussetzungen erfordert.
            
            \subsubsection{direkter Halbleiter}
            \begin{quote}
            Die Rekombination bei einem direkten Halbleiter ist relativ simpel.
            Ein freies Elektron braucht dabei nur, unter Abgabe der jeweiligen
            Energie, den Bandabstand zwischen Leitungs- und Valenzband zu
            überqueren.
            
            \begin{figure}[H]
                    \centering
                        \includegraphics[scale=0.7, trim = 1cm 1cm 1.5cm 0cm,
                        clip]{./Emissionsbilder/restliches/direkt.png}
                        \caption{Rekombination bei einem direkten Halbleiter}
                            \label{fig:./Emissionsbilder/restliches/direkt.png}
            \end{figure}
            
            
            \end{quote}       
        
            \subsubsection{indirekter Halbleiter}
            \begin{quote}
            Die Rekombination bei einem indirekten Halbleiter erfordert neben
            einem Energieunterschied auch einen Impulsunterschied, damit das
            Elektron auf dem Valenzband auftreffen kann. Dieser
            Impulsunterschied ist in der folgenden Abbildung zu erkennen.
            
            \begin{figure}[H]
                    \centering
                        \includegraphics[scale=0.7, trim = 1cm 1cm 1.5cm 0cm,
                        clip]{./Emissionsbilder/restliches/indirekt.png}
                        \caption{Rekombination bei einem indirekten Halbleiter}
                            \label{fig:./Emissionsbilder/restliches/indirekt.png}
            \end{figure}
            
            \TODO{Bildquellen, TBH-Skript S.92 einfügen}
            \end{quote}       
        
        \end{quote}
        
    
\end{quote} %sec Emissionsmessung

%--------------------------------------------------------------------
%--------------------------------------------------------------------


\end{document}
