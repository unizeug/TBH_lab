\newcommand{\institut}{}
\newcommand{\fachgebiet}{Halbleiterbauelemente}
\newcommand{\veranstaltung}{Praktikum Technologie und Bauelemente der Halbleitertechnik}
\newcommand{\pdfautor}{Dirk Barbendererde (321 836), Thomas Kapa (325319), Alona Siebert (), Özgü Dogan (326 048)}
\newcommand{\autor}{Dirk Barbendererde (321 836)\\ Thomas Kapa (325319)\\ Alona Siebert ()\\ \"Ozg\"u Dogan (326 048)}
\newcommand{\pdftitle}{Praktikum\ Technologie und Bauelemente der
Halbleitertechnik}
\newcommand{\prototitle}{Praktikum Technologie und Bauelemente der Halbleitertechnik}
\newcommand{\aufgabe}{}

\newcommand{\gruppe}{Gruppe 1}
\newcommand{\betreuer}{Betreuer:\\ Clemens Helfmeier\\ Philipp Scholz}



\input{../../packages/tu_header_9}

\setcaptionwidth{7.5cm}

\begin{document}


% \lstinputlisting{./praktikum6.sce}

\sloppy



%---------------------------------------------------------------------
%---------------------------------------------------------------------
%---------------------------------------------------------------------

\section{Einleitung}
\begin{quote}


	Die fortschreitende Miniaturisierung mikroelektronischer Bauteile hat uns
	den Übergang zur planartechnologieschen Herstellung  von Transistoren und
	Integrierten Schaltungen ermöglicht. Der Grundbaustein der Planartechnologie
	ist ein pn-Übergang.\\
	\\
	Innerhalb dieses Praktikums wird eine pn-Diode (die einfachste Form eines
	pn-Übergangs) im Reinraum hergestellt und es werden ihre Eigenschaften
	untersucht und gemessen.\\
	\\
	Die Herstellung einer Diode erfolgt in einem mehrstufigen Prozess.
 	Als erstes wird ein Wafer aus dem Silizium hergestellt und als
 	Ausgangsmaterial für weitere Prozessschritte benutzt . Danach werden dünne
 	Schichten aus Materialien mit unterschiedlichen Eigenschaften schichtweise
 	auf diesem Siliciumsubstrat (Wafer) aufgebaut und durch verschiedene
 	Verfahren wie Lithographie und Ätzen bearbeitet.\\
 	\\
 	Das Endergebnis soll ein pn-Übergang sein.\\

\end{quote} %sec Einleitung

%--------------------------------------------------------------------
%--------------------------------------------------------------------
\section{Herstellung eines pn-Übergangs}
\begin{quote}

	\subsection{Herstellungsschritte}


		\subsubsection{Wafer vorbereiten}

			\textbf{Vorbereitung der Wafer und Nummerierung:}\\
			\\
			Einige Schritte der Waferherstellung waren bereits schon vor
			unserem Praktikum durch die betreuenden Labormitarbeiter
			durchgeführt.\\
			\\
			Die an sich runden Wafer haben zwei Flats: das große tiefere Flat
			gibt die Kristallrichtung des monokristallinen Wafers an und das
			kleinere Flat gibt den Dotierstoff an. Die für unser Praktikum
			verwendeten Wafer haben p-Type Substrat und wurden  mit Bor
			vordotiert.\\
			\\
			Danach wurde jeder Wafer auf der Rückseite nummeriert
			(s. Abb. \ref{fig:WafRueckseite}).	Die Nummern waren 120501, 120502
			und 120503.\\

			\vspace{2em}

			\begin{figure}[H]
				\hspace{2.0cm}
                \includegraphics[scale=0.8, trim = 0cm 0cm 0cm 0cm,clip]
                	{./HerstellungBilder/WafRueckseite.png}
                  \caption{Schema für die Beschriftung}
                \label{fig:WafRueckseite}
            \end{figure}

            \vspace{2em}

            \textbf{Reinigung:}\\
			\\
            Die Wafer wurden nun vor dem Herstellungsprozess gereinigt. Diesen
            Schritt braucht man um diverse kleine Partikel aus der umgebenden
            Luft, die sich auf der Oberfläche des Wafers ablagern, zu entfernen.
            Außerdem können metallische Rückstände bei der Nummerierung der
            Wafer auftreten.\\
			Es wurde ein RCA-Reinigungsprozess verwendet, der aus zwei Schritten
			besteht: Standard-Clean 1 und Standard-Clean 2 (SC1 \& SC2).\\
			\\
			SC1 wird zum Entfernen von organischen Resten verwendet: Der Wafer wird
			10 Minuten lang in einer Lösung,  bestehend aus dem deionisierten
			Wasser $H_{2}O$, Ammoniak $NH_{4}+OH$ und Wasserstoffperoxid
			$H_{2}O_{2}$ im Verhältnis 5 : 1 : 1, gespült.\\
			Außerdem wurde ein HF-Dip mit 1 \%iger Lösung 25 Sekunden lang
			durchgeführt, um das natürliche Oxid von der Oberfläche zu
			entfernen.\\
			SC2 wird zum Entfernen von metallischen Resten benutzt. Dafür wird
			der Wafer mit einer wässrigen Lösung aus Salzsäure und
			Wasserstoffperoxid  im Verhältnis 6 : 1 : 1 behandelt.\\
			\\
			Dabei  soll man unbedingt die Warnhinweise beachten. Da die bei der
			Reinigung verwendeten Lösungen sehr gefährlich (sehr giftig und stark
			ätzend) sein können, muss man unbedingt bei der Arbeit mit diesen
			Lösungen geeignete Schutzbekleidung und geeignete Schutzhandschuhe
			tragen. Beim Kontakt mit diesen Lösungen soll man sofort gründlich
			mit Wasser abspülen und den Arzt konsultieren.\\
			\\
			\textbf{Oxidieren:}\\
			\\
			Nach der Reinigung waren die Wafer mit einer dünnen chemischen
			Oxidschicht ($SiO_{2}$)  bedeckt. Da diese Schicht zu dünn ist, wird ein
			RTP-Prozess (Rapid Thermal Processing) benötigt.  Der Wafer wurde
			auf ca. $1200^{\circ}C$ erhitzt.  Die umgebende Luft reagiert mit den Si-
			Atomen auf der Oberfläche und es bildet sich dabei eine ca. 60 nm
			-dicke Siliziumoxid- Schicht. Danach wurde ein PECVD Prozess
			(Plasma Enhansed Chemical Vapor Deposition) benutzt: Es scheidet
			sich die Oxidschicht  auf der Oberfläche ab, die 250 nm dick ist.\\
			\\
			Bei der PECVD erfolgt die Abscheidung von dünnen Schichten durch
			eine chemische Aktivierung des Reaktionsprozesses, die durch ein
			Plasma unterstützt wird.\\
			Für die Schichtbildung werden die Ausgangsstoffe als Gasgemisch in
			einen Rezipienten eingelassen. Durch erhitztes Plasma werden die
			Bindungen des Reaktionsgases aufgebrochen und in Radikale zersetzt,
			die sich auf dem Substrat niederschlagen und dort die chemische
			Abscheidereaktion bewirken.\\
 			Der PECVD-Prozess lief unter $400^{\circ}C$ ab.\\
 			Die Oxiddickenmessung danach ergab ca. 400nm.\\
			\\
			Nach dem PECVD- Prozess wurde noch mal RTP-Tempern benötigt, um das
			Oxid in das Gitter einzufügen und damit das Oxid an der Oberfläche
			des Substrats gut haften kann. Bei RTP wurde der Wafer 120 Sekunden
			lang auf $1000^{\circ}C$ erhitzt.\\
			Die Oxiddickenmessung mit dem Photometer danach ergab ca. 420nm.\\
			\\
			\textbf{Lithographie:}\\
			\\
			Bei dem lithographischen Prozess wurden die Strukturen für die
			Diffusion der n-Wannen vorbereitet.\\
			Als erstes werden die Wafer von den möglichen Wasser- und
			Flüssigkeitsresten befreit. Dafür wird der Wafer 30 min. lang bei
			$200^{\circ}C$ auf einer Heizplatte erhitzt.\\
			\\
			\textbf{HMDS:}\\
			\\
			Außerdem hilft es ein Haftmittel HMDS (Hexamethyldisilazan) auf den
			Wafer aufzutragen:\\
			die Wafer werden in eine Vakuumglocke gelegt, es muss 30 Sekunden
			lang abgepumpt werden, um das Vakuum zu erzeugen.\\
			HMDS  lagert sich an der Oberfläche des Wafers ab (die Dauer beträgt
			5 min). \\
			Anschließend werden die Wafer  eine Minute lang auf der Heizplatte
			getrocknet.\\
			\\
			\textbf{Belacken \& Softbake:}\\
			\\
			Die  Wafer werden auf einen Schleuderapparat gelegt und mit dem AZ
			5214  Lack beschichtet. Mit dem Schleuderprogramm werden die Wafer
			auf 4000 U/min beschleunigt.\\
			Nach der 20-minutigen Pause werden die Wafer bei $90^{\circ}C$ zwei min.
			lang getrocknet, damit der Lack fest wird.\\
			\\
			\textbf{Belichten:}\\
			\\
			Nun werden die Wafer mit der Active Area Mask  (AA-Maske) abgedeckt
			und vier Sek. belichtet.\\
			\\
			\textbf{Entwickeln \& Hardbake:}\\
			\\
			Nach der Belichtung wurde der Lack in einer Rohm Haas - Lösung 70
			Sekunden lang entwickelt.\\
			Danach wurden alle Wafer auf einer Heizplatte unter $120^{\circ}C$ 5
			min. lang erhitzt, damit der restliche Lack gegen Ätzmittel noch
			resistenter wird.\\
			Die Inspektion mit dem Fotometer ergab die Dicke des Photolackes
			ca. 1.9 µm.\\
			\\
            \textbf{Nass-chemisches Ätzen:}\\
			\\
			Mit dem nass-chemischen Ätzen werden die Fenster zum Substrat
			weggeätzt.\\
			Die Wafer wurden in der BHF-Lösung (mit Ammoniumfluorid gepufferte
			Flusssäure) ca. 6.5 min. gehalten.\\
			\\
			\textbf{Fotolack entfernen:}\\
			\\
			Der Lack wurde mit  Caro's Etch - Lösung entfernt, indem die Wafer
			für 10 min. in diese Lösung eingetaucht wurden.\\
			\\
			Die Abbildung \ref{fig:AusStruk} zeigt die Struktur, wie wir unsere
			Wafer als Ausgangsmaterial für unseres Praktikum erhalten haben:

            \vspace{2em}

            \begin{figure}[H]
				\hspace{5.0cm}
                \includegraphics[scale=0.8, trim = 0cm 0cm 0cm 0cm,clip]
                	{./HerstellungBilder/AusgangsStruktur.png}
                  \caption{Ausgangsstruktur}
                \label{fig:AusStruk}
            \end{figure}

            \vspace{2em}





	\subsection{Tagesgliederung}

		\subsubsection{1. Tag}


		\textbf{Verhaltensregeln in dem Reinraum:}\\
		\\
		Als erstes, vor dem Eintritt in den Reinraum, erhielten alle Teilnehmer eine
		Sicherheitseinweisung von Herrn Bruhns, dem Laborleiter. Die Vorschrift
		ist: Schutzoveralls, spezielle Schuhe, die eine Erdung zum Boden
		enthalten, und Handschuhe zu tragen. Bei der Arbeit mit Chemikalien
		müssen die Augen zusätzlich geschützt werden. Dazu kann man entweder
		die Gesichtsschutzmaske oder die Brille tragen.\\
		Essen und Trinken im Reinraum sind streng verboten.\\
 		Außerdem sollte das Verlassen und Wiederbetreten möglichst vermieden
 		werden, da bei jedem Betreten die Partikel eingeschleppt werden können,
 		die die Reinheit des Labors vermindern.\\
		\\
		Weil mit den toxischen Substanzen gearbeitet wird, dürfen
		Schwangere das Praktikum nicht durchführen.

		\vspace{2em}

            \begin{figure}[H]
				\hspace{4.8 cm}
                \includegraphics[scale=0.5, trim = 0cm 0cm 0cm 0cm,clip]
                	{./HerstellungBilder/Teilnehmer.png}
                  \caption{Teilnehmer}
                \label{fig:teiln}
            \end{figure}

    	\vspace{2em}

    	\textbf{1. Dotierlösung aufbringen:}\\
		\\
		Als erstes haben wir unsere drei Wafer unter dem Lichtmikroskop
		untersucht, um die Struktur der Wafer zu sehen. Die Oxide reflektieren 
		das Licht, daher ist auch die grüne Farbe im Mikroskop zu sehen
		(s. Abb. \ref{fig:mikro1}) Es wurden keine großen Schäden gefunden.

    	\vspace{2em}

    		\begin{figure}[H]
				\hspace{4.7 cm}
                \includegraphics[scale=0.5, trim = 0cm 0cm 0cm 0cm,clip]
                	{./HerstellungBilder/Mikroskopbild1.png}
                  \caption{Mikroskopaufnahme}
                \label{fig:mikro1}
            \end{figure}

    	\vspace{2em}

    	Nun wird der Dotierstoff Phosphorus p 509 aufgetragen.

    	\vspace{2em}

    	\begin{center}
                \begin{tabular}{ll}

                \hspace{-14em}
                    \begin{minipage}{0.6\textwidth}
                        \begin{figure}[H]
                        \hspace{8em}
                            \includegraphics[scale=0.7, trim = 0cm 0cm 0cm
                            0cm, clip]{./HerstellungBilder/Phosphorus.png}
                            \caption{Phosphor}
                           \label{fig:phos}
                        \end{figure}

                    \end{minipage}
                    \begin{minipage}{0.6\textwidth}

                        \begin{figure}[H]
                        \hspace{3.5em}
                            \includegraphics[scale=0.7, trim = 0cm 0cm 0cm
                            0cm, clip]{./HerstellungBilder/Dotierstoffauftragen.png}
                            \caption{Auftragen des Dotierstoffes}
                           \label{fig:aufDot}
                        \end{figure}
                    \vspace{-1.5em}

                    \end{minipage}

                \end{tabular}
		\end{center}

    	\vspace{2em}

		Um eine gleichmäßige Verteilung des Dotierstoffes zu erhalten, wurde die
		Auftragung der Phosphorlösung mit einem
		Lackschleuderbeschichtungsapparat (s. Abb. \ref{fig:aufDot})
		durchgeführt.\\
		
 		Auf einem Drehtisch wird der Wafer zentral justiert und mittels Vakuum
 		fixiert, damit der Wafer bei der Drehung am Teller gut haftet. Mit Hilfe
 		von einem Dispenser (der zwei Mal gedrückt wurde) wird der Dotierstoff auf den
 		Wafer getropft. Danach muss der Wafer sofort in Rotation gebracht werden
 		, um eine ebenmäßige Dotierstoffschicht zu bekommen. Bei der Rotation
 		wird die überflüssige Stoffmenge von der Scheibe weggeschleudert.\\
		Es bleibt nur eine sehr dünne Phosphorschicht auf dem Wafer.\\
		\\
		Bedingungen beim Aufschleudern der Phosphorlösung:\\
		Programm 4\\
		2500 U/min (Anzeige 250)\\
		2 ml Lösung\\
		\\
		Bemerkung: der Wafer 120503 wurde doppelt mit dem Dotierstoff
		beschichtet (ca. 4 ml).\\
		Anschließend wurden alle drei Wafer auf der Heizplatte  15 Minuten lang
		bei $200^{\circ}C$ gebacken. Bei diesem Prozessschritt bildet sich ein
		Phosphorglas (Silikat). Dabei wurde eine Dampfwolke beobachtet. \\
		Das Ergebnis nach diesem Vorgang ist in der Abbildung \ref{fig:Waf_phos}
		zu sehen.\\

        \vspace{2em}

    		\begin{figure}[H]
				\hspace{4.7 cm}
                \includegraphics[scale=0.5, trim = 0cm 0cm 0cm 0cm,clip]
                	{./HerstellungBilder/StrukturmitPhosphorus.png}
                  \caption{Wafer bedeckt mit Phosphorglas}
                \label{fig:Waf_phos}
            \end{figure}

        \vspace{2em}

		Eine Sichtkontrolle mit dem Lichtmikroskop hat uns folgendes ausgegeben:

     	\vspace{2em}

    		\begin{figure}[H]
				\hspace{3.5 cm}
                \includegraphics[scale=0.5, trim = 0cm 0cm 0cm 0cm,clip]
                	{./HerstellungBilder/Mikroskopbild2.png}
                  \caption{Bläschen Wafer 01}
                \label{fig:blaes}
            \end{figure}

     	\vspace{2em}

    	Zu beobachten war:\\
		Bei dem Wafer 02 waren die Bläschen größer, der Wafer 03 hatte fast gar
		keine Bläschen, die Struktur war ebenmäßiger und er enthielt so viel
		Dotierstoff, dass die Fenster kaum zu sehen waren. Der Wafer 01 hatte
		große Bläschen, die ständig Ihre Form verändert haben.\\
		\\
		\textbf{2 Diffusionsofen:}\\
		\\
		Jetzt kommen der Wafer in den Diffusionsofen.\\
		\\
		Als erstes wurden die Wafer so in dem Quarzschiffchen einsortiert, dass
		sich die Vorderseiten der Wafer 01 und 02 gegenüber stehen. Wafer 03
		wurde allein auf der anderen Seite des Quarzschiffchens platziert.

		\vspace{2em}

    	\begin{center}
                \begin{tabular}{ll}

                \hspace{-7em}
                    \begin{minipage}{0.5\textwidth}
                        \begin{figure}[H]
                        \hspace{-2em}
                            \includegraphics[scale=0.8, trim = 0cm 0cm 0cm
                            0cm, clip]{./HerstellungBilder/Quarzschiffchen.png}
                            \caption{Quarzschiffchen}
                           \label{fig:quarz}
                        \end{figure}

                    \end{minipage}
                    \begin{minipage}{0.75\textwidth}

                        \begin{figure}[H]
                        \hspace{8em}
                            \includegraphics[scale=0.7, trim = 0cm 0cm 0cm
                            0cm, clip]
                            {./HerstellungBilder/einbringeninDiffusionsofen.png}
                            \caption{Einbringen in den Diffusionsofen}
                           \label{fig:ein_diff}
                        \end{figure}
                    \vspace{-1.5em}

                    \end{minipage}

                \end{tabular}
		\end{center}

		\vspace{2em}

		Dann wird der Schlitten 65 cm tief im Ofen platziert, damit die Wafer
		sich möglichst in der Ofenmitte befinden. Zum Abschluss wurde
		anschließend noch ein Quarzhohlzylinder in den Ofen geschoben, um die
		Temperaturerhaltung zu verbessern.\\
		Der Prozess der Diffusion findet unter dem Durchströmen des
		Diffusionsofens mit dem Prozessgas für zehn min. bei der Temperatur von
		$1000 ^{\circ}C$.

		\vspace{2em}

    		\begin{figure}[H]
				\hspace{4 cm}
                \includegraphics[scale=0.75, trim = 0cm 0cm 0cm 0cm,clip]
                	{./HerstellungBilder/diffusionsofen.png}
                  \caption{Diffusionsofen}
                \label{fig:diff_ofen}
            \end{figure}

    	\vspace{2em}

		Nach dem Diffusionsprozess verblieben die Wafer noch zum langsamen
		Auskühlen über die Nacht im Ofen.\\

			Nach der Diffusion sieht unsere Struktur so aus wie in Abb.
			\ref{fig:nach_diff_ofen}:

			\vspace{2em}

    		\begin{figure}[H]
				\hspace{4 cm}
                \includegraphics[scale=0.7, trim = 0cm 0cm 0cm 0cm,clip]
                	{./HerstellungBilder/StrukturnachDiffusionsofen.png}
                  \caption{Struktur nach Diffusionsofen}
                \label{fig:nach_diff_ofen}
            \end{figure}

    		\vspace{2em}



		\subsubsection{2. Tag}


			Am zweiten Tag haben wir unsere Wafer aus dem Ofen genommen. Dabei
			haben wir beobachtet, dass der Wafer 03 glänzend und der Wafer 01
			und 02 matt waren. Danach wurden alle Wafer einer Sichtkontrolle
			unter dem Mikroskop unterzogen und mit dem Photometer wurde die
			Schichtdicke gemessen.\\

			Photometer: Das Photometer Ergolux besteht aus einem Mikroskop mit
			einem Aufsatz, der Licht bestimmter Wellenlängen (zwischen 400 nm
			und 800 nm) auf den Wafer strahlt und über die Reflexion der
			Lichtquanten der unterschiedlichen Wellenlängen Aufschluss über
			die Schichtdicke gibt.\\
			\\
			Zur Kalibrierung muss das Photometer mit einer Musterprobe kalibriert
			werden (mit dem Flat nach unten): die Musterprobe (ein nicht
			beschichteter Siliziumwafer) soll auf den Objekthalter so platziert
			werden, damit man die Bildschärfe einstellen kann. Dann startet man
			das Programm Justage. Danach wird es ohne Musterprobe auf dem
			Objekthalter noch mal gestartet.\\
			Nur dann können wir unsere Wafer messen.\\
			\
			Die Messergebnisse für unsere drei Wafer sind in der Tabelle
			\ref{tab:Phosphordicke} zusammengefasst.

			\vspace{2em}

      		\begin{table}[H]
     		  \begin{addmargin}[1cm]{3cm}
     			\centering
                    \begin{tabular}{|p{2cm}|p{2cm}|p{2cm}|p{2cm}|p{2cm}|p{2cm}|}
         			\hline
         			Wafer & oben & mittig & unten & links & rechts\\
         			\hline
        			120501 & 82.3  & 78.1  & 86.8  & 85.4  & 73.2 \\
                    120503 & 770.1 & 784.8 & 765.3 & 764.8 & 776.2 \\
                    \hline

                    \end{tabular}
              \end{addmargin}
              \caption{gemessene Phosphorglasdicke in nm}
              \label{tab:Phosphordicke}
            \end{table}

            \vspace{2em}

            Die Wafer 01 und 02 waren nicht gut messbar! Der Wafer 03  hat
            dagegen fast perfekte Messergebnisse.

            \vspace{2em}

    		\begin{center}
                \begin{tabular}{ll}

                \hspace{-7em}
                    \begin{minipage}{0.5\textwidth}
                        \begin{figure}[H]
                        \hspace{-1em}
                            \includegraphics[scale=0.8, trim = 0cm 0cm 0cm
                            0cm, clip]{./HerstellungBilder/Photometerbild.png}
                            \caption{Photometerbild}
                           \label{fig:photobild}
                        \end{figure}

                    \end{minipage}
                    \begin{minipage}{0.7\textwidth}

                        \begin{figure}[H]
                        \hspace{5em}
                            \includegraphics[scale=0.8, trim = 0cm 0cm 0cm
                            0cm, clip]
                            {./HerstellungBilder/Photometer.png}
                            \caption{Das Photometer}
                           \label{fig:photometer}
                        \end{figure}
                    \vspace{-1.5em}

                    \end{minipage}

                \end{tabular}
			\end{center}

			\vspace{2em}

			Sichtkontrolle unter dem Lichtmikroskop:\\
			Wafer02 und Wafer01: keine klare Struktur

		    \vspace{2em}

    		\begin{center}
                \begin{tabular}{ll}

                \hspace{-14em}
                    \begin{minipage}{0.8\textwidth}
                        \begin{figure}[H]
                        \hspace{8em}
                            \includegraphics[scale=1.2, trim = 0cm 0cm 0cm
                            0cm, clip]{./HerstellungBilder/MikroskopW1.png}
                            \caption{Wafer 01 nach der Diffusion}
                           \label{fig:diff01}
                        \end{figure}

                    \end{minipage}
                    \begin{minipage}{0.4\textwidth}

                        \begin{figure}[H]
                        \hspace{-2em}
                            \includegraphics[scale=1.2, trim = 0cm 0cm 0cm
                            0cm, clip]{./HerstellungBilder/MikroskopW2.png}
                            \caption{Wafer 02 nach der Diffusion}
                           \label{fig:diff02}
                        \end{figure}
                    \vspace{-1.5em}

                    \end{minipage}

                \end{tabular}
			\end{center}

    		\vspace{2em}

			Wafer03: man kann einen großen Farbunterschied sehen. Die Schicht
			machte optisch einen guten Eindruck, die Struktur ist sehr klar.\\
			\\
			\textbf{Ätzen (Entfernen des Phosphorglases):}\\
			\\
			Jetzt wird das Phosphorglas entfernt. Dazu wird zuerst der Wafer01
			in die 2.5\%-ige Flusssäure für ca. fünf Minuten lang eingetaucht,
			um das Phosphorglas von der Oberfläche zu entfernen. Der übrig
			gebleibendene Phosphor wurde anschließend mit dem Wattestäbchen
			entfernt und in Di-Wasser gesäubert.\\
			Da die Flusssäure stark ätzend ist, müssen ein extra dicker
			Schutzanzug, Augenschutz und dickere Handschuhe getragen werden
			(Abbildung \ref{fig:schutzbe}). Nach dem Ätzen prüften wir die Dicke
			mit dem Photometer und stellten fest, dass wir die ganze Schicht
			wegätzten!! Die Sichtkontrolle mit dem Lichtmikroskop zeigte, dass
			Struktur noch da war, deswegen wurde entschieden den Wafer02 für
			Schrägschliff zu verwenden.

 			\vspace{2em}

    		\begin{center}
                \begin{tabular}{ll}

                \hspace{-14em}
                    \begin{minipage}{0.8\textwidth}
                        \begin{figure}[H]
                        \hspace{5em}
                            \includegraphics[scale=1.0, trim = 0cm 0cm 0cm
                            0cm, clip]{./HerstellungBilder/Schutzbekleidung.png}
                            \caption{Schutzbekleidung}
                           \label{fig:schutzbe}
                        \end{figure}

                    \end{minipage}
                    \begin{minipage}{0.40\textwidth}

                        \begin{figure}[H]
                        \hspace{-4em}
                            \includegraphics[scale=1.0, trim = 0cm 0cm 0cm
                            0cm, clip]
                            {./HerstellungBilder/ArbeitenamLichtmikroskop.png}
                            \caption{Arbeiten am Lichtmikroskop}
                           \label{fig:arblicht}
                        \end{figure}
                    \vspace{-1.5em}

                    \end{minipage}

                \end{tabular}
			\end{center}

    		\vspace{2em}


			Der Wafer02 wurde dann zwei Mal geätzt: ein Mal sehr vorsichtig mit
			der 2.5\%-igen Flusssäure nur für eine Minute und danach mit der
			1\%-igen nur für fünf Sekunden, jedes Mal wurde die Dicke gemessen.
			Der Wafer03 wurde  schon mehrmals  geätzt: zwei Mal sehr vorsichtig
			mit der 1\%-igen Flusssäure für eine Minute, dann mit der 1\%-igen
			nur für fünf Sekunden. Jedes Mal wurde die Dicke kontrolliert.
			Die Ergebnisse der Oxiddickenmessung sind in der Tabelle
			\ref{tab:Oxiddicke} dargestellt.

			\vspace{2em}

      		\begin{table}[H]
     		  \begin{addmargin}[1cm]{3cm}
     			\centering
                    \begin{tabular}{|p{2cm}|p{2cm}|p{2cm}|p{2cm}|p{2cm}|p{2cm}|}
         			\hline
         			Wafer & oben & mittig & unten & links & rechts\\
         			\hline
        			120502 & 196.5  & 199.9  & 198   & 205   & 195 \\
                    120503 & 201 	& 202 	 & 203.7 & 186.7 & 203 \\
                    \hline

                    \end{tabular}
              \end{addmargin}
              \caption{Oxiddicke}
              \label{tab:Oxiddicke}
            \end{table}

            \vspace{2em}

            Nach dem Ätzen sieht unsere Struktur folgendermaßen aus:

            \vspace{2em}

    		\begin{figure}[H]
				\hspace{3.5 cm}
                  \includegraphics[scale=0.9, trim = 0cm 0cm 0cm 0cm,clip]
                	{./HerstellungBilder/KontaktfenstergeaetztundLackentfernt.png}
                  \caption{Struktur nach dem Ätzen}
                \label{fig:nachAetzen1}
            \end{figure}

    		\vspace{2em}

    		Anschließend wurden die Wafer mit dem Stickstoff trocken gepustet
    		und für ca. 15 min. bei $200^{\circ}C$ auf eine Heizplatte gelegt,
    		um die molekulare Flüssigkeitsreste von der Oberfläche des Wafers zu
    		entfernen (sie stören die Haftung des Photolacks).\\
			\\
    		\textbf{Lithographie:}\\
			\\
			\textbf{HMDS}\\
			\\
			Die Wafer02 und 03 werden direkt von der Heizplatte in Exikator,
			eine so genannte Vakuumglocke(s. Abb \ref{fig:vakuumgl}) mit dem  Haftmittel HMDS
			(Hexamethyldisilazan), gebracht.

            \vspace{2em}

    		\begin{center}
                \begin{tabular}{ll}

                \hspace{-14em}
                    \begin{minipage}{0.7\textwidth}
                        \begin{figure}[H]
                        \hspace{5em}
                            \includegraphics[scale=1.0, trim = 0cm 0cm 0cm
                            0cm, clip]{./HerstellungBilder/Vakuumglocke.png}
                            \caption{Vakuumglocke}
                           \label{fig:vakuumgl}
                        \end{figure}

                    \end{minipage}
                    \begin{minipage}{0.3\textwidth}

                        \begin{figure}[H]
                        \hspace{-3em}
                            \includegraphics[scale=1.0, trim = 0cm 0cm 0cm
                            0cm, clip]
                            {./HerstellungBilder/Heizplatte.png}
                            \caption{Heizplatte}
                           \label{fig:heizpl}
                        \end{figure}
                    \vspace{-1.5em}

                    \end{minipage}

                \end{tabular}
			\end{center}

    		\vspace{2em}

    		Es wird das Vakuum für 30 Sekunden lang abgepumpt, das HMDS verdampft
    		und lagert sich an der Oberfläche des Wafers ab. Es wird ca. fünf
    		Minuten beschichtet, danach wird das Druckventil sehr langsam
    		gedreht und der Druck abgelassen. Anschließend werden die Wafer
    		für eine Minute bei $120^{\circ}C$ auf die Heizplatte gelegt.\\
			\\
			\textbf{Belacken und Softbake:}\\
			\\
			Nachdem die Wafer getrocknet waren, wurden sie mit einem Positivlack
			AZ 5214 beschichtet (s.Abb  und ). Beim Drehen der Schleuder, wenn
			sich der Lack ausbreitet, beobachtet man die farbigen Wellen. Es
			sind die Interferenzwellen von den Lackschichten. Je länger die
			Schleuderzeit ist, desto gleichmäßiger ist die Lackschicht.

    		\vspace{2em}

    		\begin{center}
                \begin{tabular}{ll}

                \hspace{-14em}
                    \begin{minipage}{0.7\textwidth}
                        \begin{figure}[H]
                        \hspace{5em}
                            \includegraphics[scale=1.0, trim = 0cm 0cm 0cm
                            0cm, clip]{./HerstellungBilder/BeschichtungdesWafers.png}
                            \caption{Beschichtung des Wafers}
                           \label{fig:Beschwaf}
                        \end{figure}

                    \end{minipage}
                    \begin{minipage}{0.3\textwidth}

                        \begin{figure}[H]
                        \hspace{-1em}
                            \includegraphics[scale=1.0, trim = 0cm 0cm 0cm
                            0cm, clip]
                            {./HerstellungBilder/Lackflasche.png}
                            \caption{Lackflasche}
                           \label{fig:lackfl}
                        \end{figure}
                    \vspace{-1.5em}

                    \end{minipage}

                \end{tabular}
			\end{center}

    		\vspace{2em}

    		Nach dem Lackieren muss der Wafer 30 Minuten lang in der halboffenen
    		Schachtel ruhen, damit die Gasbläschen ausdringen können und der Lack in
    		die kleinsten Rauigkeiten der Oberfläche eindringen kann.
    		Anschließend wurden die Wafer bei $90^{\circ}C$ zwei Minuten auf der
    		Heizplatte erhitzt, um den Lack zu fixieren. Sehr wichtig, dass die
    		Wafer sehr langsam abkühlen müssen!\\
			\\
			\textbf{Belichten mit der KF-Maske:}\\
			\\
			Das Belichten des Wafers erfolgt auf dem Maskenjustierer(s.Abb.
			\ref{fig:just}): Dieser Lithographie-Apparat hat die
			Belichtungshaube, darunter ist der Waferhalter und Maskenhalter,
			Mikroskop mit einer CCD-Kamera und einem Bildschirm.
			Ganz wichtig ist es, dass die Maske und der Wafer genau übereinander
			liegen, damit der Wafer an den nötigen Stellen belichtet werden kann
			.\\
 			Als erstes wurde der Maskenjustierer angeschaltet und die Lampe
 			vorgewärmt. Auf dem Bildschirm kann man die unten erscheinenden
 			Anleitungen befolgen. Weiter wurde die Maske geladen und mit dem
 			Vakuum festgehalten. Die KF-Maske wurde genau an der AA-Maske
 			justiert(man kann die Justierung in der Abbildung
 			\ref{fig:Justiervorgang} sehen, wo die Kreuze übereinanderliegen).

 			\vspace{2em}

    		\begin{center}
                \begin{tabular}{ll}

                \hspace{-14em}
                    \begin{minipage}{0.8\textwidth}
                        \begin{figure}[H]
                        \hspace{7em}
                            \includegraphics[scale=1.0, trim = 0cm 0cm 0cm
                            0cm, clip]{./HerstellungBilder/Maskenjustierer.png}
                            \caption{Justierung der Maske}
                           \label{fig:just}
                        \end{figure}

                    \end{minipage}
                    \begin{minipage}{0.4\textwidth}

                        \begin{figure}[H]
                        \hspace{-6em}
                            \includegraphics[scale=0.9, trim = 0cm 0cm 0cm
                            0cm, clip]
                            {./HerstellungBilder/Justiervorgang.png}
                            \caption{Justiervorgang}
                           \label{fig:Justiervorgang}
                        \end{figure}
                    \vspace{-1.5em}

                    \end{minipage}

                \end{tabular}
			\end{center}

    		\vspace{2em}

    		Der Wafer wurde ausgerichtet, mit dem Vakuum festgehalten und
    		hineingeschoben. Dann justiert man den Wafer so, dass er genau unter
    		der Maske liegt (s.Abb. \ref{fig:just}).\\
			\\
			Danach drückt man die Taste Exposition und der Wafer wird durch die
			Maske belichtet. Das Licht befindet sich im UV-Bereich. Bei dem
			Belichtungsprogramm gibt es verschiedene Belichtungsarten, wir haben
			uns für die Soft-Kontakt-Belichtungsart entschieden, weil es die
			schonendste Methode ist. Der Wafer 02 wurde vier Sekunden lang
			belichtet. Der Moment der Belichtung ist in der Abbildung
			\ref{fig:belichtung} zu sehen.

    		\vspace{2em}

    		\begin{center}
                \begin{tabular}{ll}

                \hspace{-14em}
                    \begin{minipage}{0.8\textwidth}
                        \begin{figure}[H]
                        \hspace{6em}
                            \includegraphics[scale=1.0, trim = 0cm 0cm 0cm
                            0cm, clip]{./HerstellungBilder/Belichtungsvorgang.png}
                            \caption{Belichtungsvorgang}
                           \label{fig:belichtung}
                        \end{figure}

                    \end{minipage}
                    \begin{minipage}{0.4\textwidth}

                        \begin{figure}[H]
                        \hspace{-3em}
                            \includegraphics[scale=0.9, trim = 0cm 0cm 0cm
                            0cm, clip]
                            {./HerstellungBilder/DerMomentderBelichtung.png}
                            \caption{Belichtung}
                           \label{fig:belichtung2}
                        \end{figure}
                    \vspace{-1.5em}

                    \end{minipage}

                \end{tabular}
			\end{center}

    		\vspace{2em}

    		Die durchsichtigen Bereiche der Maske lassen das Licht durch und der
    		Wafer wird belichtet.\\
			Durch das Belichten wird der Lack härter. An den Stellen, die nicht
			belichtet wurden, wird der Lack später mit einer Rohm-Haas Lösung
			abgezogen.\\
			Die Belichtungszeit stimmte für den Wafer02, deswegen wurde der
			Wafer03 auch mit dieser Zeit belichtet. Leider wurde beim Justieren
			der Wafer03 ein Stückchen Lack an der Maske angeklebt. Wir haben 
			sofort überlegt, ob es beim Entwickeln zu den Fehlern kommen kann.\\
			\\
			\textbf{Entwickeln und Hardbake:}\\
			\\
			Nach dem Belichtungsprozess wurde der Lack möglichst schnell mit
			Hilfe von Entwickler Rohm-Haas entfernt. Die Wafer wurden 60
			Sekunden lang in dieser Lösung gespült.\\
			Nach dem Entwicklungsprozess muss der Entwickler ganz schnell weg
			von der Waferoberfläche und sehr gründlich mit dem Wasser abgespült
			werden (s. Abb. \ref{fig:entw}).\\
			Um den restlichen Lack gegen das Ätzmittel noch resistenter zu machen:
			Hardbake fünf Minuten lang, $120^{\circ}C$.

			\vspace{2em}

    		\begin{center}
                \begin{tabular}{ll}

                \hspace{-14em}
                    \begin{minipage}{0.7\textwidth}
                        \begin{figure}[H]
                        \hspace{4em}
                            \includegraphics[scale=1.0, trim = 0cm 0cm 0cm
                            0cm, clip]{./HerstellungBilder/Entwickeln.png}
                            \caption{Entwickeln}
                           \label{fig:entw}
                        \end{figure}

                    \end{minipage}
                    \begin{minipage}{0.6\textwidth}

                        \begin{figure}[H]
                        \hspace{0em}
                            \includegraphics[scale=0.9, trim = 0cm 0cm 0cm
                            0cm, clip]
                            {./HerstellungBilder/NachdemEntwickeln.png}
                            \caption{Nach dem Entwickeln}
                           \label{fig:nachentw}
                        \end{figure}
                    \vspace{-1.5em}

                    \end{minipage}

                \end{tabular}
			\end{center}

    		\vspace{2em}

    		\textbf{Sichtkontrolle:}\\
			\\
			Mit dem Lichtmikroskop: Nun sollen die Wafer mit dem Mikroskop
			untersucht werden. Wie wir es schon erwartet haben, der angeklebte
			Lack hat sich bemerkbar gemacht: Die Struktur ist nicht mehr sehr
			deutlich zu sehen, viele untergeätzte Stellen:

            \vspace{2em}

    		\begin{figure}[H]
				\hspace{3 cm}
                  \includegraphics[scale=0.9, trim = 0cm 0cm 0cm 0cm,clip]
                	{./HerstellungBilder/Mikroskopbild3.png}
                  \caption{Nach dem Entwickeln}
                \label{fig:nachentwickelnwaf}
            \end{figure}

    		\vspace{2em}

    		\textbf{Dektak-Messung:}\\
			\\
			Nun wurden die Lackdicken von den Wafern 02 und 03 am Dektak
			gemessen.\\
			Der Dektak ist ein Profilometer mit dem sich die mikroskopische
			Oberflächenrauigkeit ermitteln lässt.\\
			Ein Mal haben wir über dem p- und das andere Mal über dem n-Pad
			gemessen:\\

			\vspace{2em}

    		\begin{figure}[H]
				\hspace{-1.5 cm}
                  \includegraphics[scale=1, trim = 0cm 0cm 0cm 0cm,clip]
                	{./HerstellungBilder/MessungenamDektak.png}
                  \caption{Messungen am Dektak}
                \label{fig:Dektak}
            \end{figure}

    		\vspace{2em}

    		Die Tabelle \ref{tab:Lackdickenmess} zeigt das Ergebnis:

    		\vspace{2em}

      		\begin{table}[H]
     		  \begin{addmargin}[3cm]{3cm}
     			\centering
                   \begin{tabular}{|p{3cm}|p{3cm}|p{3cm}|}
         			\hline
         			Wafer & p-Pad & n-Pad\\
         			\hline
        			120502 & 1.5 µm     & 185 nm\\
        			\hline
                    120503 & 1.55 µm 	& 182 nm\\
                    \hline

                    \end{tabular}
              \end{addmargin}
              \caption{Lackdickenmessung}
              \label{tab:Lackdickenmess}
            \end{table}

            \vspace{2em}

			\textbf{Ätzen:}\\
			\\
			Diesmal wurde das Ätzen vorsichtig mehrmals durchgeführt. Dafür
			haben wir 1\%-ige gepuffte Flusssäure BHF ($HF/NH_{3}F$) benutzt.
			Jeder Wafer wurde in dieser Lösung für ca. eine min. eingetaucht und
			sofort mit Di-Wasser abgespült.

			\vspace{2em}

    		\begin{figure}[H]
				\hspace{3 cm}
                  \includegraphics[scale=1, trim = 0cm 0cm 0cm 0cm,clip]
                	{./HerstellungBilder/StrukturnachdemAetzen2.png}
                  \caption{nach dem erneuten Ätzen}
                \label{fig:ernAetz}
            \end{figure}

    		\vspace{2em}

			Zwischen den Ätzungen haben wir ständig die Oxiddicke mit Dektak
			kontrolliert. Die Dektakmessung zeigte uns, dass der Ätzvorgang sehr
			langsam abläuft. Die Kontrolle am Lichtmikroskop hat aber folgendes
			gezeigt:\\
			Wafer03 wurde total überätzt! Starke Dotierung wurde als mögliche
			Ursache dafür vermutet.

			\vspace{2em}

    		\begin{figure}[H]
				\hspace{-0.7 cm}
                  \includegraphics[scale=1, trim = 0cm 0cm 0cm 0cm,clip]
                	{./HerstellungBilder/LichtmikroskopbilderW23.png}
                  \caption{erneute Aufnahme mit dem Lichtmikroskop}
                \label{fig:ernLichtmi}
            \end{figure}

    		\vspace{2em}


    		Da der Wafer03 übergeätzt wurde, haben wir entschieden, dass
    		eine AL-Metallisierung bei ihm nicht mehr sinnvoll ist.\\
			Dagegen Wafer02 war in Ordnung, nur die Kanten waren unscharf(s.
			Abb. \ref{fig:ernLichtmi}).\\
			\\
			\textbf{Lack entfernen:}\\
			\\
			Als letztes wurde Lack von den Wafern 02 und 03 mit dem Aceton
			entfernt.\\
			Nach dem Lithographie-Schritt sieht unsere Struktur folgendermaßen
			aus:

    		\vspace{2em}

    		\begin{figure}[H]
				\hspace{3 cm}
                  \includegraphics[scale=1, trim = 0cm 0cm 0cm 0cm,clip]
                	{./HerstellungBilder/KontaktfenstergeaetztundLackentfernt.png}
                  \caption{Kontaktfenstermaske geätzt und entfernt}
                \label{fig:Konfengeaetzt}
            \end{figure}

    		\vspace{2em}

			Anschließend wurden von allen drei Wafer die organischen Stoffe mit
			dem Ätzmittel Caro's Etch entfernt.



		\subsubsection{3.Tag}


		\textbf{PVD, Metallisieren:}\\
		\\
 		PVD (physical vapour deposition) ist die physikalische
 		Gasphasenabscheidung, die für die vakuumbasierten  Beschichtung mit
 		Aluminium verwendet wird. Die Abbildung \ref{fig:PVD} zeigt die schematische
 		Darstellung eines PVD-Verdampfungsverfahrens.

 		\vspace{2em}

    		\begin{figure}[H]
				\hspace{1.5 cm}
                  \includegraphics[scale=1, trim = 0cm 0cm 0cm 0cm,clip]
                	{./HerstellungBilder/SchematischeDarstellungeinesPVD-Verdampfungsverfahrens.png}
                  \caption{PVD-Verdampfungsverfahren}
                    \begin{center}
                      \small Quelle: http://de.wikipedia.org/wiki/PhysikalischeGasphasenabscheidung
             		\end{center}
                \label{fig:PVD}
            \end{figure}

    	\vspace{2em}

 		Das abzuscheidende Material (Target) liegt in fester Form vor. Mit dem
 		Elektronenstrahl, der von einer Elektronenquelle geschossen
 		werden, werden die Ionen oder Elektronen durch ein elektromagnetisches
 		System auf das Target umgelenkt und das Material wird verdampft.\\
		\\
		Das  Material verdampft  wegen des sehr niedrigen Druckes und der
		Energie der Elektronen. Es bildet sich ein Dampf, der durch elektrische
		Felder durch die Kammer geführt wird. Dabei trifft  der Dampf auf die zu
		beschichtenden Teile und es kommt zu einer Schichtbildung.\\
		Die Elektronenstrahlkanonen setzen ein Vakuum voraus, deswegen wurde das
		Vakuum in der Bedampfungsanlage (s. Abb. \ref{fig:Bedampf}) über die
		Nacht abgepumpt.

 		\vspace{2em}

    		\begin{figure}[H]
				\hspace{0 cm}
                  \includegraphics[scale=1, trim = 0cm 0cm 0cm 0cm,clip]
                	{./HerstellungBilder/Bedampfungsanlage.png}
                  \caption{Bedampfungsanlage}
                \label{fig:Bedampf}
            \end{figure}

    	\vspace{2em}


		Bevor der Wafer in die PVD-Anlage platziert wird, wurde die
		Waferoberfläche noch Mal mit einer  1\%-igen HF-Lösung (HF-Dip) für 15
		Sekunden gereinigt. Diese Lösung dient dazu, die neuentstandene
		Oxidschicht zu entfernen.\\
		Danach wurde unser einziger Wafer02 in den PVD-Apparat mit der
		Vorderseite nach unten eingelegt. Außer dem Einlegen des Wafers wurden
		alle Schritte mit Hilfe der Steuerungstasten automatisch ausgeführt.\\
		Dann wird das Programm eingestellt und der PVD-Prozess fängt an. Die
		aufzusputternde AL-Schicht soll 500 nm betragen.\\
		\\
		Da der PVD-Prozess 30 Minuten dauerte, wurden inzwischen Wafer 01 und 03
		noch Mal mit Phosphorus beschichtet. Anschießend wurden die Wafer 30
		Minuten lang mit $200^{\circ}C$ gebacken. Diese Improvisation war
		notwendig, da die beiden Wafer keine Oxidschicht mehr hatten.\\
		Nach PVD wurde Wafer02 mit einer Salpetersäure zehn Minuten lang
		behandelt.

		\vspace{2em}

    		\begin{figure}[H]
				\hspace{3 cm}
                  \includegraphics[scale=1, trim = 0cm 0cm 0cm 0cm,clip]
                	{./HerstellungBilder/Wafer02NachderMetallisierung.png}
                  \caption{Metallisierung}
                \label{fig:metall}
            \end{figure}

    	\vspace{2em}

    	\textbf{Wafer für den Schrägschliff brechen:}\\
 		\\
		Weiter wurden Wafer 01 und 03 für den Schrägschliffwinkel gebrochen.\\
		Die Wafer wurden zuerst in der Mitte angeritzt. Unter den Wafer an
		dieser angeritzten Stelle haben wir eine alte Probe gelegt und dann
		durch einen leichten Druck den Wafer zum Brechen gebracht. Es ist leicht
		durchzuführen, da Silizium entlang der Kristallrichtung bricht. Die
		entstandenen Proben haben die Abmessungen ca. 5 X 5 mm2. Die Proben
		wurden in den vormarkierten Kästchen gelegt.\\
		\\
		\textbf{Lithographie mit der AL-Maske:}\\
		\\
		Nun werden die typischen Lithographie-Schritte bei leicht abgeänderten
		Bedingungen noch Mal durchgeführt.\\
		\\
		\textbf{Prebake, Belacken, Softbake:} \\
		\\
		Die Flüssigkeitsreste werden entfernt, indem die Wafer 30 min. lang bei
		$140^{\circ}C$ auf der Heizplatte liegen bleiben.\\
		\\
		Danach wird der Wafer 02 mit einem Lack  Az 52/4 beschichtet und
		anschließend zwei Minuten lang bei $95^{\circ}C$ getrocknet. Die Ruhezeit
		beträgt eine Stunde.\\
		\\
		\textbf{Belichten mit Al-Maske:}\\
		\\
		Der Wafer02 wird genau so belichtet, wie oben schon beschrieben wurde
		(s. Lithographie am Tag2),(s. Abb.\ref{fig:belichten3}) Bemerkung: Die
		Al-Maske wurde auch nach AA-Maske justiert und die Belichtungszeit
		beträgt jetzt nur 3 ,7 Sekunden, da Aluminium glänzend ist und Licht
		reflektiert.\\
		\\
 		Danach wurde der Lack mit der Rohm Haas-Lösung entwickelt
 		(Abb. \ref{fig:nachAetzen}).

    	\vspace{2em}

    		\begin{center}
                \begin{tabular}{ll}

                \hspace{-14em}
                    \begin{minipage}{0.8\textwidth}
                        \begin{figure}[H]
                        \hspace{6em}
                            \includegraphics[scale=0.9, trim = 0cm 0cm 0cm
                            0cm, clip]{./HerstellungBilder/BelichtendurchMEMaske.png}
                            \caption{Belichten}
                           \label{fig:belichten3}
                        \end{figure}

                    \end{minipage}
                    \begin{minipage}{0.5\textwidth}

                        \begin{figure}[H]
                        \hspace{-2em}
                            \includegraphics[scale=0.9, trim = 0cm 0cm 0cm
                            0cm, clip]
                            {./HerstellungBilder/StrukturnachdemAetzvorgang.png}
                            \caption{Struktur nach dem Ätzvorgang}
                           \label{fig:nachAetzen}
                        \end{figure}
                    \vspace{-1.5em}

                    \end{minipage}

                \end{tabular}
			\end{center}

    		\vspace{2em}

    		\textbf{Sichtkontrolle:}\\
			\\
			Nun soll der Wafer mit dem Lichtmikroskop untersucht werden. Wenn
			der Lack nicht bis zum Ende entwickelt wurde, kann man die
			Lichtinterferenz sehen, was wir auch in den kleinen Mengen gesehen
			haben.\\
			\\
			Dektak-Messung ergab, dass die Dicke nur 1,36 µm ist, aber wir
			brauchen mindestens 1,5 µm. Deswegen wurde entschieden, noch Mal
			zehn Sekunden lang nachzuentwickeln. Die nächste Messung ergab
			1,35 µm. Danach wurde der Wafer noch mit dem Photometer untersucht
			und es sprach nichts dagegen, dass der Lack nicht entwickelt wurde.\\
			\\
			\textbf{Ätzen:}\\
			\\
			Vor dem Ätzen wird der Wafer noch für fünf Minuten lang auf die
			Heizplatte bei $140^{\circ}C$ gelegt.\\
			Der Ätzvorgang wird diesmal mit einer Ätzmischung aus der
			Phosphorsäure, Essigsäure, Wasser und der Salpetersäure durchgeführt
			($H_{3}PO_{4} +H_{2}O+HNO_{3}$). Es wurde mehrmals in
			Zehner-Intervallen (je zehn Sekunden in der Lösung) geätzt, danach
			wurde schnell in das Di-Wasser eingetaucht und abgespült. Nach der
			Sicht wurde beurteilt, ob die Al-Schicht weggeätzt wurde.
			Beobachtung: viele Bläschen.

			\vspace{2em}

    		\begin{figure}[H]
				\hspace{3 cm}
                  \includegraphics[scale=1, trim = 0cm 0cm 0cm 0cm,clip]
                	{./HerstellungBilder/StrukturnachdemAetzvorgang.png}
                  \caption{Struktur nach dem Ätzvorgang}
                \label{fig:Strukaetz}
            \end{figure}

    		\vspace{2em}

    		\textbf{Kontrolle mit dem Lichtmikroskop:} \\
			\\
			Die Untersuchung mit dem Lichtmikroskop hat uns gute Bilder
			geliefert: Die Kontakte sehen gut aus, nur ein wenig untergeätzt
			(dunkler Rand, s. Abb \ref{fig:darlicht}).

			\vspace{2em}

    		\begin{figure}[H]
				\hspace{0 cm}
                  \includegraphics[scale=1, trim = 0cm 0cm 0cm 0cm,clip]
                	{./HerstellungBilder/Lichtmikroskopbilder4.png}
                  \caption{Darstellung mit dem Lichtmikroskop}
                \label{fig:darlicht}
            \end{figure}

    		\vspace{2em}

			\textbf{Photolack entfernen:}\\
			\\
			Als letzter Schritt wird der Lack entfernt. Diesmal wird er in das
			Aceton für drei Minuten eingetaucht, danach sehr schnell mit Wasser
			gesäubert. Anschließend wurde unser Wafer in einem Ultraschallbad
			gereinigt(Abb.\ref{fig:ultra}).

			\vspace{2em}

    		\begin{center}
                \begin{tabular}{ll}

                \hspace{-14em}
                    \begin{minipage}{0.7\textwidth}
                        \begin{figure}[H]
                        \hspace{7.7em}
                            \includegraphics[scale=1, trim = 0cm 0cm 0cm
                            0cm, clip]{./HerstellungBilder/Ultraschalbad.png}
                            \caption{Ultraschallbad}
                           \label{fig:ultra}
                        \end{figure}

                    \end{minipage}
                    \begin{minipage}{0.5\textwidth}

                        \begin{figure}[H]
                        \hspace{0em}
                            \includegraphics[scale=0.9, trim = 0cm 0cm 0cm
                            0cm, clip]
                            {./HerstellungBilder/LichtmikroskopbildEndstruktur.png}
                            \caption{Lichtmikroskopaufnahme}
                           \label{fig:auflicht}
                        \end{figure}
                    \vspace{-1.5em}

                    \end{minipage}

                \end{tabular}
			\end{center}

    		\vspace{2em}

    		Am Ende haben wir unseren Wafer noch Mal am Lichtmikroskop
    		untersucht. Die Ergebnisse waren gut: Die Kanten sind sehr glatt,
    		sehr gute, klare Struktur (s. Abb. \ref{fig:auflicht}).\\
			\\
			\textbf{Enddickenmessung:}\\
			\\
			Die Aluminiumdickenmessung mit dem Profilometer ergab die Werte
			zwischen 470 nm und 490 nm, was dem theoretisch erwarteten Wert von
			500 nm entsprach.\\
			\\
			\textbf{Tempern:}\\
			\\
			Dieser Temperaturprozess dient der Verbesserung der Kontakte
			zwischen dem Aluminium und dem Substrat. Dafür wurde der Wafer bei
			$400^{\circ}C$ für 30 Minuten im Ofen erhitzt.\\
			\\
			Unser Endergebnis, also ein pn-Übergang, ist schematisch  in der
			Abbildung \ref{fig:end} dargestellt.

    		\vspace{2em}

    		\begin{figure}[H]
				\hspace{2.5 cm}
                  \includegraphics[scale=1, trim = 0cm 0cm 0cm 0cm,clip]
                	{./HerstellungBilder/Endstruktur.png}
                  \caption{Endstruktur}
                \label{fig:end}
            \end{figure}

    		\vspace{2em}


    		\begin{figure}[H]
				\hspace{2.5 cm}
                  \includegraphics[scale=1, trim = 0cm 0cm 0cm 0cm,clip]
                	{./HerstellungBilder/EndergebnisWafer120502.png}
                  \caption{Endstruktur Wafer}
                \label{fig:endwaf}
            \end{figure}

    		\vspace{2em}



    	\subsubsection{4.Tag}


    		Um die pn-Übergangstiefe zu messen wurden unsere Proben zuerst
    		geschliffen (weil die pn-Tiefe nicht direkt mit dem Mikroskop
    		gemessen werden kann).\\
			Jede Probe wird zuerst auf einen Support geklebt, indem man den
			Support auf einer Heizplatte bei $120^{\circ}C$ heizt und der
			spezielle Klebstoff darauf geschmolzen wird (s.Abb. \ref{fig:kleb}).

    		\vspace{2em}

    		\begin{figure}[H]
				\hspace{4.1 cm}
                  \includegraphics[scale=1, trim = 0cm 0cm 0cm 0cm,clip]
                	{./HerstellungBilder/KlebstoffunddieProben.png}
                  \caption{Klebstoff}
                \label{fig:kleb}
            \end{figure}

    		\vspace{2em}

    		Die Proben werden dann mit einem Schleifmittel (Suspension DP aus
    		Diamantpartikeln) fünf Minuten lang geschliffen(s.
    		Abb ref{fig:schlei}), so dass sie am Rand einen Schrägschliff
    		bekommen. Danach werden die Proben abgespült und getrocknet.

   			\vspace{2em}

    		\begin{figure}[H]
				\hspace{4.1 cm}
                  \includegraphics[scale=1, trim = 0cm 0cm 0cm 0cm,clip]
                	{./HerstellungBilder/Schleifen.png}
                  \caption{Schleifen}
                \label{fig:schlei}
            \end{figure}

    		\vspace{2em}

    		\textbf{Dekoration:}\\
			\\
			Zuerst wurde mit der $HF:CH_{3}COOH:HNO_{3}$ (1:4:5) Lösung versucht, die
			Probe zu dekorieren. Es hat nicht funktioniert und deswegen wurde
			entschlossen, es mit der Verkupferung zu versuchen.\\
			\\
			Die Dekoration der Proben wird dann im Reinraum mit einer Lösung aus
			zwei Gramm $CuSO_{4}$ , fünf ml HF(40\%) und 100ml $H_{2}O$ dekoriert. Am
			Anfang des Vorgangs liegen die Proben in der Lösung mit der
			Vorderseite nach oben. Sie werden dann mit zwei "Schwanenhals"-
			Lampen drei bis fünf Sekunden belichtet, damit das Licht den Prozess
			aktivieren kann.

    		\vspace{2em}

    		\begin{figure}[H]
				\hspace{3 cm}
                  \includegraphics[scale=1, trim = 0cm 0cm 0cm 0cm,clip]
                	{./HerstellungBilder/Verkupferung.png}
                  \caption{Verkupferung}
                \label{fig:verk}
            \end{figure}

    		\vspace{2em}

    		Die Kupferlösung wird auf der Oberfläche der Probe an dem
    		angeschliffenen Gebiet aufgetragen. Da die Kupfermischung einen
    		Mangel an den Elektronen hat, lagert sich diese Lösung an dem
    		n-Gebiet, wo es mehr Elektronen gibt an. Deswegen erscheint die
    		n-leitende Seite des pn-Übergangs heller getönt als die p-leitende
    		Seite (s.Abb. \ref{fig:dek}). Anschließend werden die Proben mit dem Aceton
    		gereinigt.

    		\vspace{2em}

    		\begin{figure}[H]
				\hspace{4.2 cm}
                  \includegraphics[scale=1, trim = 0cm 0cm 0cm 0cm,clip]
                	{./HerstellungBilder/DekorationunterdemLichtmikroskop.png}
                  \caption{Dekoration unterm Lichtmikroskop}
                \label{fig:dek}
            \end{figure}

    		\vspace{2em}


			Unter dem Mikroskop ist dieser Unterschied gut sichtbar und lässt
			sich ohne Schwierigkeiten auswerten (es wird die Länge der
			Verkupferung gemessen):\\
			$Wafer01_a: 80\micro m$\\
			$Wafer01_b: 75\micro m$\\

			\textbf{Dektak-Messung:}\\
			\\
			$Wafer01_b:$\\
			\\
			Tiefe 7,0 µm, Länge 300 µm\\
			Tiefe 6,9 µm, Länge 300 µm\\
			\\
			$Wafer01_a$:\\
			\\
			Tiefe 6,2 µm, Länge 300 µm\\
			Tiefe 5,7 µm, Länge 300 µm\\

			Daraus kann man die Junctions-Tiefe ausrechnen:

			\vspace{2em}

    		\begin{figure}[H]
				\hspace{2 cm}
                  \includegraphics[scale=1, trim = 0cm 0cm 0cm 0cm,clip]
                	{./HerstellungBilder/SchematischeDarstellungzumAusrechnenderpnTiefe.png}
                  \caption{Darstellung zur Tiefenbestimmung}
                \label{fig:dartief}
            \end{figure}

    		\vspace{2em}

			Nach  der Beziehung $x_j=sin \alpha$, haben wir folgende Ergebnisse
			ausgerechnet:\\

			$Wafer01_b:$\\
			1,52 µm \\
			1,65 µm \\

			$Wafer01_a:$\\

			1,76 µm\\
			1,7 µm\\


			Die ausgerechneten Werte entsprechen unseren Erwartungen, was ein
			Indiz dafür ist, dass die Prozessschritte richtig gemacht wurden.\\
			\\
			So endet unser Praktikum im Reinraum. \\
			\\
			\textbf{Schlussbetrachtung:}\\
			\\
			Als Ergebnis unserer Tätigkeit haben wir nur einen Wafer mit vielen
			pn-Übergängen angefertigt. Die anderen Wafer haben wir für den
			Schrägschliff verwendet.\\
			\\
			Insgesamt haben alle Teilnehmer durch diese vier Tage im
			Reinraumlabor viel Wissen über die Arbeit im Reinraum, den Umgang
			mit gefährlichen Stoffen, Arbeit am Mikroskop und Profilometer, und
			die Herstellungsprozesse eines pn-Übergangs erworben.\\
			Praktische Arbeit hat unser theoretisches Wissen sehr bereichert.

    		\vspace{2em}

    		\begin{figure}[H]
				\hspace{1.5 cm}
                  \includegraphics[scale=1, trim = 0cm 0cm 0cm 0cm,clip]
                	{./HerstellungBilder/Endbild.png}
                  \caption{geschafft!!!}
                \label{fig:geschafft}
            \end{figure}

    		\vspace{2em}




\end{quote}
%--------------------------------------------------------------------
%--------------------------------------------------------------------

% \section{Kennlinie}
% \begin{quote}
% 
%     Benennung der Dateien:\\
%     Kennlinie_{Wavernummer}_Die_[{Zeile},{Spalte}]_Mess_{messung}.mat\\
% 
%     100 µA Strombegrenzung\\
%     Welche Diode?
%     Welcher Manipulator
% 
%     SMU1 = GNDi
%     SMU2 = GND
%     SMU3 = Var1
% 
%     Unterdiffusion
% 
%     unterspannung
%     0.0 & 2.4
%     0.5 & 3.8
%     0.8 & 4.5
%     1.0 & 5.0
%     1.3 & 5.5
%     1.8 & 6.7
%     2.5 & 8.2
%     3.0 & 9.2
% 
%     \TODO{müssen wir die Dies durchnummerieren?}
% \end{quote} %sec Kennlinie

%--------------------------------------------------------------------
%--------------------------------------------------------------------

\newpage

\section{Schaltverhalten}
\begin{quote}
	
	\subsection{Einführung}

	In diesem Versuch soll Schaltverhalten bei Strom- und Spannungssprüngen 
	untersucht werden. Ein Maß für die Geschwindigkeit mit der die Diode schalten 
	kann ist die Minoritätsträgerlebensdauer. Der Schaltvorgang hält an
	, bis die Minoritäten auf- bzw. abgebaut sind. Daher ist es Ziel dieses 
	Versuches über zwei unterschiedliche Verfahren diese Ladungsträgerlebensdauer 
	zu bestimmen. Dies ist zum einen der Ausschaltvorgang und zum anderen 
	die Stromkommutierung.\\
	\\
	\subsection{Theorie}
	
	Zunächst sollen das ideale und das reale Schaltverhalten gegenüber gestellt
	werden. Das ideale Verhalten charakterisiert sich durch verzögerungs- und
	verlustfreies Schalten. Die zu messenden Dioden zeigen allerdings durch
	Energiespeicher wie die Sperrschicht- und die Diffusionskapazität kein
	ideales Schaltverhalten. \\

	Dabei ist die Charakteristik des Schaltverhaltens davon abhängig, ob es
	sich um ein Spannungs- oder Stromsprung und einen Ein- oder Ausschaltvorgang
	handelt. Um das Verständnis für diese Vorgänge zu verbessern, soll im
	Folgenden näher auf einige Beispiele eingegangen werden.\\

	Bei einem Einschaltstromsprung werden zwei Fälle unterschieden: Die starke
	und die schwache Injektion. Ob starke oder schwache Injektion vorliegt
	richtet sich nach der Ladungsträgeranzahl, die von der einen Seite des
	pn-Überganges als Majoritätsträger auf die andere Seite als Minoritätsträger
	gelangen. Ist die Anzahl der auf der anderen Seite ankommenden nun
	Minoritäten in etwa so groß, oder größer wie die Majoritäten spricht man von
	starker Injekion. Andernfalls spricht man von schwacher Injektion.\\
    Wie in Bild \ref{fig:Stromeinschalten} zu erkennen, spielt in den beiden
    Fällen der Bahnwiderstand eine unterschiedliche Rolle. Nach Shockley ist der
    Bahnspannungsabfall für schwache Injektion zu vernachlässigen. Mit Hilfe der
    Boltzmanfaktoren lässt sich ein logarthmischer Verlauf der Spannung über der
    RLZ herleiten. Dieser ist in der Abbildung \ref{fig:Stromeinschalten} in der
    grob gestrichelten Kennlinie zu erkennen. Man könnte vermuten, dass ohne den
    Bahnwiderstand nur der kapazitive Anteil zur Wirkung kommt und die Spannung
	daher kaum springen darf.\\
	Kommt hingegen der Einfluss des Bahnwiderstandes bei der starken Injektion
	hinzu, dann kann ein heftiger Spannungssprung erfolgen. Der zusammengefasste 
	Spannungsverlauf ist in Bild \ref{fig:Stromeinschalten} an der 
	durchgezogenen Kennlinie zu erkennen. Durch den Bahnwiderstand bekommt die 
	Diode vergleichbar mit Zuleitungen einen induktiven Charakter und die 
	Spannung kann springen.

	\vspace{2em}
    
    \begin{figure}[H]
        \centering
        \includegraphics[scale=1]{./SchaltverhaltenBilder/Stromeinschalten.jpg}
        \caption{Spannungs- und Stromverlauf beim Stromausschaltsprung}
             \begin{center}
                 \small Quelle: Prof. Boit, Clemens Helfmeier, Philipp Scholz: Laborskript Technologie und Bauelemente der Halbleitertechnik (SS 2012), S. 78
             \end{center} 
        \label{fig:Stromeinschalten}
    \end{figure}
    
    \vspace{2em} 
    
    Bei der Stromabschaltung erhält man idealer Weise den Spannungs- und 
    Stromverlauf in Abb. \ref{fig:Stromausschalten}.
   
    \vspace{2em}
    
    \begin{figure}[H]
        \centering
        \includegraphics[scale=1]{./SchaltverhaltenBilder/Stromausschalten.jpg}
        \caption{Spannungs- und Stromverlauf beim Stromausschaltsprung}
             \begin{center}
                 \small Quelle: Prof. Boit, Clemens Helfmeier, Philipp Scholz: Laborskript Technologie und Bauelemente der Halbleitertechnik (SS 2012), S. 80
             \end{center} 
        \label{fig:Stromausschalten}
    \end{figure}
    
    \vspace{2em} 
    
    In Abb. \ref{fig:Stromausschalten} kann man im Spannungsgraph erkennen, 
    dass zum Schaltzeitpunkt die Spannung sprungartig auf einen bestimmten Wert 
    abfällt. Dies ist mit den wegfallenden Bahnwiderständen zu begründen. Über 
    diesen Spannungsabfall kann man dann auch den Bahnwiderstand bestimmen, wenn
    man den Eingangsstrom gemessen hat.\\   
	Der kontinuierliche Spannungsabfall im Anschluss lässt sich über die 
	Bilanzgleichung \ref{eq:bilanz} herleiten.
	
	\begin{equation}
         \begin{split}
             \frac{\delta p}{\delta t}=-\frac{1}{q}\nabla \cdot \vec{j_{p}}-\frac{\Delta p}{\tau_{p}}
         \end{split}
         \label{eq:bilanz}
    \end{equation}
      
    Da die Stromdichte in der Divergenz zum Zeitpunkt des Abschaltens an der 
    Stelle $w_{n}$ zu null gezwungen wird, sich die restlichen Ladungsträger 
    durch Rekombination abbauen und damit die Divergenz der Stromdichte 
    ebenfalls null wird, ergibt sich eine Differenzialgleichung mit der Lösung 
    aus Formel \ref{eq:lsg}, wenn man von lin. Proportionalität zwischen $Q_{s}$ 
    und p in Gleichung \ref{eq:bezQsp} ausgeht.
    
    \begin{equation}
         \begin{split}
             Q_{s}(t)=Q_{s0}exp\Big(-\frac{t}{\tau}\Big)
         \end{split}
         \label{eq:lsg}
    \end{equation}
    
    \begin{equation}
         \begin{split}
             Q_{s}=qAL_{p}p_{n}(w_{n})
         \end{split}
         \label{eq:bezQsp}
    \end{equation}
	
	Aus dieser Proportionalität und der Boltzmannbeziehung ergibt sich die 
	Gleichung \ref{eq:glp}.
	
	\begin{equation}
         \begin{split}
             p_{n}(w_{n},t)=p_{n0}exp\Big(\frac{U_{F}}{U_{T}}\Big)exp\Big(-\frac{t}{\tau}\Big)
         \end{split}
         \label{eq:glp}
    \end{equation}
	
	Wenn man zuletzt noch den umgestellten Boltzmannfaktor aus Gleichung
	\ref{eq:boltzfak} in Gleichung \ref{eq:glp} einsetzt, erhält man Gleichung
	\ref{eq:gerade}, welche einer Geradengleichung entspricht und aus der die 
	Minoritätsträgerlebensdauer ermittelt werden kann.
	
	\begin{equation}
         \begin{split}
             U_{j}=U_{T}ln\Big(\frac{p(w_{n})}{p_{n0}}\Big)
         \end{split}
         \label{eq:boltzfak}
    \end{equation}
	
	\begin{equation}
         \begin{split}
             U(t)=U_{F}-U_{T}\frac{t}{\tau}
         \end{split}
         \label{eq:gerade}
    \end{equation}
	
	Zusätzlich zu diesem Verfahren kann man die Minoritätsträgerlebensdauer auch
	über die Speicherzeit bei der Stromkommutierung gewinnen. Die 
	Stromkommutierung ist ein Verfahren, bei dem wie in Abb. \ref{fig:komm} zwei 
	entgegengesetzt gepolte Stromquellen parallel über einen Schalter zu einer 
	Diode geschaltet werden.  
	
	\vspace{2em}
	
	\begin{figure}[H]
        \centering
        \includegraphics[scale=1]{./SchaltverhaltenBilder/Kommutierschaltung.jpg}
        \caption{Spannungs- und Stromverlauf bei }
             \begin{center}
                 \small Quelle: Prof. Boit, Clemens Helfmeier, Philipp Scholz: Laborskript Technologie und Bauelemente der Halbleitertechnik (SS 2012), S. 81
             \end{center} 
        \label{fig:komm}
    \end{figure}
	
	\vspace{2em}
	
	Die Diode wird also abwechselnd in Sperr- und in Durchlassrichtung betrieben
	. Um die Minoritätsträgerlebensdauer ermitteln zu können wird in diesem Fall
	die Bilanzgleichung zeitabhängig betrachtet (Formel \ref{eq:bilanzzeit}). 
	Hierbei wird die Divergenz in der Bilanzgleichung nicht null und es ergibt 
	sich ein zusätzlicher Term.
	
	\begin{equation}
         \begin{split}
            \frac{\delta Q_{s}}{\delta t}=-\frac{Q_{s}}{\tau}+I
             \end{split}
         \label{eq:bilanzzeit}
    \end{equation}
    
    Wie bei der Stromabschaltung erhält man einen Lösungsansatz wie in Gleichung
    \ref{eq:lsg2}.
	
	\begin{equation}
         \begin{split}
             Q_{s}(t)=I_{F}\tau exp\Big(-\frac{t}{\tau}\Big)-I_{R0}\tau\Big(1-exp\Big(-\frac{t}{\tau}\Big)\Big)
             \end{split}
         \label{eq:lsg2}
    \end{equation}
	
	Man weiß, dass sich die Speicherladung zum Zeitpunkt $t_s$ abgebaut hat und 
	somit null ist. Dies kann man in die Gleichung einsetzen und nach $\tau$
    umstellen und erhält die Gleichung \ref{eq:tsundtau}. 
    
	\begin{equation}
         \begin{split}
             \tau= \frac{t_s}{ln\Big(1+\frac{I_F}{I_{R0}}\Big)}
             \end{split}
         \label{eq:tsundtau}
    \end{equation}
    
    Das Stromverhältnis kann man nun selber einstellen und die Speicherzeit kann 
    aus dem Spannungsverlauf gewonnen werden, der annähernd wie der in Abb. 
    \ref{fig:kommverlauf} aussehen sollte.
	
	\vspace{2em}
	
	\begin{figure}[H]
        \centering
        \includegraphics[scale=0.7]{./SchaltverhaltenBilder/Stromkommutierung.jpg}
        \caption{Spannungs- und Stromverlauf bei Stromkommutierung}
             \begin{center}
                 \small Quelle: Prof. Boit, Clemens Helfmeier, Philipp Scholz: Laborskript Technologie und Bauelemente der Halbleitertechnik (SS 2012), S. 82
             \end{center} 
        \label{fig:kommverlauf}
    \end{figure}
	
	\vspace{2em}
	
	Der kontinuierliche Abfall bis zum Zeitpunkt $t_s$ ergibt sich wie bei der
	Stromabschaltung durch das Ausräumen der in den Bahngebieten noch 
	gespeicherten Minoritäten. Wenn diese ausgeräumt sind erreicht sowohl der 
	Strom den Sperrsättigungsstrom als auch die Spannung die Sperrspannung durch  
	einen exponentieller Abfall.\\
	\\
	Um in diesem Fall den Lösungsansatz für die Differenzialgleichung zu 
	erhalten, ist man davon ausgegangen, dass die Gleichung
	
	\begin{equation}
         \begin{split}
             Q_{s}=I_{F}\tau
             \end{split}
         \label{eq:tsundtau2}
    \end{equation}
	
	Um diese Gleichung zu ermitteln, bedient man sich zuerst der beiden 
	Gleichungen \ref{eq:qs} und \ref{eq:pnnull}. 
	
	\begin{equation}
         \begin{split}
             Q_{s}=eA\int\limits_{w_{n}}^{\infty}  (p_{n}(x)-p_{n0})  \ dt
             \end{split}
         \label{eq:qs}
    \end{equation}
	
	\begin{equation}
         \begin{split}
             (p_{n}(x)-p_{n0})=p_{n0}\Big(exp\Big(\frac{q}{kT}U_{pn}\Big)-1\Big)exp\Big(-\frac{x-w_{n}}{L_p}\Big)
             \end{split}
         \label{eq:pnnull}
    \end{equation}
	
	Man setzt \ref{eq:pnnull} in \ref{eq:qs} ein und erhält Gleichung
	
	\begin{equation}
         \begin{split}
             Q_{s}=eAL_{p}p_{n0}\Big(exp\Big(\frac{q}{kT}U_{pn}\Big)-1\Big)
             \end{split}
         \label{eq:zwischenrech}
    \end{equation}
    
    Als nächstes nutzt man die Formel \ref{eq:termosp}
    
    \begin{equation}
         \begin{split}
             U_{T}=\frac{kT}{q}
             \end{split}
         \label{eq:termosp}
    \end{equation}
	
	und die Formel \ref{eq:stromgl} (entnommen aus dem Laborskript Formel 3.56).
	
	\begin{equation}
         \begin{split}
             j_{p,Diff}(w_{n})=p_{n0}\frac{eD_{p}}{L_p}\Big(exp\Big(\frac{U_{pn}}{U_T}\Big)-1\Big)
             \end{split}
         \label{eq:stromgl}
    \end{equation}
	
	Formel \ref{eq:stromgl} darf man benutzen, da man davon ausgegangen ist, das
	nur die Diffusionsströme betrachtet werden (weil die Ströme von der einen 
	Seite des pn-Übergangs als Feldströme fließen und auf der anderen Seite mit
	der selben Größe als Diffusionsströme weiterfließen und da eine p+n-Diode 
	betrachtet wird und die Diffusionsströme auf der n-Seite vernachlässigt 
	werden können.)\\
	\\
	Damit erhält man die Gleichung \ref{eq:blalbl}.
	
	\begin{equation}
         \begin{split}
             Q_{s}=A\frac{L_{p}^{2}}{D_p}j_{n}(w_n)
             \end{split}
         \label{eq:blalbl}
    \end{equation}
	
	mit den Gleichungen \ref{eq:Stromdichte} und \ref{eq:Diffusionslaenge}
	
	\begin{equation}
         \begin{split}
             I=jA
             \end{split}
         \label{eq:Stromdichte}
    \end{equation}
    
    \begin{equation}
         \begin{split}
             L_{p}=\sqrt{D_{p}\tau_{p}}
             \end{split}
         \label{eq:Diffusionslaenge}
    \end{equation}
    
    erhält man die gesuchte Gleichung \ref{eq:tsundtau2}.\\
   	\\
    Ein weiterer Effekt, den man beobachten kann ist die sogenannte 
    Sperrschichtatmung. Dabei wird die Diode in Sperrrichtung betrieben. Legt 
    man nun unterschiedliche Sperrspannungen an, kann man unterschiedliche
    Weiten der Raumladungszone feststellen. Umso größer die Sperrspannung wird, 
    desto größer wird auch die Raumladungszonenweite. Dies ist an der Formel
    \ref{eq:wrlz} zu erkennen.
    
     \begin{equation}
         \begin{split}
             w_{RLZ}=\sqrt{\frac{2\epsilon_{0}\epsilon_{r}}{e}\bigg(\frac{N_{A}+N_{B}}{N_{A}\cdot N_{B}}\bigg)(U_{D}-U)}
             \end{split}
         \label{eq:wrlz}
    \end{equation}
     
  \subsection{Durchführung und Auswertung}
     
     Um die beiden Verfahren Stromkommutierung und Stromabschaltung umzusetzen, 
     kann man die Schaltung in Abb. \ref{fig:messaufb} nutzen.
	
	\vspace{2em}
	
	\begin{figure}[H]
        \centering
        \includegraphics[scale=0.3]{./SchaltverhaltenBilder/schaltbild.JPG}
        \caption{Schaltung}
             \begin{center}
                 \small Quelle: Schaltung von Michael Sadowski und Clemens Helfmeier
             \end{center} 
        \label{fig:schalt2}
    \end{figure}
	
	\vspace{2em}
	
	Sinn der Schaltung ist vor allen Dingen die sehr geringen Kapazitäten der 
	Diode messen zu können und damit auf die Schaltzeiten zu schließen. Dazu 
	wird ein Operationsverstärker mit einer sehr geringen Eingangskapazität 
	verwendet. Würde man das Oszilloskop, welches eine Eingangskapazität im 
	Bereich von 13 pF hat, direkt davor schalten, könnte man die Kapazität der 
	Diode (im Bereich von 1 pF) nicht fehlerfrei ermitteln und damit auch nicht
	die Minoritätsträgerlebensdauer. Im Labor wurde der nicht invertierende 
	Verstärker mit einer Spannungsverstärkung von 1 genutzt. Die Kapazitäten am 
	Ausgang dienen der Spannungsstabilisierung der 50 Ohm Widerstand der 
	Zuleitungsanpssung.	\\
	Auf der linken Seite wird an dem Anschluss N+ und der Masse die Diode 
	angeschlossen. Am Eingang FG wird eine Rechtecksignal von -5 V bis 5 V mit
	Hilfe des Funktionsgenerators angelegt. Es gibt nun zwei Fälle, die 
	eintreten. Ist die Spannung negativ, kann nur die Diode D1 leiten. Da aber 
	der High-Anschluss am N+ Gebiet anliegt, ist die Diode trotz negativer 
	Spannung in Sperrrichtung gepolt. Mit die Diode D2 verhält es sich genau 
	umgekehrt.\\
	Diese Schaltung befindet sich in der silbernen Box am linken Bildrand.\\
	Es gibt nun die Möglichkeit mit Hilfe eines Drehstellers (schwarzer Steller
	auf der silbernen Box, links) zwischen einem Leerlauf, der für die 
	Stromabschaltung (also Rückwärtsspannung null) und unterschiedlich großen 
	Widerständen für unterschiedlich große Rückwärtsströme für die 
	Stromkommutierung zu schalten. Zum Messaufbau gehören weiterhin ein 
	Oszilloskop um die Spannungwerte aufzunehmen (mit integrierter USB-
	Speicherfunktion), eine Gleichspannungquelle um die Betriebsspannung des
	OPV zu liefern und der Funktionsgenerator um das Rechtecksignals zu 
	erzeugen. Dabei lässt man sich die Spannung über der Diode und das 
	Rechtecksignal anzeigen.
	
	\vspace{2em}
	
	\begin{figure}[H]
        \centering
        \includegraphics[scale=0.15]{./SchaltverhaltenBilder/Messaufbau.jpg}
        \caption{Messaufbau}
             \begin{center}
                 \small Quelle: Schaltung von Michael Sadowski und Clemens Helfmeier
             \end{center} 
        \label{fig:messaufb}
    \end{figure}
	
	\vspace{2em}
	
	Im folgenden werden die gemessenen Bahnwiderstände betrachtet. Zunächst 
	wurden die Dies auf dem Wafer in Spalten und Zeilen eingeteilt. Es gibt 
	12 Zeilen und 12 Spalten. Jeder Die hat die gleichen Grundelemente (s. Abb. 
	\ref{fig:Die}), allerdings ergeben sich fertigungsbedingt Unterschiede. So 
	sind die Dies am Rand meist mehr verunreinigt als die im Zentrum. Deshalb 
	ist es wichtig sich auf dem Wafer Dioden an unterschiedlichen Positionen 
	anzugucken.\\
	
	\vspace{2em}
	
	\begin{figure}[H]
        \centering
        \includegraphics[scale=0.8]{./SchaltverhaltenBilder/Die_uebersicht.jpg}
        \caption{Übersicht über einen Die}
             \begin{center}
                 \small Quelle: Prof. Boit, Clemens Helfmeier, Philipp Scholz: Laborskript Technologie und Bauelemente der Halbleitertechnik (SS 2012)
             \end{center} 
        \label{fig:Die}
    \end{figure}
	
	\vspace{2em}
	
	
	Für den Bahnwiderstand wurde ein Grundelement eines Dies für 
	unterschiedliche Flussströme vermessen. Das Grundelement wurde als Biggi 2 
	bezeichnet und ist die große Struktur unten links in Abb. \ref{fig:Die}. Die 
	Flussströme wurden über Rechteckpulssignale unterschiedlicher Amplitude 
	eingestellt. Die Tabelle \ref{tab:bahnwid} zeigt das Ergebnis dafür.\\
	
	
	\vspace{2em}

      		\begin{table}[H]
     		  \begin{addmargin}[3cm]{3cm}
     			\centering
                   \begin{tabular}{|p{3cm}|p{3cm}|p{3cm}|}
         			\hline
         			Rechteckpulsamlitude & Flussstrom in mA &  Bahnwiderstand in $\Omega$\\
         			\hline
        			20 & 20 & 5.25 \\
        			\hline
                    15 & 10 & 7    \\
                    \hline
                    5 & 5 & 10     \\
                    \hline

                    \end{tabular}
              \end{addmargin}
              \caption{Bahnwiderstand DIE 1 (Zeile 7 Spalte 6) Struktur: Biggi 2}
              \label{tab:bahnwid}
            \end{table}

    \vspace{2em}
	
	Auffällig ist, ist dass sich für gleiche Verhältnisse von Spannung zu Strom
	unterschiedliche Widerstände ergeben. Dies liegt daran, dass für die 
	Bahnwiderstände natürlich die Spannung an der Diode genutzt wird, die aus 
	den gemessenen Kennlinien ermittelt werden. Eine davon ist in Abb.
	\ref{fig:aussgemessen} zu sehen. Irritierend ist, dass die Kennlinie 
	vertikal invertiert ist. Das ist der Polung geschuldet, da das N+ Gebiet an 
	High und das P Gebiet an Low anliegt.\\
	Außerdem zeigt sich bei dem Sprung ein starkes Überschwingen. Dies entsteht 
	wohl durh den Funktionsgenerator.\\ 
	Außerdem ist der Verlauf nach dem Spannungsabfall durch die Bahnwiderstände
	nicht ausschließlich linear. Zuerst steigt, bzw. fällt die Kennlinie nahezu
	linear am, aber zur null hin hat sie eher einen exponentiellen Anstieg, bzw.
	Abfall. Dies könnte daran liegen, dass wenn nur noch wenige Ladungsträger
	in den Bahngebieten gespeichert sind die Beweglichkeit dieser zunimmt und 
	sie schneller durch die Majoritäten durch Rekombination ausgeräumt werden 
	können.\\
	Bei den Bahnwiderständen muss man weiterhin beachten, dass eine 
	Diodenkennlinie in Durchlassrichtung sehr steil ist. Dass heisst für eine 
	große Änderung des Stromes erfolgt nur eine kleine Änderung der Spannung. 
	Damit ist der Bahnwiderstand über das ohmsche Gesetzt antiproportional zu 
	dem Strom, wie man es in Tabelle \ref{tab:bahnwid} erkennen kann.\\
	Des weiteren ist zu erkennen, dass die Bahnwiderstände für eine Diode von 
	einigen µm doch ziemlich groß ist. Dies liegt daran, dass die eigentlichen 
	Bahnwiderstände kaum gemessen werden können, da die Widerstände von den 
	Zuleitungen und Kontakten dazukommen.
	
	\vspace{2em}
	
	\begin{figure}[H]
        \centering
        \includegraphics[scale=0.7]{./SchaltverhaltenBilder/Ausschaltvorgang_bild.jpg}
        \caption{Ausschaltvorgang gemessen} 
        \label{fig:aussgemessen}
    \end{figure}
	
	\vspace{2em}
	
	Im weiteren Verlauf werden auf dem Wafer für zwei Positionen (einmal in der
	Mitte und einmal am Rand) zwei verschiedene Strukturen vermessen. Zu der 
	Struktur Biggi 2 kommt noch die große Fingerstruktur im linken unter 
	Rechteck in der rechten unteren Ecken hinzu (s. Abb. \ref{fig:Die}). Für die 
	Stromabschaltung und den Bahnwiderstand ergaben sich die Messwerte in den 
	Tabellen \ref{tab:bahnMinbigmitt} bis \ref{tab:bahnMinfingaus}. Die 
	Spannungsamplitude des Eingangspulses liegt bei 5 V und damit fließt ein
	Vorwärtsstrom vom 5 mA.
	
	\vspace{2em}

      		\begin{table}[H]
     		  \begin{addmargin}[3cm]{3cm}
     			\centering
                   \begin{tabular}{|p{5cm}|p{5cm}|}
         			\hline
         			Bahnwiderstand &  Minoritätsträgerlebensdauer\\
         			\hline
        			15 $\Omega$ & 0.298 µs \\
        			\hline

                    \end{tabular}
              \end{addmargin}
              \caption{DIE 2 (Zeile 8 Spalte 6) Struktur: Biggi 2}
              \label{tab:bahnMinbigmitt}
            \end{table}

     \vspace{2em}
	

      		\begin{table}[H]
     		  \begin{addmargin}[3cm]{3cm}
     			\centering
                   \begin{tabular}{|p{5cm}|p{5cm}|}
         			\hline
         			Bahnwiderstand &  Minoritätsträgerlebensdauer\\
         			\hline
        			22 $\Omega$ & 1.6 µs \\
        			\hline

                    \end{tabular}
              \end{addmargin}
              \caption{DIE 3 (Zeile 6 Spalte 1) Struktur: Biggi 2}
              \label{tab:bahnMinbigauss}
            \end{table}

     \vspace{2em}

      		\begin{table}[H]
     		  \begin{addmargin}[3cm]{3cm}
     			\centering
                   \begin{tabular}{|p{5cm}|p{5cm}|}
         			\hline
         			Bahnwiderstand &  Minoritätsträgerlebensdauer\\
         			\hline
        			20 $\Omega$ & 10 µs \\
        			\hline

                    \end{tabular}
              \end{addmargin}
              \caption{DIE 2 (Zeile 8 Spalte 6) Struktur: große Finger}
              \label{tab:bahnMinfingmitt}
            \end{table}

     \vspace{2em}
    
      		\begin{table}[H]
     		  \begin{addmargin}[3cm]{3cm}
     			\centering
                   \begin{tabular}{|p{5cm}|p{5cm}|}
         			\hline
         			Bahnwiderstand &  Minoritätsträgerlebensdauer\\
         			\hline
        			15  $\Omega$ & 1,25 µs \\
        			\hline

                    \end{tabular}
              \end{addmargin}
              \caption{DIE 3 (Zeile 6 Spalte 1) Struktur: große Finger}
              \label{tab:bahnMinfingaus}
            \end{table}

     \vspace{2em}
	
	Die erwarteten 1-6 µs werden nahezu erreicht. Bei zwei Werten liegen sie 
	knapp außerhalb dieser Grenzen. Auch die Bahnwiderstände sind alle im 
	Bereich zwischen 15-22 $\Omega$, wie schon bei der Messung zuvor, bei der 
	der Bahnwiderstand für die 5 mA bei 10 $\Omega$ lag. Leider lässt sich keine
	Aussage über die Abhängigkeit des Bahnwiderstands von der Lage auf dem 
	Wafer treffen, dazu wären noch mehr Messungen nötig gewesen.\\
	Auch bei der Minoritätsträgerlebensdauer lässt sich eine Abhängigkeit von 
	der Lage auf dem Wafer erkennen. Allerdings scheint die Lebensdauer für die
	große Fingerstruktur höhere Minoritätsträgerlebensdauern zu erzielen. Dies 
	könnte daran liegen, dass die Kapazität durch die vergrößerte Fläche 
	zwischen P und N höher ist und damit auch die Speicher der Minoritäten 
	größer werden.\\
	\\
	Für die Kommutierung bekommt man, je nachdem welches Verhältnis man für 
	Vorwärts- zu Rückwärtsstrom einstellt unterschiedliche Kennlinien für eine 
	Struktur. Diese sind widerum vertikal invertiert. Es ist gut zu erkennen, 
	dass der Abfall bis zum Nulldurchgang der Spannung in der Realität wieder 
	nicht linear ist. Allerdings wird wie erwartet die Speicherzeit für ein 
	kleineres Verhältnis von Vorwärts- zu Rückwärtsstrom kleiner, da die 
	Minoritätsträgerlebensdauer konstant sein sollte (nach Formel 
	\ref{eq:tsundtau}).
	
	
	\vspace{2em}
	
	\begin{figure}[H]
        \centering
        \includegraphics[scale=0.7]{./SchaltverhaltenBilder/Kommutierung_bild.jpg}
        \caption{Graphen für die Kommutierung} 
        \label{fig:kommgraph}
    \end{figure}
	
	\vspace{2em}
	
	Die Messwerte für die Strukturen Biggi 2 und die große Fingerstruktur sind
	in den Tabellen \ref{tab:grossetab1} und \ref{tab:grossetab2} zu sehen.\\
	
	 \vspace{2em}

      		\begin{table}[H]
     		  \begin{addmargin}[-0.5cm]{3cm}
     			\centering
                   \begin{tabular}{|p{5cm}|p{5cm}|p{5cm}|}
         			\hline
         			Lebensdauer bei Verhältnis &  $\tau$ von DIE 2 (Zeile 8 Spalte 6) & $\tau$ von DIE 3 (Zeile 6 Spalte 1)\\
         			\hline
        			2:1 & 27 \micro s &  8,3 \micro s\\
        			\hline
        			1:1 & 36 \micro s &  11,5 \micro s\\
        			\hline
        			1:2 & 49 \micro s &  17,8 \micro s\\
        			\hline
        			1:4 & 71 \micro s &  27,8 \micro s\\
        			\hline
        			1:6 & 136 \micro s & 55,6 \micro s\\
        			\hline

                    \end{tabular}
              \end{addmargin}
              \caption{Minoritätsträgerlebensdauer für Biggi 2}
              \label{tab:grossetab1}
            \end{table}

     \vspace{2em}

      		\begin{table}[H]
     		  \begin{addmargin}[-0.5cm]{3cm}
     			\centering
                   \begin{tabular}{|p{5cm}|p{5cm}|p{5cm}|}
         			\hline
         			Lebensdauer bei Verhältnis &  $\tau$ von DIE 2 (Zeile 8 Spalte 6) & $\tau$ von DIE 3 (Zeile 6 Spalte 1)\\
         			\hline
        			2:1 & 18 \micro s &  6,3 \micro s\\
        			\hline
        			1:1 & 23 \micro s &  8,6  \micro s\\
        			\hline
        			1:2 & 34,9 \micro s &  14,0 \micro s\\
        			\hline
        			1:4 & 51,5 \micro s &  23,3 \micro s\\
        			\hline
        			1:6 & 104 \micro s & 52,0 \micro s\\
        			\hline

                    \end{tabular}
              \end{addmargin}
              \caption{Minoritätsträgerlebensdauer für große Fingerstruktur}
              \label{tab:grossetab2}
            \end{table}

     \vspace{2em}
      	
     Bei dieser Messreihe stehen mehr Messwerte zur Verfügung. Es ist auffällig, 
     dass für ein kleineres Verhältnis von Vorwärts- zu Rückwärtsstrom die 
     Lebensdauer zu nimmt. Idealer Weise sollte dies aber konstant bleiben.\\
     Die Werte liegen außerdem teilweise deutlich außerhalb des erwarteten
     Bereichs von 1-6 $\micro s$. \\
     Bei diesem Messverfahren ist es deutlich zu erkennen, dass es einen 
     Unterschied zwischen den Dies in der Mitte und denen am Rand. Die am Rande 
     liegenden weisen deutlich kleinere Minoritätsträgerlebensdauern auf, was 
     auf Verunreinigungen in den am Rande gelegenen Gebieten schließen lässt.  	
	
	 \subsection{Zusammenfassung}
	 
	 Dies war der erste Versuchslauf für diese Schaltung. Dafür hat sie 
	 scheinbar schon Werte geliefert, die den erwarteten naheliegen. Allerdings
	 gibt es noch einige Dinge, die man optimieren kann. So wurde zum Beispiel 
	 ein BNC Kabel als Verbindung zwischen Schaltung und Diode benutzt, welches
	 eine relativ hohe Kapazität aufweist. Der Messaufbau wäre auch ohne dieses
	 Kabel möglich.\\
	 Um die Spannung noch besser aufzulösen, könnte man außerdem die 
	 Verstärkung des nicht invertierenden Verstärkers erhöhen.\\
	 Die Anzahl der Messungen sollte man auch erhöhen, damit die Messwerte auch
	 wirklich aussagekräftig sind und es wäre sinnvoll, wenn es die Zeit erlaubt
	 , noch mehr unterschiedliche Strukturen zu vermessen, um den Einfluss der 
	 Größe und der Form auf die Bahnwiderstände und die Lebensdauern zu 
	 untersuchen.\\
	 Eine interessante Frage ist auch, warum die Lebensdauern für ein kleiner 
	 werdendes Verhältnis von Vorwärts- zu Rückwärtsstrom zunehmen, obwohl sie 
	 konstant sein sollten.
	 
	 

	
%    \newpage
% 
%       \begin{table}[h]
%                \begin{addmargin}[-1cm]{3cm}
%                \centering
%                     \begin{tabular}{|p{5cm}|p{11.2cm}|}
%          \hline
%          Messung & Werte\\
%          \hline
%          Ausschaltvorgang & $I_{F}=10, 50, 100, 150 mA$\\
% 
%          \hline
%          Stromkommutierung & $I_{R0}=15,30,60,90,120 mA$\\
% 
%          \hline
% 
% 
%                     \end{tabular}
%                 \end{addmargin}
%              \caption{Messarten}
%           \label{Messarten}
%       \end{table}
% 
%       \vspace{2em}
% 
%       \begin{table}[h]
%      \begin{addmargin}[-1cm]{3cm}
%      \centering
%                      \begin{tabular}{|p{3cm}|p{3cm}|p{10.2cm}|}
%          \hline
%          Messung & Die/Diode/Wafer & gemessen\\
%          \hline
%          $R_{B}$ für $I_{F}=10 mA$ & & \\
%                                  & & \\
%                                  & & \\
%                                  & & \\
%                                  & & \\
%                                  & & \\
%                                  & & \\
%                                  & & \\
%                                  & & \\
%                                  & & \\
%                                  & & \\
%                                  & & \\
%                                  & & \\
%          \hline
%          $R_{B}$ für $I_{F}=50 mA$ & & \\
%                                  & & \\
%                                  & & \\
%                                  & & \\
%                                  & & \\
%                                  & & \\
%                                  & & \\
%                                  & & \\
%                                  & & \\
%                                  & & \\
%                                  & & \\
%                                  & & \\
%                                  & & \\
%          \hline
%          $R_{B}$ für $I_{F}=100 mA$ & & \\
%                                  & & \\
%                                  & & \\
%                                  & & \\
%                                  & & \\
%                                  & & \\
%                                  & & \\
%                                  & & \\
%                                  & & \\
%                                  & & \\
%                                  & & \\
%                                  & & \\
%                                  & & \\
%          \hline
%          $R_{B}$ für $I_{F}=150 mA$ & &\\
%                                  & & \\
%                                  & & \\
%                                  & & \\
%                                  & & \\
%                                  & & \\
%                                  & & \\
%                                  & & \\
%                                  & & \\
%                                  & & \\
%                                  & & \\
%                                  & & \\
%                                  & & \\
% 
%          \hline
% 
%                      \end{tabular}
%                  \end{addmargin}
%              \caption{Messwerte}
%            \label{Messwerte1}
%         \end{table}
% 
%        \vspace{2em}
% 
%        \begin{table}[h]
%      \begin{addmargin}[-1cm]{3cm}
%      \centering
%                       \begin{tabular}{|p{3cm}|p{3cm}|p{10.2cm}|}
%          \hline
%          Messung & Die/Diode/Wafer & gemessen\\
%          \hline
%          $\tau$ für $I_{F}=150 mA$ & & \\
%                                  & & \\
%                                  & & \\
%                                  & & \\
%                                  & & \\
%                                  & & \\
%                                  & & \\
%                                  & & \\
%                                  & & \\
%                                  & & \\
%                                  & & \\
%                                  & & \\
%                                  & & \\
%          \hline
%          $t_{s}$ & & \\
%                                  & & \\
%                                  & & \\
%                                  & & \\
%                                  & & \\
%                                  & & \\
%                                  & & \\
%                                  & & \\
%                                  & & \\
%                                  & & \\
%                                  & & \\
%                                  & & \\
%                                  & & \\
%          \hline
%          $Q_{s}$ & &\\
%                                  & & \\
%                                  & & \\
%                                  & & \\
%                                  & & \\
%                                  & & \\
%                                  & & \\
%                                  & & \\
%                                  & & \\
%                                  & & \\
%                                  & & \\
%                                  & & \\
%                                  & & \\
%          \hline
%          $\tau$ & &\\
%                                  & & \\
%                                  & & \\
%                                  & & \\
%                                  & & \\
%                                  & & \\
%                                  & & \\
%                                  & & \\
%                                  & & \\
%                                  & & \\
%                                  & & \\
%                                  & & \\
%                                  & & \\
%          \hline
% 
%                            \end{tabular}
%                     \end{addmargin}
%              \caption{Messwerte}
%          \label{Messwerte2}
%       \end{table}
% 
%       \vspace{2em}
% 
%       \begin{table}[h]
%                  \begin{addmargin}[-1cm]{3cm}
%      \centering
%                      \begin{tabular}{|p{3cm}|p{3cm}|p{10.2cm}|}
%          \hline
%          Messung & Die/Diode/Wafer & gemessen\\
%          \hline
%          Tempertatur & & \\
%                                  & & \\
%                                  & & \\
%                                  & & \\
%                                  & & \\
%                                  & & \\
%                                  & & \\
%                                  & & \\
%                                  & & \\
%                                  & & \\
%                                  & & \\
%                                  & & \\
%                                  & & \\
%          \hline
%          Vorwärtsstrom & & \\
%                                  & & \\
%                                  & & \\
%                                  & & \\
%                                  & & \\
%                                  & & \\
%                                  & & \\
%                                  & & \\
%                                  & & \\
%                                  & & \\
%                                  & & \\
%                                  & & \\
%                                  & & \\
%          \hline
%          Rückwärtsstrom & &\\
%                                  & & \\
%                                  & & \\
%                                  & & \\
%                                  & & \\
%                                  & & \\
%                                  & & \\
%                                  & & \\
%                                  & & \\
%                                  & & \\
%                                  & & \\
%                                  & & \\
%                                  & & \\
%          \hline
%                      \end{tabular}
%                   \end{addmargin}
%              \caption{Sonstige Angaben zur Messung}
%          \label{AngabenZurMessung}
%       \end{table}
% 
%       \noindent
% 
%        \vspace{2em}

\end{quote} %sec Schaltverhalten


%--------------------------------------------------------------------
%--------------------------------------------------------------------
\newpage

\section{Emissionsmessung}
\begin{quote}

    Eine Emissionsmessung ist in der Halbleitertechnologie insofern interessant,
    weil sie Erkenntnisse über charakteristische Eigenschaften des Halbleiters,
    in unserem Fall die selbst hergestellte Diode, liefert. Dazu gehören die
    Ladungsträgerlebensdauer $\tau_{n,p}$ und die Diffusionslänge $L_{n,p}$. Auf
    die Lebensdauer lässt sich mithilfe der Diffusionslänge und des
    Diffusionskoeffizienten schließen. Dieser ist ein materialabhängiger Wert,
    welcher von uns nicht weiter beachtet wird. Der folgende Zusammenhang hilft
    bei der Berechnung von $\tau_{n,p}$:

    \begin{equation*}
        \begin{split}
            L_{n,p} = \sqrt{\tau_{n,p} \cdot D_{n,p}}
        \end{split}
    \end{equation*}

    Ziel unserer Messung aber war die Bestimmung der Diffusionslänge in unserer
    Diode. Dieser wurde über die realisierte Intensitätsmessung anhand der
    Emissionen in der Diode ermittelt. Dabei wurden folgende
    Proportionalitätsverhätnisse verwendet:

    \begin{equation*}
        \begin{split}
            I(x) \sim \ \Delta n \sim \ exp(-\frac{x}{L_n})
        \end{split}
    \end{equation*}

    Hierbei werden Strahlungsintensität ins Verhältnis mit der\\
    Minoritätsüberschussladungträgerkonzentration und dieser wiederum
    ins Verhältnis mit einem Exponentialtherm gesetzt. Daher kann man die
    Intensitätsmessung direkt mit diesem Therm in Verbindung setzen,
    welcher in seinem Argument die gesuchte Diffusionslänge beinhaltet. Weiteres
    zur Berechnung des $L_n$ steht in der Auswertung der Messung.\\

    Um aber die Intensitätsmessung verstehen zu können müssen einige
    grundlegende Theorien der Halbleiter bezüglich ihrer Typen und ihrer
    Rekombinationsarten behandelt und nachvollzogen werden.\\
    Daher gibt es vor der Versuchsdurchführung und der Auswertung zunächst eine
    kleine Exkursion in den theoretischen Bereich.

        \subsection{Direkter und indirekter Halbleiter }
        \begin{quote}

        Es gibt zwei Arten von Halbleitern, die direkten und die indirekten
        Halbleiter. Diese unterscheiden sich darin, dass die Rekombination eines
        Ladungsträgers aus dem Leitungs- in das Valenzband unterschiedliche
        Vorraussetzungen erfordert.

            \subsubsection{direkter Halbleiter}
            \begin{quote}
            Die Rekombination bei einem direkten Halbleiter ist relativ simpel.
            Ein freies Elektron braucht dabei nur, unter Abgabe der jeweiligen
            Energie, den Bandabstand zwischen Leitungs- und Valenzband zu
            überqueren.

            \begin{figure}[H]
                    \centering
                        \includegraphics[scale=0.72, trim = 1cm 0cm 1.5cm 0cm,
                        clip]{./Emissionsbilder/restliches/direkt.png}
                        \caption{Rekombination bei einem direkten Halbleiter}
                            \label{fig:./Emissionsbilder/restliches/direkt.png}
            \end{figure}


            \end{quote}

            \subsubsection{indirekter Halbleiter}
            \begin{quote}
            Die Rekombination bei einem indirekten Halbleiter erfordert neben
            einem Energieunterschied auch einen Impulsunterschied, damit das
            Elektron auf dem Valenzband auftreffen kann. Dieser
            Impulsunterschied ist in der folgenden Abbildung zu erkennen.

            \begin{figure}[H]
                    \centering
                        \includegraphics[scale=0.73, trim = 1cm 0cm 1.5cm 0cm,
                        clip]{./Emissionsbilder/restliches/indirekt.png}
                        \caption{Rekombination bei einem indirekten Halbleiter}
                            \label{fig:./Emissionsbilder/restliches/indirekt.png}
            \end{figure}

            \TODO{Bildquellen, TBH-Skript S.92 einfügen}
            \end{quote}

            Ein weiterer Unterschied zwischen direkten und indirekten
            Halbleitern ist der, dass direkte Halbleiter hauptsächlich
            strahlend rekombinieren, wohingegen indirekte Halbleiter
            stochastisch betrachtet fast nur nichtstrahlend rekombinieren.
            Dennoch findet mit sehr geringer Wahrscheinlichkeit auch vereinzelt
            strahlende Rekombination in den indirekten Halbleitern statt.\\
            Bevor weiter darauf eingegangen wird, in wie fern dies für die
            Emissionsmessung von Bedeutung ist, werden die Rekombinationsarten,
            strahlend und nichtstrahlend, wiederholt.

        \end{quote}

        \subsection{Rekombinationsmechanismen}
        \begin{quote}

        Eine Rekombination erfolgt stets unter Abgabe von einer
        Energiedifferenz. Dabei wird unterschieden, ob diese Energie in Form
        eines Lichtquants oder von Wärme abgegeben wird, wodurch auch die
        Beschreibung strahlend oder nichtstrahlend entsteht.\\
        Nun folgen je ein Beispiel für diese Machanismen.

            \subsubsection{strahlende Rekombination}
            \begin{quote}
 
            Die strahlende Rekombination, hauptsächlich bei direkten Halbleitern
            zu sehen, besagt, dass der rekombinierende Ladungsträger die
            Energiedifferenz, welche er zurücklegt, in Form eines Photons
            freigibt. Die Energie dieses Photons beträgt genau die Energie des
            Bandabstands zwischen den beiden Bändern. Anders ausgedrückt kann
            man sie auch Rekombinationsenergie nennen:

            \begin{equation*}
                \begin{split}
                    W_{Rek} = h \cdot \nu
                \end{split}
            \end{equation*}

            Diese Rekombinationsenergie setzt sich aus des Produkt aus dem
            Planckschen Wirkungsquantums $h$ und die Frequenz des entstehenden
            Lichts $\nu$.\\

            Ein Beispiel für die strahlende Rekombination ist die
            Band-Band-Rekombination, welche in der folgenden Abbildung
            dargestellt wurde:

            \begin{figure}[H]
                    \centering
                        \includegraphics[scale=0.68, trim = 1cm 0cm 1.5cm 0cm,
                        clip]{./Emissionsbilder/restliches/bandband.png}
                        \caption{strahlende Band-Band-Rekombination}
                            \label{fig:./Emissionsbilder/restliches/direk2.png}
            \end{figure}

            Man erkennt ein Elektron, welches beim Rekombinieren, die bereits
            erwähnte Rekombinationsernergie in Form eines Photons abgibt. Es
            kann aber auch vorkommen, dass die abgegebene Energie an ein
            weiteres Eektron im Leitungsband abgegeben wird, wodurch dieser auf
            ein höheres Energieniveau im Leitungsband angehoben und wieder runterfallen
            kann. Diese Möglichkeit ist bei der strahlenden Rekombination die
            unwarscheinlichere Variante.

            \end{quote}

            \subsubsection{nichtstrahlende Rekombination}
            \begin{quote}

            Die nichtstrahlende Rekombination findet hauptsächlich bei
            indirekten Halbleitern statt. Dabei wird die Rekombinationsenergie
            an ein weiteres Elektron im Leitungsband abgegeben, welches auf ein
            höheres Energieniveau angehoben wird und unter Abgabe von
            thermischer Energie wieder runterfallen kann.\\
            Als ein Beispiel für die nichtstrahlende Rekombination wird die
            Augerrekombination dargestellt:

            \begin{figure}[H]
                    \centering
                        \includegraphics[scale=0.78, trim = 1cm 0cm 1.5cm 0cm,
                        clip]{./Emissionsbilder/restliches/auger.png}
                        \caption{nichtstrahlende Auger-Rekombination}
                            \label{fig:./Emissionsbilder/restliches/auger.png}
            \end{figure}

            \TODO{Bildquellen einfügen: TBH-Skript S.93,94}
            \end{quote}

        Die für die Emissiosmessung verwendete Diode besteht aus Silizium,
        welcher ein indirekter Halbleiter ist und dessen Rekombinationen
        hauptsächlch nichtstrahlend sind. Wie aber schon bei den Halbleitertypen
        erwähnt, kann es bei indirekten Halbleitern auch mit sehr geringer
        Wahrscheinlichkeit zu strahlender Rekombination kommen. Da die Abgabe
        von thermischer Energie zu einem Temperaturunterschied in der Diode
        führen würde, bräuchte man bei den Emissionsmessung sehr feine und genau
        Thermometer, welche im $\mu m$-Bereich nicht realisierbar sind. Daher
        wird die stochastisch in geringer Menge vorhandene strahlende
        Rekombination betrachtet um über die mit lichtempfindlichen
        Kameras gemessene Lichtintensitäten auf die gesuchte Diffusionslänge
        schließen zu können.

        \end{quote}

        \vspace{1.5em}

        Emission entsteht bei der Rekombination eines freien Ladungsträgers,
        welches nur in Durchlassrichtung bei einer Diode erwartet wird. Dabei
        werden die Majoritäten ins entgegen gesetztes Gebiet injeziert und
        rekombinieren dort als Minoritäten mit den oppositär gepolten
        Ladungsträgern. Die dabei freigesetzte Energie kann dann als Emission
        wargenommen werden. Emission kann aber auch in Sperrrichtung entstehen.
        Die sogennante Feldemission findet dabei ausschließlich in der
        ausgebreiteten Raumladungszone statt. Sobald eine Sperrspannung an der
        Diode angelegt wird, werden die freien Ladungsträger aus der
        Raumladungszone gesaugt und erfahren dabei eine kinetische Energie,
        dessen Stärke von der Größe der Sperrspannung abhängt. Während der
        beschleunigten Bewegung der Ladungsträger aus der Raumladungszone,
        können diese mit Gitteratomen zusammenprallen und Elektronen aus diesem
        Atom auf ein höheres Energieniveau anheben. Wie bei der nichtstrahlenden
        Rekombination entsteht bei diesem Vorfall hauptsächlich thermische
        Energie,es kann aber auch, mit einer sehr kleinen wahrscheinlichkeit
        eine strahlende Rekombination vorkommen.\\
        Die Feldemission ist im Vergleich zu der Emission in Flussrichtung
        unbedeutend klein. Daher ist für die Intensitätsmessung nur eine
        Emission in Flussrichtung relevant.

        \vspace{1.5em}

        \subsection{Messaufbau}
        \begin{quote}

        Die Emissionsmessung erfolgt, wie in der Einleitung erwähnt, durch eine
        Intensitätsmessung während der Rekombinationen in Flussrichtung der
        Diode. Die Messung wird mithilfe des Phemos 1000 durchgeführt, welches
        ein Prüfgerät in der Halbleitertechnik ist und hauptsächlich für
        Fehlerdetektion dient. Um diese geringe Lichtintensität
        messen zu können, braucht man eine lichtempfindliche CCD Kamera. Diese
        ist im Phemos integriert und muss während der gesamten Messung auf
        $-50$°C gekühlt werden. So werden thermische Rauscheinflüsse der sehr
        empfindlichen Kamera vermieden um das Ergebnis nicht zu verfälschen.\\

        Der Wafer, mit mehreren Diodenstrukturen, wird in dem Innenraum des
        Phemos auf einer Vakuumplatte fixiert, sodass der erwünschte Bereich des
        Wafers mit dem passenden Objektiv vergrössert werden kann. Diese
        Vergrößerung hilft bei der Kontaktierung des p- und des n-Pads der Diode
        anhand Nadelspitzen, welche in der unteren Abbildung im Licht des
        Mikroskops zu erkennen sind.

        \begin{center}
                \begin{tabular}{ll}

                \hspace{-8em}
                    \begin{minipage}{0.6\textwidth}

                        \begin{figure}[H]
                            \label{fig:}
                            \includegraphics[scale=0.7, trim = 0cm 0cm 0cm
                            0cm, clip]{./Emissionsbilder/restliches/phemos1.JPG}
                            %FIXME [width=640px,
                             %height=474px]
                            \caption{Innenraum des Phemos}
                        \end{figure}

                    \end{minipage}
                    \begin{minipage}{0.6\textwidth}

                        \begin{figure}[H]
                            \label{fig:adsfgdsfgefgsd}
                            \includegraphics[scale=0.7, trim = 0cm 0cm 0cm
                            0cm, clip]{./Emissionsbilder/restliches/phemos2.JPG}
                            %FIXME [width=640px,
                             %height=474px]
                            \caption{kontaktierte Diode auf dem Wafer}
                        \end{figure}
                    \vspace{-1.5em}

                    \end{minipage}

                \end{tabular}
                \end{center}

        \vspace{2em}

        Anhand eines Live-Bilds der CCD Kamera, kann man bis zum Start der
        Messung den fokussierten Bereich bepannungsquelle für den
        Durchlassbetrieb der Diode wurde der HP4145A, ein sensibeles
        Analysegerät für Halbleiterbauelemente, verwendet, mit welchem eine
        konstante Spannung für eine einstellbare Zeitspanne eingestellt werden
        kann. Somit konnte man fest davon ausgehen, dass die Diode ohne einen
        Spannungseinbruch konstant in Durchlassrichtung betrieben wurde.\\
        Die Messdauer der Emission wurde mithilfe der Phemos-Software
        eingestellt. Mit welchen Durchlassspannungen und wie lange gemessen
        wurde, wird in der Auswertung der Messergebnisse angegeben.

        \vspace{1em}

        Am Ende jeder Messung wurde noch eine Dark-Substraction durchgeführt.
        Bei dieser Dark-Substraction wird die Durchlassspannung abgeschaltet,
        wodurch die Diode nicht mehr als aktiv betrachtet wird. Die Emissionsmessung wird mit der
        gleichen Messdauer und deaktivierter Diode nochmal durchgeführt und von
        der eigentlichen Emissionsmessung subtrahiert, damit jegliche
        Rauscheinflüsse des Phemos selbst aus den Messergebnissen ausgeschlossen
        werden können.

        \end{quote}

        \subsection{Messergebnisse}
        \begin{quote}

        Die Active-Area- und Kontaktfenstermasken wurden so erstellt, dass der
        Wafer am Ende der Produktion Dioden auf jedem Die besaß, die für diese
        Emissionsmessung hergestellt wurden. Diese Emissions-Dioden haben die
        Besonderheit, dass zwischen p- und n-Pad das Substrat nicht mit Siliziumoxid beschichtet
        ist, damit an den pn-Übergang zwischen n-Gebiet und p-Substrat
        unverfälschte Emission gemessen werden kann.\\
        Man hätte annehmen können, dass dort, wo Oxid auf dem pn-Übergang lag,
        keine Emission aus dem Wafer austreten und gemessen werden kann. Die Praxis bewies aber das
        Gegenteil. Unabhängig von der Auswertung der Messung
        konnte an den pn-Übergängen mit Siliziumoxid darüber eine viel stärkere
        Emission gemessen werden. Der Grund dafür liegt in dem Brechungsindex
        und der Dicke der Oxidschicht, denn durch die Reflexionen der Strahlung an
        den Übergängen von Silizium zu Siliziumoxid und Siliziumoxid zur Luft
        werden die Strahlen so reflektiert, dass konstruktive Interferenz
        zwischen den Lichtwellen entsteht und somit die Emission als viel
        stärker wahrgenommen wird. Da dies nicht die wirkliche Intensität der
        Emission am pn-Übergang unserer Diode wiederspiegelt, wurden diese mit
        Oxid überlagerten Stellen in der Emissionsmessung vernachlässigt.\\

        Gemessen wurde nur der Bereich von der Grenze des n-Gebiets bis in
        das p-Substrat hinein. Als Hilfe dafür wurde das jeweilige Emissionsbild
        der Phemos-Software verwendet. Diese zeigte mit farblichen Unterschieden
        (rot = starke Emission, blau = schwache Emission) wo der sinnvolle
        Messbereich begann und endete. Die folgenden Abbildungen zeigen die
        Diode vor und nach der Emissionsmessung auf dem Die in Zeile $6$ und
        Spalte $3$ des Wafers. Zunächst wurde eine große, dann eine kleinere
        Emissions-Diode ausgemessen. Bei beiden betrug die Durchlassspannung
        $1,2\ V$ bei $25\ mA$ und einer Messdauer von $4s$.

         \begin{center}
                \begin{tabular}{ll}

                \hspace{-10em}
                    \begin{minipage}{0.6\textwidth}

                        \begin{figure}[H]
                            \label{fig:dfgsddsgf}
                            \includegraphics[scale=0.25, trim = 0cm 0cm 0cm
                            0cm,
                            clip]{./Emissionsbilder/eins/nach_Kontaktierung_vorMessung.jpg}
                            %FIXME [width=640px, height=474px]
                            \caption{kontaktierte Diode, Live-Bild der CCD
                            Kamera}
                        \end{figure}

                    \end{minipage}
                    \begin{minipage}{0.6\textwidth}

                         \begin{figure}[H]
                            \label{fig:asdfasdf}
                            \includegraphics[scale=0.25, trim = 0cm 0cm 0cm
                            0cm,
                            clip]{./Emissionsbilder/eins/nach_Emission_mit_Distanzen.jpg}
                            %FIXME [width=640px, height=474px]
                            \caption{selbe Diode nach der Emissionsmessung}
                        \end{figure}
                   \vspace{-1.5em}

                    \end{minipage}

                \end{tabular}
                \end{center}

        \vspace{2em}

        Der gelbe Strich wurde anhand der Maus gezogen und zeigt den Bereich im
        Substrat, welcher genauer ausgewertet wurde. Mehr dazu folgt in der
        Auswertung.\\

        Als nächstes folgen die Abbildungen der kleineren Emissions-Diode auf
        dem selben Die:


         \begin{center}
                \begin{tabular}{ll}

                \hspace{-10em}
                    \begin{minipage}{0.6\textwidth}

                        \begin{figure}[H]
                            \label{fig:asdfsdfasdgasfdgsadg}
                            \includegraphics[scale=0.25, trim = 0cm 0cm 0cm
                            0cm,
                            clip]{./Emissionsbilder/zwei/nack_Kontaktierung.jpg}
                            %FIXME [width=640px, height=474px]
                            \caption{kontaktierte Diode, Live-Bild der CCD
                            Kamera}
                        \end{figure}

                    \end{minipage}
                    \begin{minipage}{0.6\textwidth}

                         \begin{figure}[H]
                            \label{fig:asdfasdfasdfggg}
                            \includegraphics[scale=0.25, trim = 0cm 0cm 0cm
                            0cm,
                            clip]{./Emissionsbilder/zwei/nach_Emissionsmessung_Intensitat_Distanz.jpg}
                            %FIXME [width=640px, height=474px]
                            \caption{selbe Diode nach der Emissionsmessung}
                        \end{figure}
                   \vspace{-1.5em}

                    \end{minipage}

                \end{tabular}
                \end{center}

        \vspace{2em}

        Nun wurde noch ein Die am Rand des Wafers ausgemessen um nach
        Unterschieden zwischen den Messergebnissen kontrolliert werden zu
        können. Auch auf dem Die in Zeile $15$ und Spalte $8$ wurde erst eine
        große, dann eine kleine Emissions-Diode untersucht. Die Durchlassspannung bei der
        großen Diode betrug $1,2\ V$ bei einem Strom von $25\ mA$. Die Messdauer
        wurde aus $3s$ variiert.


         \begin{center}
                \begin{tabular}{ll}

                \hspace{-10em}
                    \begin{minipage}{0.6\textwidth}

                        \begin{figure}[H]
                            \label{fig:rtaretewtrwer}
                            \includegraphics[scale=0.25, trim = 0cm 0cm 0cm
                            0cm,
                            clip]{./Emissionsbilder/drei/nach_Kontaktierung.jpg}
                            %FIXME [width=640px, height=474px]
                            \caption{kontaktierte Diode, Live-Bild der CCD
                            Kamera}
                        \end{figure}

                    \end{minipage}
                    \begin{minipage}{0.6\textwidth}

                         \begin{figure}[H]
                            \label{fig:bvxcvx}
                            \includegraphics[scale=0.25, trim = 0cm 0cm 0cm
                            0cm,
                            clip]{./Emissionsbilder/drei/nach_Emissionsmessung_Distanz.jpg}
                            %FIXME [width=640px, height=474px]
                            \caption{selbe Diode nach der Emissionsmessung}
                        \end{figure}
                   \vspace{-1.5em}

                    \end{minipage}

                \end{tabular}
                \end{center}

        \vspace{2em}

        Die Durchlassspannung der kleineren Diode betrug $1,5\ V$ bei $30\ mA$
        Strom. Die Messdauer wurde auf $4s$ gestellt. Vor und nach der Messung
        wurden folgende Bilder aufgezeichnet:


         \begin{center}
                \begin{tabular}{ll}

                \hspace{-10em}
                    \begin{minipage}{0.6\textwidth}

                        \begin{figure}[H]
                            \label{fig:wwwerwerew}
                            \includegraphics[scale=0.25, trim = 0cm 0cm 0cm
                            0cm,
                            clip]{./Emissionsbilder/vier/nach_Kontaktierung.jpg}
                            %FIXME [width=640px, height=474px]
                            \caption{kontaktierte Diode, Live-Bild der CCD
                            Kamera}
                        \end{figure}

                    \end{minipage}
                    \begin{minipage}{0.6\textwidth}

                         \begin{figure}[H]
                            \label{fig:iouiouoouio}
                            \includegraphics[scale=0.25, trim = 0cm 0cm 0cm
                            0cm,
                            clip]{./Emissionsbilder/vier/nach_Emission_Distanz.jpg}
                            %FIXME [width=640px, height=474px]
                            \caption{selbe Diode nach der Emissionsmessung}
                        \end{figure}
                   \vspace{-1.5em}

                    \end{minipage}

                \end{tabular}
                \end{center}

        \vspace{2em}

        Bislang ist nur eine Emissionsintensität zu sehen. Mithilfe der
        Phemos-Software konnte daraus ein Intensitätsprofil erstellt werden,
        woraus am Ende dann auf die Diffusionslänge zurück geschlossen werden
        kann.

        \vspace{1.5em}

        Es wurde auch eine Fingerstrukter-Diode für die Messung verwendet,
        welche sich ebenfalls auf dem äußeren Die befand. Eine
        Diffusionslängenberechnung wurde für diese Diode nicht durchgeführt, da
        am Phemos ein Objektiv gewählt wurde, mit der die gesamte Diode auf des
        Bildschirm sichtbar war. Daher sind die Fingerstrukturen zwar sichtbar,
        aber die pn-Übergänge zwischen n-Finger und dem p-Substrat dazwischen,
        die Grenze an der die Emission erwartet wird, sind viel zu klein
        abgebildet, als dass eine Intensitätskurve mit der geringen Vergrößerung
        sinnvoll wäre. Daher sind nur die Kontaktierung und das Ergebniss der
        Emissionsmessung im Protokoll aufgeführt. Die Durchlassspannung während
        der Messung betrug $5\ V$ bei einem Strom von $20\ mA$ und einer
        Messdauer von $4s$.


            \begin{center}
                \begin{tabular}{ll}

                \hspace{-10em}
                    \begin{minipage}{0.6\textwidth}

                        \begin{figure}[H]
                            \label{fig:nbmkhjuk}
                            \includegraphics[scale=0.25, trim = 0cm 0cm 0cm
                            0cm,
                            clip]{./Emissionsbilder/fuenf/nach_Kontaktierung.jpg}
                            %FIXME [width=640px, height=474px]
                            \caption{kontaktierte Diode, Live-Bild der CCD
                            Kamera}
                        \end{figure}

                    \end{minipage}
                    \begin{minipage}{0.6\textwidth}

                         \begin{figure}[H]
                            \label{fig:zttrzttz}
                            \includegraphics[scale=0.25, trim = 0cm 0cm 0cm
                            0cm,
                            clip]{./Emissionsbilder/fuenf/SuperImpose.jpg}
                            %FIXME [width=640px, height=474px]
                            \caption{selbe Diode nach der Emissionsmessung}
                        \end{figure}
                   \vspace{-1.5em}

                    \end{minipage}

                \end{tabular}
                \end{center}

        \vspace{2em}

        Man kann erkennen, dass die n-Finger, welche von links nach rechts
        Richtung p-Pad reichen, die Positionen der pn-Übergänge bilden. Daher
        wäre eigentlich eine Emission entlang der Ränder der n-Finger zu erwarten.
        Eine Erklärung, warum aber die größte Emission nur an der Stelle
        auftritt, wo der Abstand zwischen n- und p-Pad am geringsten ist, wäre, dass die
        Rekombination hinter der Raumladungszone dort viel größer ist, wo der
        Halbleiter den geringsten Widerstand aufweist. Der Strom könnte also
        dort am größten werden, wodurch eine stärkere Emission erklärt wäre.

        \end{quote}


        \subsection{Auswertung}
        \begin{quote}

        Die Phemos-Software ist in der Lage, aus der Messung eine Wertetabelle
        zu erstellen, welche die jeweilige Strahlungsintensität an dem
        betrachteten Punkt im Substrat ausgibt. Mit diesen Werten konnte die
        Strahlungsintensität, welche den Erwartungen nach einen exponentiellen
        Abstieg aufweisen sollte, nachgebildet werden. Um zur gesuchten
        Diffusionslänge zu gelangen wird diese Kurve nun logarithmisch
        aufgetragen. Denn wenn man sich nochmal die relevante Proportionalität
        ansieht:

        \begin{equation*}
        \begin{split}
            I(x) \sim \ \Delta n \sim \ exp(-\frac{x}{L_n})\\
            log(I(x)) \sim -\frac{x}{L_n}
        \end{split}
        \end{equation*}

        dann erkennt man, dass eine Logarithmierung des Exponentialtherms nur
        das Argument übrig lässt. Daher sollte aus jeder Intensitätskurve
        eine absteigende Gerade entstehen, welche eine Steigung von $-\frac{1}{L_n}$
        besitzt. Über den Reziprokwert dieser Steigung kann man dann die
        Diffusionslänge $L_n$ der Dioden berechnen.

        \vspace{1em}

        Beginnend mit der ersten Messung wird die Intensitätskurve mit dem
        exponentiellen und dem logarithmischen Verlauf geplottet und analysiert:

        \begin{figure}[H]
                    \centering
                        \includegraphics[scale=0.53, trim = 1cm 6cm 1.5cm 8cm,
                        clip]{./Emissionsbilder/eins/Intensitatsmessung.pdf}
                        \caption{Intensitätskurve der 1.Messung, exponentiell
                        und logarithmisch}
                            \label{fig:./Emissionsbilder/eins/Intensitatsmessung.pdf}
        \end{figure}


        Die blaue Kurve zeit den exponentiellen, die rote Kurve den
        logarithmischen Verlauf. Aus dieser wird nun der Kehrwert der Steigung
        ermittelt, welche eine Diffusionslänge von $304,2\ \mu m$ liefert.\\

        Die zweite Messung an der kleineren Diode im selben Die liefert
        folgende Intensitätsverläufe

        \begin{figure}[H]
                    \centering
                        \includegraphics[scale=0.53, trim = 1cm 6cm 1.5cm 8cm,
                        clip]{./Emissionsbilder/zwei/Intensitatsmessung.pdf}
                        \caption{Intensitätskurve der 2.Messung, exponentiell
                        und logarithmisch}
                            \label{fig:./Emissionsbilder/zwei/Intensitatsmessung.pdf}
        \end{figure}

        und, mit der analogen Berechnung, eine Diffusionslänge von $304,6\ \mu
        m$.\\

        Die dritte Messung an der großen Emissions-Diode am Rand des Wafers,
        liefert folgende Intensitätsverläufe

        \begin{figure}[H]
                    \centering
                        \includegraphics[scale=0.53, trim = 1cm 6cm 1.5cm 8cm,
                        clip]{./Emissionsbilder/drei/Intensitatsmessung.pdf}
                        \caption{Intensitätskurve der 3.Messung, exponentiell
                        und logarithmisch}
                            \label{fig:./Emissionsbilder/drei/Intensitatsmessung.pdf}
        \end{figure}

        und eine Diffusionslänge von $304,3\ \mu m$, wohingegen die kleinere
        Diode auf dem äußeren Die diese Kurven

        \begin{figure}[H]
                    \centering
                        \includegraphics[scale=0.53, trim = 1cm 6cm 1.5cm 8cm,
                        clip]{./Emissionsbilder/vier/Intensitatsmessung.pdf}
                        \caption{Intensitätskurve der 4.Messung, exponentiell
                        und logarithmisch}
                            \label{fig:./Emissionsbilder/vier/Intensitatsmessung.pdf}
        \end{figure}

        mit der Diffusionslänge von erneut $304,6\ \mu m$ wiedergibt.

        \vspace{1.5em}

        Obwohl diese Werte weit über dem Erwartungswert einer Diffusionslänge
        von $150 - 200\ \mu m$ liegen, spricht es für die Messung, dass alle
        Werte ziemlich gleich groß sind. Da alle gemessenen Dioden sich auf
        demselben Wafer befinden und somit alle gleich dotiert und behandelt wurden, ist
        es auch verständlich, dass die Diffusionslängen nicht bedeutend viel
        voneinander abweichen.\\
        Die vermutete Ursache für die dennoch zu großen Diffusionslängen könnte
        in dem Dotierungsschritt während der Wafer-Herstellung liegen. Eventuell
        wurde im Reinraum ein unbemerkter Fehler begangen, der zu größeren
        $L_n$s führte.

        \end{quote}

\end{quote} %sec Emissionsmessung

%--------------------------------------------------------------------
%--------------------------------------------------------------------

\newpage

\begin{thebibliography}{999}

% \bibitem{Boris}Boris Henckell: Ein Paar sachen geklaut.. ähhh inspirationen geholt
% \href{http://www.krachler.com/fileadmin/user_upload/arbeiten/Reglersynthese_Christian_Krachler.pdf}{Reglersynthese nach dem Frequenzkennlinienverfahren}, S16, S22, 08.05.2012


% Name, Vorname.; evtl. Name2, Vorname2.: Titel des Dokumentes
% oder Buches, Zeitschrift/Verlag/URL (Auflage, Erscheinungsort, -jahr), ggf. Seitenzahlen
%\bibitem [Wiki10] {DigitaleMesskette2} \url{www.wikipedia.org}, Zugriff 22.03.2010

\bibitem [1]{TBH1} Prof. Boit, Clemens Helfmeier, Philipp Scholz: Laborskript 
Technologie und Bauelemente der Halbleitertechnik (SS 2012)
\bibitem [2]{TBH2} Prof. Boit, Clemens Helfmeier, Philipp Scholz: Laborskript 
Praktikum Grundlagen und Bauelemente (WS 2011)
\bibitem [2]{TBH3} Silizium-Planartechnologie, Grundprozesse, Physik und 
Bauelemente; H.-G. Wagemann, T.Schönauer; B.G.Teubner Verlag/GWV Fachverlage 
GmbH, Wiesbaden 2003;
\bibitem [3]{TBH4} \url{www.isis.tu-berlin.de/course/view.php?id=6010}, Zugriff
13.09.2012
\bibitem [3]{TBH5} \url{www.wikipedia.org/wiki/Physikalische_Gasphasenabscheidun
g}, Zugriff 13.09.2012
\end{thebibliography}

%--------------------------------------------------------------------
%--------------------------------------------------------------------

\end{document}
