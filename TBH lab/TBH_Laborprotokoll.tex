\newcommand{\institut}{}
\newcommand{\fachgebiet}{Halbleiterbauelemente}
\newcommand{\veranstaltung}{Praktikum Technologie und Bauelemente der Halbleitertechnik}
\newcommand{\pdfautor}{Dirk Barbendererde (321 836), Thomas Kapa (), Alona
Siebert (), Özgü Dogan (326048)}
\newcommand{\autor}{Dirk Barbendererde (321 836)\\ Thomas Kapa ()\\ Alona
Siebert ()\\ Özgü Dogan (326 048)}
\newcommand{\pdftitle}{Praktikum\ Technologie und Bauelemente der
Halbleitertechnik}
\newcommand{\prototitle}{Praktikum Technologie und Bauelemente der Halbleitertechnik}
\newcommand{\aufgabe}{}

\newcommand{\gruppe}{Gruppe 1}
\newcommand{\betreuer}{Betreuer:\\ Clemens Helfmeier\\ Philipp Scholz}



\input{../../packages/tu_header_9}

\setcaptionwidth{7.5cm}

\begin{document}


%     \lstinputlisting{./praktikum6.sce}



%---------------------------------------------------------------------
%---------------------------------------------------------------------
%---------------------------------------------------------------------

\section{Einleitung}
\begin{quote}

	Die fortschreitende Miniaturisierung mikroelektronischer Bauteile hat uns 
	den Übergang zur planartechnologieschen Herstellung  von Transistoren und 
	Integrierten Schaltungen ermöglicht. Der Grundbaustein der Planartechnologie 
	ist ein pn-Übergang.\\

	Innerhalb dieses Praktikums wird eine pn-Diode (die einfachste Form eines 
	pn-Übergangs) im Reinraum hergestellt und es werden ihre Eigenschaften 
	untersucht und gemessen.\\

	Die Herstellung einer Diode erfolgt in einem mehrstufigen Prozess.
 	Als erstes wird ein Wafer (s. Abb. 1) aus dem Silizium hergestellt und als 
 	Ausgangsmaterial für weitere Prozessschritte benutzt . Danach werden dünne 
 	Schichten aus Materialien mit unterschiedlichen Eigenschaften schichtweise 
 	auf diesem Siliciumsubstrat (Wafer) aufgebaut und durch verschiedene 
 	Verfahren wie Lithographie und Ätzen bearbeitet.\\
 	

\end{quote} %sec Einleitung

%--------------------------------------------------------------------
%--------------------------------------------------------------------
\section{Herstellung eines pn-Übergangs}
\begin{quote}

	\subsection{Herstellungsschritte}
	\begin{quote}
	
		\subsubsection{Wafer vorbereiten}
        \begin{quote}
			Vorbereitung der Wafer und Nummerierung:\\
			
			Einige Schritte der Waferherstellung waren bereits schon vor 
			unserem Praktikum durch die betreuenden Labormitarbeiter 
			durchgeführt.\\

			Die an sich runden Wafer haben zwei Flats:  großes tieferes Flat 
			gibt die Kristallrichtung des monokristallinen Wafers an und das 
			kleinere Flat gibt den Dotierstoff an. Die für unser Praktikum 
			verwendeten Wafer haben p-Type Substrat und wurden  mit Bor 
			vordotiert.\\
			
			Danach wurde jeder Wafer auf der Rückseite nummeriert (s. \ref{fig:WafRueckseite}). 
			Die Nummern waren 120501, 120502 und 120503.\\
			
			\vspace{2em}
			
			\begin{figure}[H]
				\hspace{2.0cm}
                \includegraphics[scale=0.8, trim = 0cm 0cm 0cm 0cm,clip]
                	{./HerstellungBilder/WafRueckseite.png}
                  \caption{Schema für die Beschriftung}
                \label{fig:WafRueckseite}
            \end{figure}
            
            \vspace{2em}
            
            Reinigung:\\
            
            Die Wafer wurden nun vor dem Herstellungsprozess gereinigt. Diesen 
            Schritt brauch man um diverse kleine Partikeln aus der umgebenden 
            Luft, die sich auf der Oberfläche des Wafers ablagern, zu entfernen. 
            Außerdem können metallische Rückstände bei der Nummerierung der 
            Wafer auftreten.\\
			Es wurde ein RCA-Reinigungsprozess verwendet, der aus zwei Schritten 
			besteht: Standard-Clean 1 und Standard-Clean 2 (SC1 & SC2).\\

			SC1 wird zum Entfernen von organischen Resten verwendet: Wafer wird 
			10 Minuten lang in einer Lösung,  bestehend aus dem deionisierten 
			Wasser H2O, Ammoniak NH4+OH und Wasserstoffperoxid H2O2 im 
			Verhältnis 5 : 1 : 1, gespült.\\
			Außerdem wurde ein HF-Dip mit 1 %iger Lösung 25 Sekunden lang 
			%durchgeführt, um das natürliche Oxid von der Oberfläche zu 
			%entfernen.
			SC2 wird zum Entfernen von metallischen Resten benutzt. Dafür wird 
			der Wafer mit einer wässrigen Lösung aus Salzsäure und 
			Wasserstoffperoxid  im Verhältnis 6 : 1 : 1 behandelt.\\

			Dabei  soll man unbedingt die Warnhinweise beachten. Da bei der 
			Reinigung verwendete Lösungen sehr gefährlich (sehr giftig und stark 
			ätzend) sein können, muss man unbedingt bei der Arbeit mit diesen 
			Lösungen geeignete Schutzbekleidung und geeignete Schutzhandschuhe 
			tragen. Beim Kontakt mit diesen Lösungen soll man sofort gründlich
			mit Wasser abspülen und den Arzt konsultieren.\\
			
			Oxidieren:\\
			
			Nach der Reinigung waren die Wafer mit einer dünnen chemischen 
			Oxidschicht (SiO2)  bedeckt. Da diese Schicht zu dünn ist, wird ein 
			RTP-Prozess (Rapid Thermal Processing) benötigt.  Der Wafer wurde 
			auf ca. 1200 ° C erhitzt.  Die umgebende Luft reagiert mit den Si- 
			Atomen auf der Oberfläche und es bildet sich dabei eine ca. 60 nm 
			-dicke Siliziumoxid- Schicht. Danach wurde ein PECVD Prozess 
			(Plasma Enhansed Chemical Vapor Deposition) benutzt: Es scheidet 
			sich die Oxidschicht  auf der Oberfläche ab, die 250nm dick ist.\\

			Bei der PECVD erfolgt die Abscheidung von dünnen Schichten durch 
			eine chemische Aktivierung des Reaktionsprozesses, die durch ein 
			Plasma unterstützt wird.\\
			Für die Schichtbildung werden die Ausgangsstoffe als Gasgemisch in 
			einen Rezipienten eingelassen. Durch erhitztes Plasma werden die 
			Bindungen des Reaktionsgases aufgebrochen und in Radikale zersetzt, 
			die sich auf dem Substrat niederschlagen und dort die chemische 
			Abscheidereaktion bewirken.\\
 			Der PECVD-Prozess lief unter 400°C ab.\\
 			Die Oxiddickenmessung danach ergab ca. 400nm.\\

			Nach dem PECVD- Prozess wurde noch mal RTP-Tempern benötigt, um das 
			Oxid in das Gitter einzufügen und damit das Oxid an der Oberfläche 
			des Substrats gut haften kann. Bei RTP wurde der Wafer 120 Sekunden  
			lang auf 1000°C erhitzt.\\
			Die Oxiddickenmessung mit dem Photometer danach ergab ca. 420nm.\\
			
			Lithographie:\\
			
			Bei dem lithographischen Prozess wurden die Strukturen für die 
			Diffusion der n-Wannen vorbereitet.\\
			Als erstes werden die Wafer von den möglichen Wasser- und 
			Flüssigkeitsresten befreit. Dafür wird der Wafer 30 min. lang bei 
			200°C auf einer Heizplatte erhitzt.\\
			
			HMDS:\\

			Außerdem hilft es ein Haftmittel HMDS (Hexamethyldisilazan) auf den 
			Wafer aufzutragen:\\
			die Wafer werden in eine Vakuumglocke gelegt, es muss 30 Sekunden 
			lang abgepumpt werden, um das Vakuum zu erzeugen.\\
			HMDS  lagert sich an der Oberfläche des Wafers ab (die Dauer beträgt 
			5 min). \\
			Anschließend werden die Wafer  eine Minute lang auf der Heizplatte 
			getrocknet.\\
			
			Belacken & Softbake:\\
			
			Die  Wafer werden auf einen Schleuderapparat gelegt und mit dem AZ 
			5214  Lack beschichtet. Mit dem Schleuderprogramm werden die Wafer 
			auf 4000 U/min beschleunigt.\\
			Nach der 20-minutigen Pause werden die Wafer bei 90°C zwei min. 
			lang getrocknet, damit der Lack fest wird.\\
			
			Belichten:\\
			
			Nun werden die Wafer mit der Active Area Mask  (AA-Maske) abgedeckt 
			und vier Sek. belichtet.\\ 
			
			Entwickeln & Hardbake:\\
			
			Nach der Belichtung wurde  der Lack in einer Rohm Haas - Lösung 70 
			Sekunden lang entwickelt.\\
			Danach wurden alle Wafer auf einer Heizplatte unter 120°C 5 min. 
			lang erhitzt, damit der restlicher Lack gegen Ätzmittel noch 
			resistenter wird.\\
			Die Inspektion mit dem Fotometer ergab die Dicke des Photolackes 
			ca.1.9 µm.\\
            
            Nass-chemisches Ätzen:\\

			Mit dem Nass-chemischen Ätzen  werden die Fenster zum Substrat 
			weggeätzt.\\
			Die Wafer wurden in der BHF-Lösung (mit Ammoniumfluorid gepufferte 
			Flusssäure) ca. 6.5 min. gehalten.\\
			
			Fotolack entfernen:\\

			Der Lack wurde mit  Caro's Etch - Lösung entfernt, indem die Wafer 
			für 10 min. lang in diese Lösung eingetaucht wurden.\\

			Die Abbildung \ref{fig:AusStruk} zeigt die Struktur, wie wir unsere 
			Wafer als Ausgangsmaterial für unseres Praktikum erhalten haben:
            
            \vspace{2em}
            
            \begin{figure}[H]
				\hspace{5.0cm}
                \includegraphics[scale=0.8, trim = 0cm 0cm 0cm 0cm,clip]
                	{./HerstellungBilder/AusgangsStruktur.png}
                  \caption{Ausgangsstruktur}
                \label{fig:AusStruk}
            \end{figure}
            
            \vspace{2em}
            
			\end{quote}
	
	\subsection{Tagesgliederung}
	
		\subsubsection{1. Tag}
	
		Verhaltensregeln in dem Reinraum:\\

		Als erstes, vor dem Eintritt in Reinraum, erhielten alle Teilnehmer eine 
		Sicherheitseinweisung vom Herrn Bruhns, dem Laborleiter. Die Vorschrift 
		ist: Schutzoveralls, spezielle Schuhe, die eine Erdung zum Boden 
		enthalten, und Handschuhe zu tragen. Bei der Arbeit mit Chemikalien 
		müssen die Augen zusätzlich geschützt werden. Dazu kann man entweder 
		die Gesichtsschutzmaske oder die Brille tragen.\\
		Essen und Trinken im Reinraum sind streng verboten.\\
 		Außerdem sollte das Verlassen und Wiederbetreten möglichst vermieden 
 		werden, da bei jedem Betreten die Partikeln eingeschleppt werden können,
 		die die Reinheit des Labors vermindern.\\

		Weil mit den toxischen Substanzen gearbeitet wird, dürfen die 
		Schwangeren das Praktikum nicht durchführen.
	
		\vspace{2em}
            
            \begin{figure}[H]
				\hspace{4.8 cm}
                \includegraphics[scale=0.5, trim = 0cm 0cm 0cm 0cm,clip]
                	{./HerstellungBilder/Teilnehmer.png}
                  \caption{Teilnehmer}
                \label{fig:teiln}
            \end{figure}
            
    	\vspace{2em}
    
    	1. Dotierlösung aufbringen:\\
    
		Als erstes haben wir unsere drei Wafer unter dem Lichtmikroskop 
		untersucht, um die Struktur der Wafer zu sehen. Da die Oxide das Licht 
		reflektieren, daher ist auch die grüne Farbe im Mikroskop zu sehen 
		(s. Abb. \ref{fig:mikro1}) Es wurden keine großen Schäden gefunden.
    
    	\vspace{2em}
            
    		\begin{figure}[H]
				\hspace{4.7 cm}
                \includegraphics[scale=0.5, trim = 0cm 0cm 0cm 0cm,clip]
                	{./HerstellungBilder/Mikroskopbild1.png}
                  \caption{Mikroskopaufnahme}
                \label{fig:mikro1}
            \end{figure}
            
    	\vspace{2em}
            
    	Nun wird der Dotierstoff Phosphorus p 509 aufgetragen.        
    
    	\vspace{2em}
    
    	\begin{center}
                \begin{tabular}{ll}

                \hspace{-14em}
                    \begin{minipage}{0.6\textwidth}
                        \begin{figure}[H]
                        \hspace{8em}
                            \includegraphics[scale=0.7, trim = 0cm 0cm 0cm
                            0cm, clip]{./HerstellungBilder/Phosphorus.png}
                            \caption{Phosphor}
                           \label{fig:phos}
                        \end{figure}

                    \end{minipage}
                    \begin{minipage}{0.6\textwidth}

                        \begin{figure}[H]
                        \hspace{3.5em}
                            \includegraphics[scale=0.7, trim = 0cm 0cm 0cm
                            0cm, clip]{./HerstellungBilder/Dotierstoffauftragen.png}
                            \caption{Auftragen des Dotierstoffes}
                           \label{fig:aufDot}
                        \end{figure}
                    \vspace{-1.5em}

                    \end{minipage}

                \end{tabular}
		\end{center}
    
    	\vspace{2em}
    
		Um gleichmäßige Verteilung des Dotierstoffes zu bekommen, wurde die 
		Auftragung der Phosphorlösung mit einem 
		Lackschleuderbeschichtungsapparat (s. Abb.) durchgeführt.\\
 		Auf einem Drehtisch wird der Wafer zentral justiert und mittels Vakuum
 		fixiert, damit der Wafer bei der Drehung am Teller gut haftet. Mit Hilfe 
 		von Dispenser (die zwei	Mal gedrückt wurde) wird der Dotierstoff auf den
 		Wafer getropft. Danach muss der Wafer sofort in Rotation gebracht werden
 		, um eine ebenmäßige Dotierstoffschicht zu bekommen. Bei der Rotation 
 		wird die überflüssige Stoffmenge von der Scheibe weggeschleudert.\\
		Es bleibt nur eine sehr dünne Phosphorschicht auf dem Wafer.\\

		Bedingungen beim Aufschleudern der Phosphorlösung:\\
		Programm 4\\
		2500 U/min (Anzeige 250)\\
		2 ml Lösung\\

		Bemerkung: der Wafer 120503 wurde doppelt mit dem Dotierstoff 
		beschichtet (ca. 4 ml).\\
		Anschließend wurden alle drei Wafer auf der Heizplatte  15 Minuten lang 
		bei 200°C gebacken. Bei diesem Prozessschritt bildet sich ein 
		Phosphorglas (Silikat). Dabei wurde eine Dampfwolke beobachtet. \\
		Das Ergebnis nach diesem Vorgang ist in der Abbildung \ref{fig:Waf_phos}  
		zu sehen.\\   
    
        \vspace{2em}
            
    		\begin{figure}[H]
				\hspace{4.7 cm}
                \includegraphics[scale=0.5, trim = 0cm 0cm 0cm 0cm,clip]
                	{./HerstellungBilder/StrukturmitPhosphorus.png}
                  \caption{Wafer bedeckt mit Phosphorglas}
                \label{fig:Waf_phos}
            \end{figure}
            
        \vspace{2em}
            
		Eine Sichtkontrolle mit dem Lichtmikroskop hat uns folgendes ausgegeben:
         
     	\vspace{2em}
            
    		\begin{figure}[H]
				\hspace{3.5 cm}
                \includegraphics[scale=0.5, trim = 0cm 0cm 0cm 0cm,clip]
                	{./HerstellungBilder/Mikroskopbild2.png}
                  \caption{Bläschen Wafer 01}
                \label{fig:blaes}
            \end{figure}
      
     	\vspace{2em}
     
    	Zu beobachten war:\\
		Bei dem Wafer 02 waren Bläschen stärker, der Wafer 03 hatte fast gar 
		keine Bläschen, die Struktur war ebenmäßiger und er enthielt so viel 
		Dotierstoff, dass die Fenster kaum zu sehen waren. Der Wafer 01 hatte 
		große Bläschen, die ständig Ihre Form verändert haben.
            
		\end{quote}
	
		2 Diffusionsofen:\\

		Jetzt kommen die Wafer in den Diffusionsofen.\\
	
		Als erstes wurden die Wafer  so in dem Quarzschiffchen einsortiert, dass 
		sich die Vorderseiten der Wafer 01 und 02 gegenüber stehen. Wafer 03 
		wurde einsam auf der anderen Seite des Quarzschiffchens platziert.
	
		\vspace{2em}
    
    	\begin{center}
                \begin{tabular}{ll}

                \hspace{-7em}
                    \begin{minipage}{0.5\textwidth}
                        \begin{figure}[H]
                        \hspace{-2em}
                            \includegraphics[scale=0.8, trim = 0cm 0cm 0cm
                            0cm, clip]{./HerstellungBilder/Quarzschiffchen.png}
                            \caption{Quarzschiffchen}
                           \label{fig:quarz}
                        \end{figure}

                    \end{minipage}
                    \begin{minipage}{0.75\textwidth}

                        \begin{figure}[H]
                        \hspace{8em}
                            \includegraphics[scale=0.7, trim = 0cm 0cm 0cm
                            0cm, clip]
                            {./HerstellungBilder/einbringeninDiffusionsofen.png}
                            \caption{Einbringen in den Diffusionsofen}
                           \label{fig:ein_diff}
                        \end{figure}
                    \vspace{-1.5em}

                    \end{minipage}

                \end{tabular}
		\end{center}
	
		\vspace{2em}
	
		Dann wird das Schlitten 65 cm tief im Ofen platziert, damit die Wafer 
		sich möglichst in der Ofenmitte befinden. Zum Abschluss wurde 
		anschließend noch ein Quarzhohlzylinder in den Ofen geschoben, um die 
		Temperaturhaltung zu verbessern.\\
		Der Prozess der Diffusion findet unter dem Durchströmen des 
		Diffusionsofens mit dem Prozessgas für zehn min. bei der Temperatur von 
		1000 °C. 
	
		\vspace{2em}
            
    		\begin{figure}[H]
				\hspace{4 cm}
                \includegraphics[scale=0.75, trim = 0cm 0cm 0cm 0cm,clip]
                	{./HerstellungBilder/diffusionsofen.png}
                  \caption{Diffusionsofen}
                \label{fig:diff_ofen}
            \end{figure}
            
    	\vspace{2em}
	
		Nach dem Diffusionsprozess verblieben die Wafer noch zum langsamen 
		Auskühlen über die Nacht im Ofen.\\
	
		Nach der Diffusion sieht unsere Struktur so aus wie in Abb. 
		\ref{fig:nach_diff_ofen}:
	
		\vspace{2em}
            
    		\begin{figure}[H]
				\hspace{4 cm}
                \includegraphics[scale=0.7, trim = 0cm 0cm 0cm 0cm,clip]
                	{./HerstellungBilder/StrukturnachDiffusionsofen.png}
                  \caption{Struktur nach Diffusionsofen}
                \label{fig:nach_diff_ofen}
            \end{figure}
            
    		\vspace{2em}
	
		\subsubsection{2. Tag}
		
		Am zweiten Tag haben wir unsere Wafer aus dem Ofen genommen. Dabei haben 
		wir beobachtet, dass der Wafer 03 glänzend und der Wafer 01 und 02 matt 
		waren. Danach wurden alle Wafer einer Sichtkontrolle unter dem Mikroskop 
		unterzogen und  mit dem Photometer wurde die Schichtdicke gemessen.\\

			Photometer: Das Photometer Ergolux besteht aus einem Mikroskop mit 
			einem  Aufsatz, der Licht bestimmter Wellenlängen (zwischen 400 nm 
			und 800 nm ) auf den Wafer strahlt und über die Reflexion der 
			Lichtquanten der unterschiedlichen Wellenlängen den Aufschluss über 
			die Schichtdicke gibt.\\

			Zur Kalibrierung muss das Photometer mit einer Musterprobe justiert 
			werden(mit dem Flat nach unten):  die Musterprobe (ein nicht 
			beschichteter Siliziumwafer) soll auf den Objekthalter so platziert 
			werden, damit man die Bildschärfe einstellen kann. Dann startet man 
			das Programm  Justage. Danach wird es ohne Musterprobe auf dem 
			Objekthalter noch mal gestartet.\\
			Nur dann  können wir unsere Wafer messen.\\

			Die Messergebnisse für unsere drei Wafer sind in der Tabelle unten 
			zusammengefasst.
			
			\vspace{2em}

      		\begin{table}[h]
     		  \begin{addmargin}[1cm]{3cm}
     			\centering
                    \begin{tabular}{|p{2cm}|p{2cm}|p{2cm}|p{2cm}|p{2cm}|p{2cm}|}
         			\hline
         			Wafer & oben & mittig & unten & links & rechts\\
         			\hline 
        			120501 & 82.3  & 78.1  & 86.8  & 85.4  & 73.2 \\
                    120503 & 770.1 & 784.8 & 765.3 & 764.8 & 776.2 \\
                    \hline
        
                    \end{tabular}
              \end{addmargin}
              \caption{gemessene Phosphorglasdicke in nm}
              \label{Phosphordicke}
            \end{table}

            \vspace{2em}
            
            Die Wafer 01 und 02 waren nicht gut messbar! Der Wafer 03  hat 
            dagegen fast perfekte Messergebnisse. 
			
\end{quote}
%--------------------------------------------------------------------
%--------------------------------------------------------------------

% \section{Kennlinie}
% \begin{quote}
%     
%     Benennung der Dateien:\\
%     Kennlinie_{Wavernummer}_Die_[{Zeile},{Spalte}]_Mess_{messung}.mat\\
%     
%     100 µA Strombegrenzung\\
%     Welche Diode?
%     Welcher Manipulator
%     
%     SMU1 = GNDi
%     SMU2 = GND
%     SMU3 = Var1
%     
%     Unterdiffusion
%     
%     unterspannung
%     0.0 & 2.4
%     0.5 & 3.8
%     0.8 & 4.5
%     1.0 & 5.0
%     1.3 & 5.5
%     1.8 & 6.7
%     2.5 & 8.2
%     3.0 & 9.2
%     
%     \TODO{müssen wir die Dies durchnummerieren?}
% \end{quote} %sec Kennlinie
% 
% %--------------------------------------------------------------------
% %--------------------------------------------------------------------
% 
% \section{Schaltverhalten}
% \begin{quote}
%     
%   
% 
% 
% 	In diesem Versuch soll Schaltverhalten bei Strom- und Spannungssprüngen 
% 	untersucht werden. Ein Maß für die Geschwindigkeit mit der die Diode 
% 	schalten kann ist die 
% 	Minoritätsträgerlebensdauer. Der Schaltvorgang hält 
% 	an, bis diese auf- bzw. abgebaut sind. Daher ist es Ziel dieses Versuches 
% 	über zwei unterschiedliche Verfahren diese Ladungsträgerlebensdauer zu 
% 	bestimmen. Dies ist zum einen der Stromausschaltvorgang und zum anderen die
%     Stromkommutierung.\\
% 
% 	Zunächst sollen das ideale und das reale Schaltverhalten gegenüber gestellt 
% 	werden. Das ideale Verhalten charakterisiert sich durch verzögerungs- und 
% 	verlustfreies Schalten. Die zu messenden Dioden zeigen allerdings durch 
% 	Energiespeicher wie die Sperrschicht- und die Diffusionskapazität kein 
% 	ideales Schaltverhalten. \\
% 
% 	Dabei ist die Charakteristik des Schaltverhaltens davon abhängig, ob es 
% 	sich um ein Spannungs- oder Stromsprung und einen Ein- oder Ausschaltvorgang 
% 	handelt. Um das Verständnis für diese Vorgänge zu verbessern, soll im 
% 	Folgenden näher auf einige Beispiele eingegangen werden.\\
% 
% 	Bei einem Einschaltstromsprung werden zwei Fälle unterschieden: Die starke 
% 	und die schwache Injektion. Ob starke oder schwache Injektion vorliegt 
% 	richtet sich nach der Ladungsträgeranzahl, die von der einen Seite des 
% 	pn-Überganges als Majoritätsträger auf die andere Seite als Minoritätsträger 
% 	gelangen. Ist die Anzahl der auf der anderen Seite ankommenden nun 
% 	Minoritäten in etwa so groß, oder größer wie die Majoritäten spricht man von
% 	starker Injekion. Andernfalls spricht man von schwacher Injektion.\\
%     Wie in Bild \ref{fig:Stromeinschalten} zu erkennen, spielt in den beiden 
%     Fällen der Bahnwiderstand eine unterschiedliche Rolle. Nach Shockley ist der 
%     Bahnspannungsabfall für schwache Injektion zu vernachlässigen. Mit Hilfe der 
%     Boltzmanfaktoren lässt sich ein logarthmischer Verlauf der Spannung über der 
%     RLZ herleiten. Dieser ist in der Abbildung \ref{fig:Stromeinschalten} in der 
%     grob gestrichelten Kennlinie zu erkennen. Man könnte vermuten, dass ohne den 
%     Bahnwiderstand nur der kapazitive Anteil zur Wirkung kommt und die Spannung
% 	daher kaum springen darf.\\
% 	Kommt hingegen der Einfluss des Bahnwiderstandes bei der starken Injektion 
% 	hinzu, dann kann ein heftiger
% 	Spannungssprung erfolgen. Der zusammengefasste Spannungsverlauf ist in Bild 
% 	\ref{fig:Stromeinschalten} an der durchgezogenen Kennlinie zu erkennen. 
% 	Durch den Bahnwiderstand bekommt die Diode vergleichbar mit Zuleitungen 
% 	einen induktiven Charakter und die Spannung kann springen.
% 
%     \begin{figure}[h]
%         \centering
%         \includegraphics[scale=1]{Bilder/Stromeinschalten}
%         \caption{Spannungsverlauf beim Stromeinschaltsprung für starke und schwache Injektion \footnotemark}
%         \label{fig:Stromeinschalten}
%     \end{figure}
%          \footnotetext{Prof. Boit, Clemens Helfmeier, Philipp Scholz: Laborskript Technologie und Bauelemente der
% 	Halbleitertechnik (SS 2012), S. 78}
%    
%    \newpage
% 
%       \begin{table}[h]
%                \begin{addmargin}[-1cm]{3cm}
%                \centering
%                     \begin{tabular}{|p{5cm}|p{11.2cm}|}
%          \hline
%          Messung & Werte\\
%          \hline
%          Ausschaltvorgang & $I_{F}=10, 50, 100, 150 mA$\\
%    
%          \hline
%          Stromkommutierung & $I_{R0}=15,30,60,90,120 mA$\\
%          
%          \hline
%          
% 
%                     \end{tabular}
%                 \end{addmargin}
%              \caption{Messarten}
%           \label{Messarten}
%       \end{table}
%      
%       \vspace{2em}
% 
%       \begin{table}[h]
%      \begin{addmargin}[-1cm]{3cm}
%      \centering
%                      \begin{tabular}{|p{3cm}|p{3cm}|p{10.2cm}|}
%          \hline
%          Messung & Die/Diode/Wafer & gemessen\\
%          \hline
%          $R_{B} für I_{F}=10 mA$ & & \\
%                                  & & \\
%                                  & & \\
%                                  & & \\
%                                  & & \\
%                                  & & \\
%                                  & & \\
%                                  & & \\
%                                  & & \\
%                                  & & \\
%                                  & & \\
%                                  & & \\
%                                  & & \\
%          \hline
%          $R_{B} für I_{F}=50 mA$ & & \\
%                                  & & \\
%                                  & & \\
%                                  & & \\
%                                  & & \\
%                                  & & \\
%                                  & & \\
%                                  & & \\
%                                  & & \\
%                                  & & \\
%                                  & & \\
%                                  & & \\
%                                  & & \\
%          \hline
%          $R_{B} für I_{F}=100 mA$ & & \\
%                                  & & \\
%                                  & & \\
%                                  & & \\
%                                  & & \\
%                                  & & \\
%                                  & & \\
%                                  & & \\
%                                  & & \\
%                                  & & \\
%                                  & & \\
%                                  & & \\
%                                  & & \\
%          \hline
%          $R_{B} für I_{F}=150 mA$ & &\\
%                                  & & \\
%                                  & & \\
%                                  & & \\
%                                  & & \\
%                                  & & \\
%                                  & & \\
%                                  & & \\
%                                  & & \\
%                                  & & \\
%                                  & & \\
%                                  & & \\
%                                  & & \\
%    
%          \hline
% 
%                      \end{tabular}
%                  \end{addmargin}
%              \caption{Messwerte}
%            \label{Messwerte1}
%         \end{table}
%       
%        \vspace{2em}
%        
%        \begin{table}[h]
%      \begin{addmargin}[-1cm]{3cm}
%      \centering
%                       \begin{tabular}{|p{3cm}|p{3cm}|p{10.2cm}|}
%          \hline
%          Messung & Die/Diode/Wafer & gemessen\\
%          \hline
%          $\tau für I_{F}=150 mA$ & & \\
%                                  & & \\
%                                  & & \\
%                                  & & \\
%                                  & & \\
%                                  & & \\
%                                  & & \\
%                                  & & \\
%                                  & & \\
%                                  & & \\
%                                  & & \\
%                                  & & \\
%                                  & & \\
%          \hline
%          $t_{s}$ & & \\
%                                  & & \\
%                                  & & \\
%                                  & & \\
%                                  & & \\
%                                  & & \\
%                                  & & \\
%                                  & & \\
%                                  & & \\
%                                  & & \\
%                                  & & \\
%                                  & & \\
%                                  & & \\
%          \hline
%          $Q_{s}$ & &\\
%                                  & & \\
%                                  & & \\
%                                  & & \\
%                                  & & \\
%                                  & & \\
%                                  & & \\
%                                  & & \\
%                                  & & \\
%                                  & & \\
%                                  & & \\
%                                  & & \\
%                                  & & \\
%          \hline
%          $\tau$ & &\\
%                                  & & \\
%                                  & & \\
%                                  & & \\
%                                  & & \\
%                                  & & \\
%                                  & & \\
%                                  & & \\
%                                  & & \\
%                                  & & \\
%                                  & & \\
%                                  & & \\
%                                  & & \\
%          \hline
% 
%                            \end{tabular}
%                     \end{addmargin}
%              \caption{Messwerte}
%          \label{Messwerte2}
%       \end{table}
%       
%       \vspace{2em}
%       
%       \begin{table}[h]
%                  \begin{addmargin}[-1cm]{3cm}
%      \centering
%                      \begin{tabular}{|p{3cm}|p{3cm}|p{10.2cm}|}
%          \hline
%          Messung & Die/Diode/Wafer & gemessen\\
%          \hline
%          Tempertatur & & \\
%                                  & & \\
%                                  & & \\
%                                  & & \\
%                                  & & \\
%                                  & & \\
%                                  & & \\
%                                  & & \\
%                                  & & \\
%                                  & & \\
%                                  & & \\
%                                  & & \\
%                                  & & \\
%          \hline
%          Vorwärtsstrom & & \\
%                                  & & \\
%                                  & & \\
%                                  & & \\
%                                  & & \\
%                                  & & \\
%                                  & & \\
%                                  & & \\
%                                  & & \\
%                                  & & \\
%                                  & & \\
%                                  & & \\
%                                  & & \\
%          \hline
%          Rückwärtsstrom & &\\
%                                  & & \\
%                                  & & \\
%                                  & & \\
%                                  & & \\
%                                  & & \\
%                                  & & \\
%                                  & & \\
%                                  & & \\
%                                  & & \\
%                                  & & \\
%                                  & & \\
%                                  & & \\
%          \hline
%                      \end{tabular}
%                   \end{addmargin}
%              \caption{Sonstige Angaben zur Messung}
%          \label{AngabenZurMessung}
%       \end{table}
%       
%       \noindent
%       
%        \vspace{2em}
%       
% \end{quote} %sec Schaltverhalten
% 
% 
% 
% \begin{thebibliography}{999}
% 
% % \bibitem{Boris}Boris Henckell: Ein Paar sachen geklaut.. ähhh inspirationen geholt
% % \href{http://www.krachler.com/fileadmin/user_upload/arbeiten/Reglersynthese_Christian_Krachler.pdf}{Reglersynthese nach dem Frequenzkennlinienverfahren}, S16, S22, 08.05.2012
% 
% 
% % Name, Vorname.; evtl. Name2, Vorname2.: Titel des Dokumentes
% % oder Buches, Zeitschrift/Verlag/URL (Auflage, Erscheinungsort, -jahr), ggf. Seitenzahlen
% %\bibitem [Wiki10] {DigitaleMesskette2} \url{www.wikipedia.org}, Zugriff 22.03.2010
% 
% \bibitem [1]{TBH} Prof. Boit, Clemens Helfmeier, Philipp Scholz: Laborskript Technologie und Bauelemente der
% Halbleitertechnik (SS 2012)
% \end{thebibliography}
%     
% \end{quote} %sec Schaltverhalten
% 
% %--------------------------------------------------------------------
% %--------------------------------------------------------------------
% 
% \section{Emissionsmessung}
% \begin{quote}
%     
%     Eine Emissionsmessung ist in der Halbleitertechnologie insofern interessant,
%     weil sie Erkenntnisse über charakteristische Eigenschaften des Halbleiters,
%     in unserem Fall die selbst hergestellte Diode, liefert. Dazu gehören die
%     Ladungsträgerlebensdauer $\tau_{n,p}$ und die Diffusionslänge $L_{n,p}$. Auf
%     die Lebensdauer lässt sich mithilfe der Diffusionslänge und des
%     Diffusionskoeffizienten schließen. Dieser ist ein materialabhängiger Wert,
%     welcher von uns nicht weiter beachtet wird. Der folgende Zusammenhang hilft
%     bei der Berechnung von $\tau_{n,p}$:
%     
%     \begin{equation*}
%         \begin{split}
%             L_{n,p} = \sqrt{\tau_{n,p} \cdot D_{n,p}} 
%         \end{split}
%     \end{equation*}
%     
%     Ziel unserer Messung aber war die Bestimmung der Diffusionslänge in unserer
%     Diode. Dieser wurde über die realisierte Intensitätsmessung anhand der
%     Emissionen in der Diode ermittelt. Dabei wurden folgende
%     Proportionalitätsverhätnisse verwendet:
%     
%     \begin{equation*}
%         \begin{split}
%             I(x) \sim \ \Delta n \sim \ exp(-\frac{x}{L_n}) 
%         \end{split}
%     \end{equation*}
%     
%     Hierbei werden Strahlungsintensität ins Verhältnis mit der\\
%     Minoritätsüberschussladungträgerkonzentration und dieser wiederum 
%     ins Verhältnis mit einem Exponentialtherm gesetzt. Daher kann man die
%     Intensitätsmessung direkt mit diesem Therm in Verbindung setzen, 
%     welcher in seinem Argument die gesuchte Diffusionslänge beinhaltet. Weiteres
%     zur Berechnung des $L_n$ steht in der Auswertung der Messung.\\
%     
%     Um aber die Intensitätsmessung verstehen zu können müssen einige
%     grundlegende Theorien der Halbleiter bezüglich ihrer Typen und ihrer
%     Rekombinationsarten behandelt und nachvollzogen werden.\\ 
%     Daher gibt es vor der Versuchsdurchführung und der Auswertung zunächst eine
%     kleine Exkursion in den theoretischen Bereich.
%     
%         \subsection{Direkter und indirekter Halbleiter }
%         \begin{quote}
%         
%         Es gibt zwei Arten von Halbleitern, die direkten und die indirekten
%         Halbleiter. Diese unterscheiden sich darin, dass die Rekombination eines
%         Ladungsträgers aus dem Leitungs- in das Valenzband unterschiedliche
%         Vorraussetzungen erfordert.
%             
%             \subsubsection{direkter Halbleiter}
%             \begin{quote}
%             Die Rekombination bei einem direkten Halbleiter ist relativ simpel.
%             Ein freies Elektron braucht dabei nur, unter Abgabe der jeweiligen
%             Energie, den Bandabstand zwischen Leitungs- und Valenzband zu
%             überqueren.
%             
%             \begin{figure}[H]
%                     \centering
%                         \includegraphics[scale=0.72, trim = 1cm 0cm 1.5cm 0cm,
%                         clip]{./Emissionsbilder/restliches/direkt.png}
%                         \caption{Rekombination bei einem direkten Halbleiter}
%                             \label{fig:./Emissionsbilder/restliches/direkt.png}
%             \end{figure}
%             
%             
%             \end{quote}       
%         
%             \subsubsection{indirekter Halbleiter}
%             \begin{quote}
%             Die Rekombination bei einem indirekten Halbleiter erfordert neben
%             einem Energieunterschied auch einen Impulsunterschied, damit das
%             Elektron auf dem Valenzband auftreffen kann. Dieser
%             Impulsunterschied ist in der folgenden Abbildung zu erkennen.
%             
%             \begin{figure}[H]
%                     \centering
%                         \includegraphics[scale=0.73, trim = 1cm 0cm 1.5cm 0cm,
%                         clip]{./Emissionsbilder/restliches/indirekt.png}
%                         \caption{Rekombination bei einem indirekten Halbleiter}
%                             \label{fig:./Emissionsbilder/restliches/indirekt.png}
%             \end{figure}
%             
%             \TODO{Bildquellen, TBH-Skript S.92 einfügen}
%             \end{quote}       
%             
%             Ein weiterer Unterschied zwischen direkten und indirekten
%             Halbleitern ist der, dass direkte Halbleiter hauptsächlich
%             strahlend rekombinieren, wohingegen indirekte Halbleiter
%             stochastisch betrachtet fast nur nichtstrahlend rekombinieren.
%             Dennoch findet mit sehr geringer Wahrscheinlichkeit auch vereinzelt
%             strahlende Rekombination in den indirekten Halbleitern statt.\\
%             Bevor weiter darauf eingegangen wird, in wie fern dies für die
%             Emissionsmessung von Bedeutung ist, werden die Rekombinationsarten,
%             strahlend und nichtstrahlend, wiederholt.
%             
%         \end{quote}
%         
%         \subsection{Rekombinationsmechanismen}
%         \begin{quote}
%         
%         Eine Rekombination erfolgt stets unter Abgabe von einer
%         Energiedifferenz. Dabei wird unterschieden, ob diese Energie in Form
%         eines Lichtquants oder von Wärme abgegeben wird, wodurch auch die
%         Beschreibung strahlend oder nichtstrahlend entsteht.\\
%         Nun folgen je ein Beispiel für diese Machanismen.
%         
%             \subsubsection{strahlende Rekombination}
%             \begin{quote}
%             
%             Die strahlende Rekombination, hauptsächlich bei direkten Halbleitern
%             zu sehen, besagt, dass der rekombinierende Ladungsträger die
%             Energiedifferenz, welche er zurücklegt, in Form eines Photons
%             freigibt. Die Energie dieses Photons beträgt genau die Energie des
%             Bandabstands zwischen den beiden Bändern. Anders ausgedrückt kann
%             man sie auch Rekombinationsenergie nennen:
%                     
%             \begin{equation*}
%                 \begin{split}
%                     W_{Rek} = h \cdot \nu 
%                 \end{split}
%             \end{equation*}
%             
%             Diese Rekombinationsenergie setzt sich aus des Produkt aus dem
%             Planckschen Wirkungsquantums $h$ und die Frequenz des entstehenden
%             Lichts $\nu$.\\
%             
%             Ein Beispiel für die strahlende Rekombination ist die
%             Band-Band-Rekombination, welche in der folgenden Abbildung
%             dargestellt wurde:
%             
%             \begin{figure}[H]
%                     \centering
%                         \includegraphics[scale=0.68, trim = 1cm 0cm 1.5cm 0cm,
%                         clip]{./Emissionsbilder/restliches/bandband.png}
%                         \caption{strahlende Band-Band-Rekombination}
%                             \label{fig:./Emissionsbilder/restliches/direkt.png}
%             \end{figure}
%             
%             Man erkennt ein Elektron, welches beim Rekombinieren, die bereits
%             erwähnte Rekombinationsernergie in Form eines Photons abgibt. Es
%             kann aber auch vorkommen, dass die abgegebene Energie an ein
%             weiteres Eektron im Leitungsband abgegeben wird, wodurch dieser auf
%             ein höheres Energieniveau im Leitungsband angehoben und wieder runterfallen 
%             kann. Diese Möglichkeit ist bei der strahlenden Rekombination die
%             unwarscheinlichere Variante.
%             
%             \end{quote}
%             
%             \subsubsection{nichtstrahlende Rekombination}
%             \begin{quote}
%             
%             Die nichtstrahlende Rekombination findet hauptsächlich bei
%             indirekten Halbleitern statt. Dabei wird die Rekombinationsenergie
%             an ein weiteres Elektron im Leitungsband abgegeben, welches auf ein
%             höheres Energieniveau angehoben wird und unter Abgabe von
%             thermischer Energie wieder runterfallen kann.\\
%             Als ein Beispiel für die nichtstrahlende Rekombination wird die
%             Augerrekombination dargestellt:
%             
%             \begin{figure}[H]
%                     \centering
%                         \includegraphics[scale=0.78, trim = 1cm 0cm 1.5cm 0cm,
%                         clip]{./Emissionsbilder/restliches/auger.png}
%                         \caption{nichtstrahlende Auger-Rekombination}
%                             \label{fig:./Emissionsbilder/restliches/auger.png}
%             \end{figure}
%             
%             \TODO{Bildquellen einfügen: TBH-Skript S.93,94}
%             \end{quote}
%         
%         Die für die Emissiosmessung verwendete Diode besteht aus Silizium,
%         welcher ein indirekter Halbleiter ist und dessen Rekombinationen
%         hauptsächlch nichtstrahlend sind. Wie aber schon bei den Halbleitertypen
%         erwähnt, kann es bei indirekten Halbleitern auch mit sehr geringer
%         Wahrscheinlichkeit zu strahlender Rekombination kommen. Da die Abgabe
%         von thermischer Energie zu einem Temperaturunterschied in der Diode
%         führen würde, bräuchte man bei den Emissionsmessung sehr feine und genau
%         Thermometer, welche im $\mu m$-Bereich nicht realisierbar sind. Daher
%         wird die stochastisch in geringer Menge vorhandene strahlende
%         Rekombination betrachtet um über die mit lichtempfindlichen
%         Kameras gemessene Lichtintensitäten auf die gesuchte Diffusionslänge
%         schließen zu können.
%         
%         \end{quote}
%         
%         \vspace{1.5em}
%         
%         Emission entsteht bei der Rekombination eines freien Ladungsträgers,
%         welches nur in Durchlassrichtung bei einer Diode erwartet wird. Dabei
%         werden die Majoritäten ins entgegen gesetztes Gebiet injeziert und
%         rekombinieren dort als Minoritäten mit den oppositär gepolten
%         Ladungsträgern. Die dabei freigesetzte Energie kann dann als Emission
%         wargenommen werden. Emission kann aber auch in Sperrrichtung entstehen.
%         Die sogennante Feldemission findet dabei ausschließlich in der
%         ausgebreiteten Raumladungszone statt. Sobald eine Sperrspannung an der
%         Diode angelegt wird, werden die freien Ladungsträger aus der
%         Raumladungszone gesaugt und erfahren dabei eine kinetische Energie,
%         dessen Stärke von der Größe der Sperrspannung abhängt. Während der
%         beschleunigten Bewegung der Ladungsträger aus der Raumladungszone,
%         können diese mit Gitteratomen zusammenprallen und Elektronen aus diesem
%         Atom auf ein höheres Energieniveau anheben. Wie bei der nichtstrahlenden
%         Rekombination entsteht bei diesem Vorfall hauptsächlich thermische
%         Energie,es kann aber auch, mit einer sehr kleinen wahrscheinlichkeit
%         eine strahlende Rekombination vorkommen.\\
%         Die Feldemission ist im Vergleich zu der Emission in Flussrichtung
%         unbedeutend klein. Daher ist für die Intensitätsmessung nur eine
%         Emission in Flussrichtung relevant.
%         
%         \vspace{1.5em}
%         
%         \subsection{Messaufbau}
%         \begin{quote}
%         
%         Die Emissionsmessung erfolgt, wie in der Einleitung erwähnt, durch eine
%         Intensitätsmessung während der Rekombinationen in Flussrichtung der
%         Diode. Die Messung wird mithilfe des Phemos 1000 durchgeführt, welches
%         ein Prüfgerät in der Halbleitertechnik ist und hauptsächlich für
%         Fehlerdetektion dient. Um diese geringe Lichtintensität
%         messen zu können, braucht man eine lichtempfindliche CCD Kamera. Diese
%         ist im Phemos integriert und muss während der gesamten Messung auf
%         $-50$°C gekühlt werden. So werden thermische Rauscheinflüsse der sehr 
%         empfindlichen Kamera vermieden um das Ergebnis nicht zu verfälschen.\\
%         
%         Der Wafer, mit mehreren Diodenstrukturen, wird in dem Innenraum des
%         Phemos auf einer Vakuumplatte fixiert, sodass der erwünschte Bereich des
%         Wafers mit dem passenden Objektiv vergrössert werden kann. Diese
%         Vergrößerung hilft bei der Kontaktierung des p- und des n-Pads der Diode
%         anhand Nadelspitzen, welche in der unteren Abbildung im Licht des
%         Mikroskops zu erkennen sind.
%         
%         \begin{center}
%                 \begin{tabular}{ll}
%     
%                 \hspace{-8em}
%                     \begin{minipage}{0.6\textwidth}
%     
%                         \begin{figure}[H]
%                             \label{fig:}
%                             \includegraphics[scale=0.7, trim = 0cm 0cm 0cm
%                             0cm, clip]{./Emissionsbilder/restliches/phemos1.JPG}
%                             %FIXME [width=640px,
%                              %height=474px]
%                             \caption{Innenraum des Phemos}
%                         \end{figure}
%     
%                     \end{minipage}
%                     \begin{minipage}{0.6\textwidth}
%     
%                         \begin{figure}[H]
%                             \label{fig:}
%                             \includegraphics[scale=0.7, trim = 0cm 0cm 0cm
%                             0cm, clip]{./Emissionsbilder/restliches/phemos2.JPG}
%                             %FIXME [width=640px,
%                              %height=474px]
%                             \caption{kontaktierte Diode auf dem Wafer}
%                         \end{figure}
%                     \vspace{-1.5em}
%     
%                     \end{minipage}
%     
%                 \end{tabular}
%                 \end{center}
%         
%         \vspace{2em}
%         
%         Anhand eines Live-Bilds der CCD Kamera, kann man bis zum Start der
%         Messung den fokussierten Bereich bepannungsquelle für den
%         Durchlassbetrieb der Diode wurde der HP4145A, ein sensibeles
%         Analysegerät für Halbleiterbauelemente, verwendet, mit welchem eine
%         konstante Spannung für eine einstellbare Zeitspanne eingestellt werden
%         kann. Somit konnte man fest davon ausgehen, dass die Diode ohne einen
%         Spannungseinbruch konstant in Durchlassrichtung betrieben wurde.\\
%         Die Messdauer der Emission wurde mithilfe der Phemos-Software
%         eingestellt. Mit welchen Durchlassspannungen und wie lange gemessen
%         wurde, wird in der Auswertung der Messergebnisse angegeben.
%         
%         \vspace{1em}
%         
%         Am Ende jeder Messung wurde noch eine Dark-Substraction durchgeführt.
%         Bei dieser Dark-Substraction wird die Durchlassspannung abgeschaltet,
%         wodurch die Diode nicht mehr als aktiv betrachtet wird. Die Emissionsmessung wird mit der
%         gleichen Messdauer und deaktivierter Diode nochmal durchgeführt und von
%         der eigentlichen Emissionsmessung subtrahiert, damit jegliche
%         Rauscheinflüsse des Phemos selbst aus den Messergebnissen ausgeschlossen
%         werden können. 
%         
%         \end{quote}
%         
%         \subsection{Messergebnisse}
%         \begin{quote}
%         
%         Die Active-Area- und Kontaktfenstermasken wurden so erstellt, dass der
%         Wafer am Ende der Produktion Dioden auf jedem Die besaß, die für diese
%         Emissionsmessung hergestellt wurden. Diese Emissions-Dioden haben die
%         Besonderheit, dass zwischen p- und n-Pad das Substrat nicht mit Siliziumoxid beschichtet
%         ist, damit an den pn-Übergang zwischen n-Gebiet und p-Substrat
%         unverfälschte Emission gemessen werden kann.\\ 
%         Man hätte annehmen können, dass dort, wo Oxid auf dem pn-Übergang lag,
%         keine Emission aus dem Wafer austreten und gemessen werden kann. Die Praxis bewies aber das
%         Gegenteil. Unabhängig von der Auswertung der Messung
%         konnte an den pn-Übergängen mit Siliziumoxid darüber eine viel stärkere
%         Emission gemessen werden. Der Grund dafür liegt in dem Brechungsindex
%         und der Dicke der Oxidschicht, denn durch die Reflexionen der Strahlung an
%         den Übergängen von Silizium zu Siliziumoxid und Siliziumoxid zur Luft
%         werden die Strahlen so reflektiert, dass konstruktive Interferenz
%         zwischen den Lichtwellen entsteht und somit die Emission als viel
%         stärker wahrgenommen wird. Da dies nicht die wirkliche Intensität der
%         Emission am pn-Übergang unserer Diode wiederspiegelt, wurden diese mit
%         Oxid überlagerten Stellen in der Emissionsmessung vernachlässigt.\\
%         
%         Gemessen wurde nur der Bereich von der Grenze des n-Gebiets bis in
%         das p-Substrat hinein. Als Hilfe dafür wurde das jeweilige Emissionsbild
%         der Phemos-Software verwendet. Diese zeigte mit farblichen Unterschieden
%         (rot = starke Emission, blau = schwache Emission) wo der sinnvolle
%         Messbereich begann und endete. Die folgenden Abbildungen zeigen die
%         Diode vor und nach der Emissionsmessung auf dem Die in Zeile $6$ und
%         Spalte $3$ des Wafers. Zunächst wurde eine große, dann eine kleinere
%         Emissions-Diode ausgemessen. Bei beiden betrug die Durchlassspannung
%         $1,2\ V$ bei $25\ mA$ und einer Messdauer von $4s$.
%         
%          \begin{center}
%                 \begin{tabular}{ll}
%     
%                 \hspace{-10em}
%                     \begin{minipage}{0.6\textwidth}
%     
%                         \begin{figure}[H]
%                             \label{fig:}
%                             \includegraphics[scale=0.25, trim = 0cm 0cm 0cm
%                             0cm,
%                             clip]{./Emissionsbilder/eins/nach_Kontaktierung_vorMessung.jpg}
%                             %FIXME [width=640px, height=474px]
%                             \caption{kontaktierte Diode, Live-Bild der CCD
%                             Kamera}
%                         \end{figure}
%     
%                     \end{minipage}
%                     \begin{minipage}{0.6\textwidth}
%     
%                          \begin{figure}[H]
%                             \label{fig:}
%                             \includegraphics[scale=0.25, trim = 0cm 0cm 0cm
%                             0cm,
%                             clip]{./Emissionsbilder/eins/nach_Emission_mit_Distanzen.jpg}
%                             %FIXME [width=640px, height=474px]
%                             \caption{selbe Diode nach der Emissionsmessung}
%                         \end{figure}
%                    \vspace{-1.5em}
%     
%                     \end{minipage}
%     
%                 \end{tabular}
%                 \end{center}
%                 
%         \vspace{2em}
%         
%         Der gelbe Strich wurde anhand der Maus gezogen und zeigt den Bereich im
%         Substrat, welcher genauer ausgewertet wurde. Mehr dazu folgt in der
%         Auswertung.\\
%         
%         Als nächstes folgen die Abbildungen der kleineren Emissions-Diode auf
%         dem selben Die:
%              
%         
%          \begin{center}
%                 \begin{tabular}{ll}
%     
%                 \hspace{-10em}
%                     \begin{minipage}{0.6\textwidth}
%     
%                         \begin{figure}[H]
%                             \label{fig:}
%                             \includegraphics[scale=0.25, trim = 0cm 0cm 0cm
%                             0cm,
%                             clip]{./Emissionsbilder/zwei/nack_Kontaktierung.jpg}
%                             %FIXME [width=640px, height=474px]
%                             \caption{kontaktierte Diode, Live-Bild der CCD
%                             Kamera}
%                         \end{figure}
%     
%                     \end{minipage}
%                     \begin{minipage}{0.6\textwidth}
%     
%                          \begin{figure}[H]
%                             \label{fig:}
%                             \includegraphics[scale=0.25, trim = 0cm 0cm 0cm
%                             0cm,
%                             clip]{./Emissionsbilder/zwei/nach_Emissionsmessung_Intensitat_Distanz.jpg}
%                             %FIXME [width=640px, height=474px]
%                             \caption{selbe Diode nach der Emissionsmessung}
%                         \end{figure}
%                    \vspace{-1.5em}
%     
%                     \end{minipage}
%     
%                 \end{tabular}
%                 \end{center}
%                 
%         \vspace{2em}
%         
%         Nun wurde noch ein Die am Rand des Wafers ausgemessen um nach
%         Unterschieden zwischen den Messergebnissen kontrolliert werden zu
%         können. Auch auf dem Die in Zeile $15$ und Spalte $8$ wurde erst eine
%         große, dann eine kleine Emissions-Diode untersucht. Die Durchlassspannung bei der
%         großen Diode betrug $1,2\ V$ bei einem Strom von $25\ mA$. Die Messdauer
%         wurde aus $3s$ variiert.
%         
%         
%          \begin{center}
%                 \begin{tabular}{ll}
%     
%                 \hspace{-10em}
%                     \begin{minipage}{0.6\textwidth}
%     
%                         \begin{figure}[H]
%                             \label{fig:}
%                             \includegraphics[scale=0.25, trim = 0cm 0cm 0cm
%                             0cm,
%                             clip]{./Emissionsbilder/drei/nach_Kontaktierung.jpg}
%                             %FIXME [width=640px, height=474px]
%                             \caption{kontaktierte Diode, Live-Bild der CCD
%                             Kamera}
%                         \end{figure}
%     
%                     \end{minipage}
%                     \begin{minipage}{0.6\textwidth}
%     
%                          \begin{figure}[H]
%                             \label{fig:}
%                             \includegraphics[scale=0.25, trim = 0cm 0cm 0cm
%                             0cm,
%                             clip]{./Emissionsbilder/drei/nach_Emissionsmessung_Distanz.jpg}
%                             %FIXME [width=640px, height=474px]
%                             \caption{selbe Diode nach der Emissionsmessung}
%                         \end{figure}
%                    \vspace{-1.5em}
%     
%                     \end{minipage}
%     
%                 \end{tabular}
%                 \end{center}
%                 
%         \vspace{2em}
%         
%         Die Durchlassspannung der kleineren Diode betrug $1,5\ V$ bei $30\ mA$
%         Strom. Die Messdauer wurde auf $4s$ gestellt. Vor und nach der Messung
%         wurden folgende Bilder aufgezeichnet:
%         
%          
%          \begin{center}
%                 \begin{tabular}{ll}
%     
%                 \hspace{-10em}
%                     \begin{minipage}{0.6\textwidth}
%     
%                         \begin{figure}[H]
%                             \label{fig:}
%                             \includegraphics[scale=0.25, trim = 0cm 0cm 0cm
%                             0cm,
%                             clip]{./Emissionsbilder/vier/nach_Kontaktierung.jpg}
%                             %FIXME [width=640px, height=474px]
%                             \caption{kontaktierte Diode, Live-Bild der CCD
%                             Kamera}
%                         \end{figure}
%     
%                     \end{minipage}
%                     \begin{minipage}{0.6\textwidth}
%     
%                          \begin{figure}[H]
%                             \label{fig:}
%                             \includegraphics[scale=0.25, trim = 0cm 0cm 0cm
%                             0cm,
%                             clip]{./Emissionsbilder/vier/nach_Emission_Distanz.jpg}
%                             %FIXME [width=640px, height=474px]
%                             \caption{selbe Diode nach der Emissionsmessung}
%                         \end{figure}
%                    \vspace{-1.5em}
%     
%                     \end{minipage}
%     
%                 \end{tabular}
%                 \end{center}
%                 
%         \vspace{2em}
%         
%         Bislang ist nur eine Emissionsintensität zu sehen. Mithilfe der
%         Phemos-Software konnte daraus ein Intensitätsprofil erstellt werden,
%         woraus am Ende dann auf die Diffusionslänge zurück geschlossen werden
%         kann.
%         
%         \vspace{1.5em}
%         
%         Es wurde auch eine Fingerstrukter-Diode für die Messung verwendet,
%         welche sich ebenfalls auf dem äußeren Die befand. Eine
%         Diffusionslängenberechnung wurde für diese Diode nicht durchgeführt, da
%         am Phemos ein Objektiv gewählt wurde, mit der die gesamte Diode auf des
%         Bildschirm sichtbar war. Daher sind die Fingerstrukturen zwar sichtbar,
%         aber die pn-Übergänge zwischen n-Finger und dem p-Substrat dazwischen,
%         die Grenze an der die Emission erwartet wird, sind viel zu klein
%         abgebildet, als dass eine Intensitätskurve mit der geringen Vergrößerung
%         sinnvoll wäre. Daher sind nur die Kontaktierung und das Ergebniss der
%         Emissionsmessung im Protokoll aufgeführt. Die Durchlassspannung während
%         der Messung betrug $5\ V$ bei einem Strom von $20\ mA$ und einer
%         Messdauer von $4s$. 
%         
%         
%             \begin{center}
%                 \begin{tabular}{ll}
%     
%                 \hspace{-10em}
%                     \begin{minipage}{0.6\textwidth}
%     
%                         \begin{figure}[H]
%                             \label{fig:}
%                             \includegraphics[scale=0.25, trim = 0cm 0cm 0cm
%                             0cm,
%                             clip]{./Emissionsbilder/fuenf/nach_Kontaktierung.jpg}
%                             %FIXME [width=640px, height=474px]
%                             \caption{kontaktierte Diode, Live-Bild der CCD
%                             Kamera}
%                         \end{figure}
%     
%                     \end{minipage}
%                     \begin{minipage}{0.6\textwidth}
%     
%                          \begin{figure}[H]
%                             \label{fig:}
%                             \includegraphics[scale=0.25, trim = 0cm 0cm 0cm
%                             0cm,
%                             clip]{./Emissionsbilder/fuenf/SuperImpose.jpg}
%                             %FIXME [width=640px, height=474px]
%                             \caption{selbe Diode nach der Emissionsmessung}
%                         \end{figure}
%                    \vspace{-1.5em}
%     
%                     \end{minipage}
%     
%                 \end{tabular}
%                 \end{center}
%                 
%         \vspace{2em}
%         
%         Man kann erkennen, dass die n-Finger, welche von links nach rechts
%         Richtung p-Pad reichen, die Positionen der pn-Übergänge bilden. Daher
%         wäre eigentlich eine Emission entlang der Ränder der n-Finger zu erwarten.
%         Eine Erklärung, warum aber die größte Emission nur an der Stelle
%         auftritt, wo der Abstand zwischen n- und p-Pad am geringsten ist, wäre, dass die
%         Rekombination hinter der Raumladungszone dort viel größer ist, wo der
%         Halbleiter den geringsten Widerstand aufweist. Der Strom könnte also
%         dort am größten werden, wodurch eine stärkere Emission erklärt wäre.
%         
%         \end{quote}
%         
%         
%         \subsection{Auswertung}
%         \begin{quote}
%         
%         Die Phemos-Software ist in der Lage, aus der Messung eine Wertetabelle
%         zu erstellen, welche die jeweilige Strahlungsintensität an dem
%         betrachteten Punkt im Substrat ausgibt. Mit diesen Werten konnte die
%         Strahlungsintensität, welche den Erwartungen nach einen exponentiellen
%         Abstieg aufweisen sollte, nachgebildet werden. Um zur gesuchten
%         Diffusionslänge zu gelangen wird diese Kurve nun logarithmisch
%         aufgetragen. Denn wenn man sich nochmal die relevante Proportionalität
%         ansieht:
%         
%         \begin{equation*}
%         \begin{split}
%             I(x) \sim \ \Delta n \sim \ exp(-\frac{x}{L_n})\\
%             log(I(x)) \sim -\frac{x}{L_n} 
%         \end{split}
%         \end{equation*}
%         
%         dann erkennt man, dass eine Logarithmierung des Exponentialtherms nur
%         das Argument übrig lässt. Daher sollte aus jeder Intensitätskurve
%         eine absteigende Gerade entstehen, welche eine Steigung von $-\frac{1}{L_n}$
%         besitzt. Über den Reziprokwert dieser Steigung kann man dann die
%         Diffusionslänge $L_n$ der Dioden berechnen.
%         
%         \vspace{1em}
%         
%         Beginnend mit der ersten Messung wird die Intensitätskurve mit dem
%         exponentiellen und dem logarithmischen Verlauf geplottet und analysiert:
%         
%         \begin{figure}[H]
%                     \centering
%                         \includegraphics[scale=0.53, trim = 1cm 6cm 1.5cm 8cm,
%                         clip]{./Emissionsbilder/eins/Intensitatsmessung.pdf}
%                         \caption{Intensitätskurve der 1.Messung, exponentiell
%                         und logarithmisch}
%                             \label{fig:./Emissionsbilder/eins/Intensitatsmessung.pdf}
%         \end{figure}
%             
%         
%         Die blaue Kurve zeit den exponentiellen, die rote Kurve den
%         logarithmischen Verlauf. Aus dieser wird nun der Kehrwert der Steigung
%         ermittelt, welche eine Diffusionslänge von $304,2\ \mu m$ liefert.\\
%         
%         Die zweite Messung an der kleineren Diode im selben Die liefert
%         folgende Intensitätsverläufe
%         
%         \begin{figure}[H]
%                     \centering
%                         \includegraphics[scale=0.53, trim = 1cm 6cm 1.5cm 8cm,
%                         clip]{./Emissionsbilder/zwei/Intensitatsmessung.pdf}
%                         \caption{Intensitätskurve der 2.Messung, exponentiell
%                         und logarithmisch}
%                             \label{fig:./Emissionsbilder/zwei/Intensitatsmessung.pdf}
%         \end{figure}
%         
%         und, mit der analogen Berechnung, eine Diffusionslänge von $304,6\ \mu
%         m$.\\
%         
%         Die dritte Messung an der großen Emissions-Diode am Rand des Wafers,
%         liefert folgende Intensitätsverläufe
%         
%         \begin{figure}[H]
%                     \centering
%                         \includegraphics[scale=0.53, trim = 1cm 6cm 1.5cm 8cm,
%                         clip]{./Emissionsbilder/drei/Intensitatsmessung.pdf}
%                         \caption{Intensitätskurve der 3.Messung, exponentiell
%                         und logarithmisch}
%                             \label{fig:./Emissionsbilder/drei/Intensitatsmessung.pdf}
%         \end{figure}
%         
%         und eine Diffusionslänge von $304,3\ \mu m$, wohingegen die kleinere
%         Diode auf dem äußeren Die diese Kurven
%         
%         \begin{figure}[H]
%                     \centering
%                         \includegraphics[scale=0.53, trim = 1cm 6cm 1.5cm 8cm,
%                         clip]{./Emissionsbilder/vier/Intensitatsmessung.pdf}
%                         \caption{Intensitätskurve der 4.Messung, exponentiell
%                         und logarithmisch}
%                             \label{fig:./Emissionsbilder/vier/Intensitatsmessung.pdf}
%         \end{figure}
%         
%         mit der Diffusionslänge von erneut $304,6\ \mu m$ wiedergibt.
%         
%         \vspace{1.5em}
%         
%         Obwohl diese Werte weit über dem Erwartungswert einer Diffusionslänge
%         von $150 - 200\ \mu m$ liegen, spricht es für die Messung, dass alle
%         Werte ziemlich gleich groß sind. Da alle gemessenen Dioden sich auf
%         demselben Wafer befinden und somit alle gleich dotiert und behandelt wurden, ist
%         es auch verständlich, dass die Diffusionslängen nicht bedeutend viel
%         voneinander abweichen.\\
%         Die vermutete Ursache für die dennoch zu großen Diffusionslängen könnte
%         in dem Dotierungsschritt während der Wafer-Herstellung liegen. Eventuell
%         wurde im Reinraum ein unbemerkter Fehler begangen, der zu größeren
%         $L_n$s führte.
%         
%         \end{quote}
%     
% \end{quote} %sec Emissionsmessung

%--------------------------------------------------------------------
%--------------------------------------------------------------------

\end{document}
