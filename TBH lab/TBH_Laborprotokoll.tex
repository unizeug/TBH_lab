\newcommand{\institut}{}
\newcommand{\fachgebiet}{Halbleiterbauelemente}
\newcommand{\veranstaltung}{Praktikum Technologie und Bauelemente der Halbleitertechnik}
\newcommand{\pdfautor}{Dirk Barbendererde (321 836), Thomas Kapa (), Alona Siebert (), Özgü Dogan (326 048)}
\newcommand{\autor}{Dirk Barbendererde (321 836)\\ Thomas Kapa ()\\ Alona Siebert ()\\ \"Ozg\"u Dogan (326 048)}
\newcommand{\pdftitle}{Praktikum\ Technologie und Bauelemente der
Halbleitertechnik}
\newcommand{\prototitle}{Praktikum Technologie und Bauelemente der Halbleitertechnik}
\newcommand{\aufgabe}{}

\newcommand{\gruppe}{Gruppe 1}
\newcommand{\betreuer}{Betreuer:\\ Clemens Helfmeier\\ Philipp Scholz}
 

\input{../../packages/tu_header_9}



\setcaptionwidth{7.5cm}

\begin{document}


%     \lstinputlisting{./praktikum6.sce}



%---------------------------------------------------------------------
%---------------------------------------------------------------------
%---------------------------------------------------------------------

\section{Kennlinie}
\begin{quote}
    \subsection{Messaufbau}
    \begin{quote}
        Die erste Untersuchung die wir unseren Dioden unterzogen haben war die Aufnahme der Kennlinien. Dazu muss der
        pn-Übergang unter einem Lichtmikroskop kontaktiert werden. Die beiden Gebiete werden jeweils mit einem Kanal des
        Parameter-Analyzer 4145A von HP verbunden, der einen sweep der Spannung von \SI{-1}{\volt} bis \SI{1}{\volt} macht
        und dabei den Strom misst.\\
        Diese Messung haben wir auf verschiedenen Dies gurchgeführt um diese miteinander vergleichen
        zu können.
    \end{quote} % Sub: Messaufbau
    
    
    \subsection{Auswahl der zu messenden Diode}
    \begin{quote}
        Um möglichst aussagekräftige Messwerte zu bekommen haben wir uns dazu entschieden jeweils die größte Diode auf
        den Dies für unsere Messungen zu benutzen. Davon haben wir uns versprochen, dass sowohl Durchlass- als
        auch Sperrstrom am größten sind und somit besser zu messen sein würden als bei kleineren Dioden.
        Nach der Auswertung anderer Messungen und \TODO{einigen weiteren Überlegungen} stellte sich aber heraus, dass
        dies ein Trugschluss war. Eine Größere quadratische Diode führt zwar sehr wohl einen höheren Strom als eine
        kleinere Quadratische, bei Durchlass- und Sperrstrom kommt es aber nicht auf die Größe sondern auf die
        pn-Übergangs-Fläche an. Daher kann eine der Fläche nach nur halb so große Diode, mit verzahnter Fingerstruktur,
        noch mehr Strom führen als die flächenmäßig größte Diode für die wir uns entschieden haben. Deshalb wird am Ende
        der Auswertung zusätzlich noch eine solche fingerstruktur betrachtet.
        
    \end{quote} % Subsub: Auswahl der zu messneden Diode
    
    
    \subsection{Erwartete Kennlinie}
    \begin{quote}
        
        Bei intakten Dioden erwarten wir im Grunde eine Kennlinie nach der Formel von Shockley:
        
        
        
        \begin{equation}
        \begin{split}
            J_{ges} = J_0 \cdot (e^ \frac{u}{U_t} -1)
        \end{split}
        \label{equ:Schockley}
        \end{equation}
        
        
        jedoch inklusive realitätsbedingte Abweichungen.
        
        \begin{figure}[H]
            \centering
            \includegraphics[scale=0.7, trim = 0cm 0cm 0cm 0cm, clip]{KennlinienBilder/reale_Kennlinie}
            \caption{Reale kennlinie einer Diode}
            \label{fig:reale_Kennlinie}
        \end{figure}
        
        
        Wie in Abb. \ref{fig:reale_Kennlinie} logatythmisch dargestellt erwarten wir, dass nur der mittlere Teil
        zwischen \SI{0,2}{\volt} und \SI{0,5}{\volt} der Schockley-Gleichung folgt. Bei höheren Spannungen ist der
        Wiederstand der Bahngebiete nicht mehr vernachlässigbar klein und wird sich deutlich mtrommindernd auf die
        Kennlinie auswirken. Außerdem wird sich durch die nichteinhaltung der Shockleybedingungen die Rekombination in
        der Raumladungszone bemerkbar machen. Bei kleinen Spannungen (unter ca.
        \SI{0,2}{\volt}) fließt deutlich mehr Strom aufgrund von Rekombination als durch den Potentialunterschied,
        wodurch der mehr Strom fließt als nach Gl. \ref{equ:Schockley}.\\
        
        Die in Abb. \ref{fig:reale_Kennlinie} dargestellte mangelhafte Messgenauigkeit haben wir nicht erwartet. Wir
        sind davon ausgegangen, dass selbst bei sehr guten Dioden, also in diesem Fall welchen mit sehr kleinen
        Sperrströmen, der Parameter-Analyzer keine Probleme hat diese zu messen.\\
        \\
        Außerdem werden wir den Strom betragsmäßig über der Spannung auftragen, was es ermöglicht zur illustration auch
        negative Ströme logarythmisch dar zu stellen. In unserer Auswertung sind Ströme bei negatien Spannungen also als
        negative Ströme auf zu fassen.
        
        
        
    \end{quote} % Sub: Erwartete Kennlinie
    
    
    \subsection{Messergebnisse}
    \begin{quote}
        
        
        
        
    \end{quote} % Sub: Messergebnisse
    
    
    
    
    \subsection{Auswertung}
    \begin{quote}
        
    \end{quote} % Sub: Auswertung
    
    
    
    
    Quellen:
    kennlinie: Laborskript Technologie und Bauelemente der Halbleitertechnik (Aydin Asri, Arminius Kuckelt, Michael
    Sadowski, Julian Utehs und Tristan Visentin) S.66
    
    
    
    erst 
    
    
    Benennung der Dateien:\\
    Kennlinie_{Wavernummer}_Die_[{Zeile},{Spalte}]_Mess_{messung}.mat\\
    
    100 µA Strombegrenzung\\
    Welche Diode?
    Welcher Manipulator
    
    SMU1 = GND
    SMU2 = GND
    SMU3 = Var1
    
    Unterdiffusion
    
    unterspannung
    0.0 & 2.4
    0.5 & 3.8
    0.8 & 4.5
    1.0 & 5.0
    1.3 & 5.5
    1.8 & 6.7
    2.5 & 8.2
    3.0 & 9.2
    
    
    
    
    
    was wir messen (querschnitt)
    was für ne Kennlinie wir erwarten
    
    änderung der Kennlinie mit beleuchtung
    messung mit beleuchtung gegen messung unbeleuchtet
    
    verschiedene messungen auf dem wafer unterchiede?
    
    sperrverhalten
    
    
    \end{quote} %sec Kennlinie

%--------------------------------------------------------------------
%--------------------------------------------------------------------

\section{Schaltverhalten}
\begin{quote}
    
\end{quote} %sec Schaltverhalten

%--------------------------------------------------------------------
%--------------------------------------------------------------------

\section{Emissionsmessung}
\begin{quote}
    
    Eine Emissionsmessung ist in der Halbleitertechnologie insofern interessant,
    weil sie Erkenntnisse über charakteristische Eigenschaften des Halbleiters,
    in unserem Fall die selbst hergestellte Diode, liefert. Dazu gehören die
    Ladungsträgerlebensdauer $\tau_{n,p}$ und die Diffusionslänge $L_{n,p}$. Auf
    die Lebensdauer lässt sich mithilfe der Diffusionslänge und des
    Diffusionskoeffizienten schließen. Dieser ist ein materialabhängiger Wert,
    welcher von uns nicht weiter beachtet wird. Der folgende Zusammenhang hilft
    bei der Berechnung von $\tau_{n,p}$:
    
    \begin{equation*}
        \begin{split}
            L_{n,p} = \sqrt{\tau_{n,p} \cdot D_{n,p}} 
        \end{split}
    \end{equation*}
    
    Ziel unserer Messung aber war die Bestimmung der Diffusionslänge in unserer
    Diode. Dieser wurde über die realisierte Intensitätsmessung anhand der
    Emissionen in der Diode ermittelt. Dabei wurden folgende
    Proportionalitätsverhätnisse verwendet:
    
    \begin{equation*}
        \begin{split}
            I(x) \sim \ \Delta n \sim \ exp(-\frac{x}{L_n}) 
        \end{split}
    \end{equation*}
    
    Hierbei werden Strahlungsintensität ins Verhältnis mit der\\
    Minoritätsüberschussladungträgerkonzentration und dieser wiederum 
    ins Verhältnis mit einem Exponentialtherm gesetzt. Daher kann man die
    Intensitätsmessung direkt mit diesem Therm in Verbindung setzen, 
    welcher in seinem Argument die gesuchte Diffusionslänge beinhaltet. Weiteres
    zur Berechnung des $L_n$ steht in der Auswertung der Messung.\\
    
    Um aber die Intensitätsmessung verstehen zu können müssen einige
    grundlegende Theorien der Halbleiter bezüglich ihrer Typen und ihrer
    Rekombinationsarten behandelt und nachvollzogen werden.\\ 
    Daher gibt es vor der Versuchsdurchführung und der Auswertung zunächst eine
    kleine Exkursion in den theoretischen Bereich.
    
        \subsection{Direkter und indirekter Halbleiter }
        \begin{quote}
        
        Es gibt zwei Arten von Halbleitern, die direkten und die indirekten
        Halbleiter. Diese unterscheiden sich darin, dass die Rekombination eines
        Ladungsträgers aus dem Leitungs- in das Valenzband unterschiedliche
        Vorraussetzungen erfordert.
            
            \subsubsection{direkter Halbleiter}
            \begin{quote}
            Die Rekombination bei einem direkten Halbleiter ist relativ simpel.
            Ein freies Elektron braucht dabei nur, unter Abgabe der jeweiligen
            Energie, den Bandabstand zwischen Leitungs- und Valenzband zu
            überqueren.
            
            \begin{figure}[H]
                    \centering
                        \includegraphics[scale=0.72, trim = 1cm 0cm 1.5cm 0cm,
                        clip]{./Emissionsbilder/restliches/direkt.png}
                        \caption{Rekombination bei einem direkten Halbleiter}
                            \label{fig:./Emissionsbilder/restliches/direkt.png}
            \end{figure}
            
            
            \end{quote}       
        
            \subsubsection{indirekter Halbleiter}
            \begin{quote}
            Die Rekombination bei einem indirekten Halbleiter erfordert neben
            einem Energieunterschied auch einen Impulsunterschied, damit das
            Elektron auf dem Valenzband auftreffen kann. Dieser
            Impulsunterschied ist in der folgenden Abbildung zu erkennen.
            
            \begin{figure}[H]
                    \centering
                        \includegraphics[scale=0.73, trim = 1cm 0cm 1.5cm 0cm,
                        clip]{./Emissionsbilder/restliches/indirekt.png}
                        \caption{Rekombination bei einem indirekten Halbleiter}
                            \label{fig:./Emissionsbilder/restliches/indirekt.png}
            \end{figure}
            
            \TODO{Bildquellen, TBH-Skript S.92 einfügen}
            \end{quote}       
            
            Ein weiterer Unterschied zwischen direkten und indirekten
            Halbleitern ist der, dass direkte Halbleiter hauptsächlich
            strahlend rekombinieren, wohingegen indirekte Halbleiter
            stochastisch betrachtet fast nur nichtstrahlend rekombinieren.
            Dennoch findet mit sehr geringer Wahrscheinlichkeit auch vereinzelt
            strahlende Rekombination in den indirekten Halbleitern statt.\\
            Bevor weiter darauf eingegangen wird, in wie fern dies für die
            Emissionsmessung von Bedeutung ist, werden die Rekombinationsarten,
            strahlend und nichtstrahlend, wiederholt.
            
        \end{quote}
        
        \subsection{Rekombinationsmechanismen}
        \begin{quote}
        
        Eine Rekombination erfolgt stets unter Abgabe von einer
        Energiedifferenz. Dabei wird unterschieden, ob diese Energie in Form
        eines Lichtquants oder von Wärme abgegeben wird, wodurch auch die
        Beschreibung strahlend oder nichtstrahlend entsteht.\\
        Nun folgen je ein Beispiel für diese Machanismen.
        
            \subsubsection{strahlende Rekombination}
            \begin{quote}
            
            Die strahlende Rekombination, hauptsächlich bei direkten Halbleitern
            zu sehen, besagt, dass der rekombinierende Ladungsträger die
            Energiedifferenz, welche er zurücklegt, in Form eines Photons
            freigibt. Die Energie dieses Photons beträgt genau die Energie des
            Bandabstands zwischen den beiden Bändern. Anders ausgedrückt kann
            man sie auch Rekombinationsenergie nennen:
                    
            \begin{equation*}
                \begin{split}
                    W_{Rek} = h \cdot \nu 
                \end{split}
            \end{equation*}
            
            Diese Rekombinationsenergie setzt sich aus des Produkt aus dem
            Planckschen Wirkungsquantums $h$ und die Frequenz des entstehenden
            Lichts $\nu$.\\
            
            Ein Beispiel für die strahlende Rekombination ist die
            Band-Band-Rekombination, welche in der folgenden Abbildung
            dargestellt wurde:
            
            \begin{figure}[H]
                    \centering
                        \includegraphics[scale=0.68, trim = 1cm 0cm 1.5cm 0cm,
                        clip]{./Emissionsbilder/restliches/bandband.png}
                        \caption{strahlende Band-Band-Rekombination}
                            \label{fig:./Emissionsbilder/restliches/direkt.png}
            \end{figure}
            
            Man erkennt ein Elektron, welches beim Rekombinieren, die bereits
            erwähnte Rekombinationsernergie in Form eines Photons abgibt. Es
            kann aber auch vorkommen, dass die abgegebene Energie an ein
            weiteres Eektron im Leitungsband abgegeben wird, wodurch dieser auf
            ein höheres Energieniveau im Leitungsband angehoben und wieder runterfallen 
            kann. Diese Möglichkeit ist bei der strahlenden Rekombination die
            unwarscheinlichere Variante.
            
            \end{quote}
            
            \subsubsection{nichtstrahlende Rekombination}
            \begin{quote}
            
            Die nichtstrahlende Rekombination findet hauptsächlich bei
            indirekten Halbleitern statt. Dabei wird die Rekombinationsenergie
            an ein weiteres Elektron im Leitungsband abgegeben, welches auf ein
            höheres Energieniveau angehoben wird und unter Abgabe von
            thermischer Energie wieder runterfallen kann.\\
            Als ein Beispiel für die nichtstrahlende Rekombination wird die
            Augerrekombination dargestellt:
            
            \begin{figure}[H]
                    \centering
                        \includegraphics[scale=0.78, trim = 1cm 0cm 1.5cm 0cm,
                        clip]{./Emissionsbilder/restliches/auger.png}
                        \caption{nichtstrahlende Auger-Rekombination}
                            \label{fig:./Emissionsbilder/restliches/auger.png}
            \end{figure}
            
            \TODO{Bildquellen einfügen: TBH-Skript S.93,94}
            \end{quote}
        
        Die für die Emissiosmessung verwendete Diode besteht aus Silizium,
        welcher ein indirekter Halbleiter ist und dessen Rekombinationen
        hauptsächlch nichtstrahlend sind. Wie aber schon bei den Halbleitertypen
        erwähnt, kann es bei indirekten Halbleitern auch mit sehr geringer
        Wahrscheinlichkeit zu strahlender Rekombination kommen. Da die Abgabe
        von thermischer Energie zu einem Temperaturunterschied in der Diode
        führen würde, bräuchte man bei den Emissionsmessung sehr feine und genau
        Thermometer, welche im $\mu m$-Bereich nicht realisierbar sind. Daher
        wird die stochastisch in geringer Menge vorhandene strahlende
        Rekombination betrachtet um über die mit lichtempfindlichen
        Kameras gemessene Lichtintensitäten auf die gesuchte Diffusionslänge
        schließen zu können.
        
        \end{quote}
        
        \vspace{1.5em}
        
        Emission entsteht bei der Rekombination eines freien Ladungsträgers,
        welches nur in Durchlassrichtung bei einer Diode erwartet wird. Dabei
        werden die Majoritäten ins entgegen gesetztes Gebiet injeziert und
        rekombinieren dort als Minoritäten mit den oppositär gepolten
        Ladungsträgern. Die dabei freigesetzte Energie kann dann als Emission
        wargenommen werden. Emission kann aber auch in Sperrrichtung entstehen.
        Die sogennante Feldemission findet dabei ausschließlich in der
        ausgebreiteten Raumladungszone statt. Sobald eine Sperrspannung an der
        Diode angelegt wird, werden die freien Ladungsträger aus der
        Raumladungszone gesaugt und erfahren dabei eine kinetische Energie,
        dessen Stärke von der Größe der Sperrspannung abhängt. Während der
        beschleunigten Bewegung der Ladungsträger aus der Raumladungszone,
        können diese mit Gitteratomen zusammenprallen und Elektronen aus diesem
        Atom auf ein höheres Energieniveau anheben. Wie bei der nichtstrahlenden
        Rekombination entsteht bei diesem Vorfall hauptsächlich thermische
        Energie,es kann aber auch, mit einer sehr kleinen wahrscheinlichkeit
        eine strahlende Rekombination vorkommen.\\
        Die Feldemission ist im Vergleich zu der Emission in Flussrichtung
        unbedeutend klein. Daher ist für die Intensitätsmessung nur eine
        Emission in Flussrichtung relevant.
        
        \vspace{1.5em}
        
        \subsection{Messaufbau}
        \begin{quote}
        
        Die Emissionsmessung erfolgt, wie in der Einleitung erwähnt, durch eine
        Intensitätsmessung während der Rekombinationen in Flussrichtung der
        Diode. Die Messung wird mithilfe des Phemos 1000 durchgeführt, welches
        ein Prüfgerät in der Halbleitertechnik ist und hauptsächlich für
        Fehlerdetektion dient. Um diese geringe Lichtintensität
        messen zu können, braucht man eine lichtempfindliche CCD Kamera. Diese
        ist im Phemos integriert und muss während der gesamten Messung auf
        $-50$°C gekühlt werden. So werden thermische Rauscheinflüsse der sehr 
        empfindlichen Kamera vermieden um das Ergebnis nicht zu verfälschen.\\
        
        Der Wafer, mit mehreren Diodenstrukturen, wird in dem Innenraum des
        Phemos auf einer Vakuumplatte fixiert, sodass der erwünschte Bereich des
        Wafers mit dem passenden Objektiv vergrössert werden kann. Diese
        Vergrößerung hilft bei der Kontaktierung des p- und des n-Pads der Diode
        anhand Nadelspitzen, welche in der unteren Abbildung im Licht des
        Mikroskops zu erkennen sind.
        
        \begin{center}
                \begin{tabular}{ll}
    
                \hspace{-8em}
                    \begin{minipage}{0.6\textwidth}
    
                        \begin{figure}[H]
                            \label{fig:}
                            \includegraphics[scale=0.7, trim = 0cm 0cm 0cm
                            0cm, clip]{./Emissionsbilder/restliches/phemos1.JPG}
                            %FIXME [width=640px,
                             %height=474px]
                            \caption{Innenraum des Phemos}
                        \end{figure}
    
                    \end{minipage}
                    \begin{minipage}{0.6\textwidth}
    
                        \begin{figure}[H]
                            \label{fig:}
                            \includegraphics[scale=0.7, trim = 0cm 0cm 0cm
                            0cm, clip]{./Emissionsbilder/restliches/phemos2.JPG}
                            %FIXME [width=640px,
                             %height=474px]
                            \caption{kontaktierte Diode auf dem Wafer}
                        \end{figure}
                    \vspace{-1.5em}
    
                    \end{minipage}
    
                \end{tabular}
                \end{center}
        
        \vspace{2em}
        
        Anhand eines Live-Bilds der CCD Kamera, kann man bis zum Start der
        Messung den fokussierten Bereich bepannungsquelle für den
        Durchlassbetrieb der Diode wurde der HP4145A, ein sensibeles
        Analysegerät für Halbleiterbauelemente, verwendet, mit welchem eine
        konstante Spannung für eine einstellbare Zeitspanne eingestellt werden
        kann. Somit konnte man fest davon ausgehen, dass die Diode ohne einen
        Spannungseinbruch konstant in Durchlassrichtung betrieben wurde.\\
        Die Messdauer der Emission wurde mithilfe der Phemos-Software
        eingestellt. Mit welchen Durchlassspannungen und wie lange gemessen
        wurde, wird in der Auswertung der Messergebnisse angegeben.
        
        \vspace{1em}
        
        Am Ende jeder Messung wurde noch eine Dark-Substraction durchgeführt.
        Bei dieser Dark-Substraction wird die Durchlassspannung abgeschaltet,
        wodurch die Diode nicht mehr als aktiv betrachtet wird. Die Emissionsmessung wird mit der
        gleichen Messdauer und deaktivierter Diode nochmal durchgeführt und von
        der eigentlichen Emissionsmessung subtrahiert, damit jegliche
        Rauscheinflüsse des Phemos selbst aus den Messergebnissen ausgeschlossen
        werden können. 
        
        \end{quote}
        
        \subsection{Messergebnisse}
        \begin{quote}
        
        Die Active-Area- und Kontaktfenstermasken wurden so erstellt, dass der
        Wafer am Ende der Produktion Dioden auf jedem Die besaß, die für diese
        Emissionsmessung hergestellt wurden. Diese Emissions-Dioden haben die
        Besonderheit, dass zwischen p- und n-Pad das Substrat nicht mit Siliziumoxid beschichtet
        ist, damit an den pn-Übergang zwischen n-Gebiet und p-Substrat
        unverfälschte Emission gemessen werden kann.\\ 
        Man hätte annehmen können, dass dort, wo Oxid auf dem pn-Übergang lag,
        keine Emission aus dem Wafer austreten und gemessen werden kann. Die Praxis bewies aber das
        Gegenteil. Unabhängig von der Auswertung der Messung
        konnte an den pn-Übergängen mit Siliziumoxid darüber eine viel stärkere
        Emission gemessen werden. Der Grund dafür liegt in dem Brechungsindex
        und der Dicke der Oxidschicht, denn durch die Reflexionen der Strahlung an
        den Übergängen von Silizium zu Siliziumoxid und Siliziumoxid zur Luft
        werden die Strahlen so reflektiert, dass konstruktive Interferenz
        zwischen den Lichtwellen entsteht und somit die Emission als viel
        stärker wahrgenommen wird. Da dies nicht die wirkliche Intensität der
        Emission am pn-Übergang unserer Diode wiederspiegelt, wurden diese mit
        Oxid überlagerten Stellen in der Emissionsmessung vernachlässigt.\\
        
        Gemessen wurde nur der Bereich von der Grenze des n-Gebiets bis in
        das p-Substrat hinein. Als Hilfe dafür wurde das jeweilige Emissionsbild
        der Phemos-Software verwendet. Diese zeigte mit farblichen Unterschieden
        (rot = starke Emission, blau = schwache Emission) wo der sinnvolle
        Messbereich begann und endete. Die folgenden Abbildungen zeigen die
        Diode vor und nach der Emissionsmessung auf dem Die in Zeile $6$ und
        Spalte $3$ des Wafers. Zunächst wurde eine große, dann eine kleinere
        Emissions-Diode ausgemessen. Bei beiden betrug die Durchlassspannung
        $1,2\ V$ bei $25\ mA$ und einer Messdauer von $4s$.
        
         \begin{center}
                \begin{tabular}{ll}
    
                \hspace{-10em}
                    \begin{minipage}{0.6\textwidth}
    
                        \begin{figure}[H]
                            \label{fig:}
                            \includegraphics[scale=0.25, trim = 0cm 0cm 0cm
                            0cm,
                            clip]{./Emissionsbilder/eins/nach_Kontaktierung_vorMessung.jpg}
                            %FIXME [width=640px, height=474px]
                            \caption{kontaktierte Diode, Live-Bild der CCD
                            Kamera}
                        \end{figure}
    
                    \end{minipage}
                    \begin{minipage}{0.6\textwidth}
    
                         \begin{figure}[H]
                            \label{fig:}
                            \includegraphics[scale=0.25, trim = 0cm 0cm 0cm
                            0cm,
                            clip]{./Emissionsbilder/eins/nach_Emission_mit_Distanzen.jpg}
                            %FIXME [width=640px, height=474px]
                            \caption{selbe Diode nach der Emissionsmessung}
                        \end{figure}
                   \vspace{-1.5em}
    
                    \end{minipage}
    
                \end{tabular}
                \end{center}
                
        \vspace{2em}
        
        Der gelbe Strich wurde anhand der Maus gezogen und zeigt den Bereich im
        Substrat, welcher genauer ausgewertet wurde. Mehr dazu folgt in der
        Auswertung.\\
        
        Als nächstes folgen die Abbildungen der kleineren Emissions-Diode auf
        dem selben Die:
             
        
         \begin{center}
                \begin{tabular}{ll}
    
                \hspace{-10em}
                    \begin{minipage}{0.6\textwidth}
    
                        \begin{figure}[H]
                            \label{fig:}
                            \includegraphics[scale=0.25, trim = 0cm 0cm 0cm
                            0cm,
                            clip]{./Emissionsbilder/zwei/nack_Kontaktierung.jpg}
                            %FIXME [width=640px, height=474px]
                            \caption{kontaktierte Diode, Live-Bild der CCD
                            Kamera}
                        \end{figure}
    
                    \end{minipage}
                    \begin{minipage}{0.6\textwidth}
    
                         \begin{figure}[H]
                            \label{fig:}
                            \includegraphics[scale=0.25, trim = 0cm 0cm 0cm
                            0cm,
                            clip]{./Emissionsbilder/zwei/nach_Emissionsmessung_Intensitat_Distanz.jpg}
                            %FIXME [width=640px, height=474px]
                            \caption{selbe Diode nach der Emissionsmessung}
                        \end{figure}
                   \vspace{-1.5em}
    
                    \end{minipage}
    
                \end{tabular}
                \end{center}
                
        \vspace{2em}
        
        Nun wurde noch ein Die am Rand des Wafers ausgemessen um nach
        Unterschieden zwischen den Messergebnissen kontrolliert werden zu
        können. Auch auf dem Die in Zeile $15$ und Spalte $8$ wurde erst eine
        große, dann eine kleine Emissions-Diode untersucht. Die Durchlassspannung bei der
        großen Diode betrug $1,2\ V$ bei einem Strom von $25\ mA$. Die Messdauer
        wurde aus $3s$ variiert.
        
        
         \begin{center}
                \begin{tabular}{ll}
    
                \hspace{-10em}
                    \begin{minipage}{0.6\textwidth}
    
                        \begin{figure}[H]
                            \label{fig:}
                            \includegraphics[scale=0.25, trim = 0cm 0cm 0cm
                            0cm,
                            clip]{./Emissionsbilder/drei/nach_Kontaktierung.jpg}
                            %FIXME [width=640px, height=474px]
                            \caption{kontaktierte Diode, Live-Bild der CCD
                            Kamera}
                        \end{figure}
    
                    \end{minipage}
                    \begin{minipage}{0.6\textwidth}
    
                         \begin{figure}[H]
                            \label{fig:}
                            \includegraphics[scale=0.25, trim = 0cm 0cm 0cm
                            0cm,
                            clip]{./Emissionsbilder/drei/nach_Emissionsmessung_Distanz.jpg}
                            %FIXME [width=640px, height=474px]
                            \caption{selbe Diode nach der Emissionsmessung}
                        \end{figure}
                   \vspace{-1.5em}
    
                    \end{minipage}
    
                \end{tabular}
                \end{center}
                
        \vspace{2em}
        
        Die Durchlassspannung der kleineren Diode betrug $1,5\ V$ bei $30\ mA$
        Strom. Die Messdauer wurde auf $4s$ gestellt. Vor und nach der Messung
        wurden folgende Bilder aufgezeichnet:
        
         
         \begin{center}
                \begin{tabular}{ll}
    
                \hspace{-10em}
                    \begin{minipage}{0.6\textwidth}
    
                        \begin{figure}[H]
                            \label{fig:}
                            \includegraphics[scale=0.25, trim = 0cm 0cm 0cm
                            0cm,
                            clip]{./Emissionsbilder/vier/nach_Kontaktierung.jpg}
                            %FIXME [width=640px, height=474px]
                            \caption{kontaktierte Diode, Live-Bild der CCD
                            Kamera}
                        \end{figure}
    
                    \end{minipage}
                    \begin{minipage}{0.6\textwidth}
    
                         \begin{figure}[H]
                            \label{fig:}
                            \includegraphics[scale=0.25, trim = 0cm 0cm 0cm
                            0cm,
                            clip]{./Emissionsbilder/vier/nach_Emission_Distanz.jpg}
                            %FIXME [width=640px, height=474px]
                            \caption{selbe Diode nach der Emissionsmessung}
                        \end{figure}
                   \vspace{-1.5em}
    
                    \end{minipage}
    
                \end{tabular}
                \end{center}
                
        \vspace{2em}
        
        Bislang ist nur eine Emissionsintensität zu sehen. Mithilfe der
        Phemos-Software konnte daraus ein Intensitätsprofil erstellt werden,
        woraus am Ende dann auf die Diffusionslänge zurück geschlossen werden
        kann.
        
        \vspace{1.5em}
        
        Es wurde auch eine Fingerstrukter-Diode für die Messung verwendet,
        welche sich ebenfalls auf dem äußeren Die befand. Eine
        Diffusionslängenberechnung wurde für diese Diode nicht durchgeführt, da
        am Phemos ein Objektiv gewählt wurde, mit der die gesamte Diode auf des
        Bildschirm sichtbar war. Daher sind die Fingerstrukturen zwar sichtbar,
        aber die pn-Übergänge zwischen n-Finger und dem p-Substrat dazwischen,
        die Grenze an der die Emission erwartet wird, sind viel zu klein
        abgebildet, als dass eine Intensitätskurve mit der geringen Vergrößerung
        sinnvoll wäre. Daher sind nur die Kontaktierung und das Ergebniss der
        Emissionsmessung im Protokoll aufgeführt. Die Durchlassspannung während
        der Messung betrug $5\ V$ bei einem Strom von $20\ mA$ und einer
        Messdauer von $4s$. 
        
        
            \begin{center}
                \begin{tabular}{ll}
    
                \hspace{-10em}
                    \begin{minipage}{0.6\textwidth}
    
                        \begin{figure}[H]
                            \label{fig:}
                            \includegraphics[scale=0.25, trim = 0cm 0cm 0cm
                            0cm,
                            clip]{./Emissionsbilder/fuenf/nach_Kontaktierung.jpg}
                            %FIXME [width=640px, height=474px]
                            \caption{kontaktierte Diode, Live-Bild der CCD
                            Kamera}
                        \end{figure}
    
                    \end{minipage}
                    \begin{minipage}{0.6\textwidth}
    
                         \begin{figure}[H]
                            \label{fig:}
                            \includegraphics[scale=0.25, trim = 0cm 0cm 0cm
                            0cm,
                            clip]{./Emissionsbilder/fuenf/SuperImpose.jpg}
                            %FIXME [width=640px, height=474px]
                            \caption{selbe Diode nach der Emissionsmessung}
                        \end{figure}
                   \vspace{-1.5em}
    
                    \end{minipage}
    
                \end{tabular}
                \end{center}
                
        \vspace{2em}
        
        Man kann erkennen, dass die n-Finger, welche von links nach rechts
        Richtung p-Pad reichen, die Positionen der pn-Übergänge bilden. Daher
        wäre eigentlich eine Emission entlang der Ränder der n-Finger zu erwarten.
        Eine Erklärung, warum aber die größte Emission nur an der Stelle
        auftritt, wo der Abstand zwischen n- und p-Pad am geringsten ist, wäre, dass die
        Rekombination hinter der Raumladungszone dort viel größer ist, wo der
        Halbleiter den geringsten Widerstand aufweist. Der Strom könnte also
        dort am größten werden, wodurch eine stärkere Emission erklärt wäre.
        
        \end{quote}
        
        
        \subsection{Auswertung}
        \begin{quote}
        
        Die Phemos-Software ist in der Lage, aus der Messung eine Wertetabelle
        zu erstellen, welche die jeweilige Strahlungsintensität an dem
        betrachteten Punkt im Substrat ausgibt. Mit diesen Werten konnte die
        Strahlungsintensität, welche den Erwartungen nach einen exponentiellen
        Abstieg aufweisen sollte, nachgebildet werden. Um zur gesuchten
        Diffusionslänge zu gelangen wird diese Kurve nun logarithmisch
        aufgetragen. Denn wenn man sich nochmal die relevante Proportionalität
        ansieht:
        
        \begin{equation*}
        \begin{split}
            I(x) \sim \ \Delta n \sim \ exp(-\frac{x}{L_n})\\
            log(I(x)) \sim -\frac{x}{L_n} 
        \end{split}
        \end{equation*}
        
        dann erkennt man, dass eine Logarithmierung des Exponentialtherms nur
        das Argument übrig lässt. Daher sollte aus jeder Intensitätskurve
        eine absteigende Gerade entstehen, welche eine Steigung von $-\frac{1}{L_n}$
        besitzt. Über den Reziprokwert dieser Steigung kann man dann die
        Diffusionslänge $L_n$ der Dioden berechnen.
        
        \vspace{1em}
        
        Beginnend mit der ersten Messung wird die Intensitätskurve mit dem
        exponentiellen und dem logarithmischen Verlauf geplottet und analysiert:
        
        \begin{figure}[H]
                    \centering
                        \includegraphics[scale=0.53, trim = 1cm 6cm 1.5cm 8cm,
                        clip]{./Emissionsbilder/eins/Intensitatsmessung.pdf}
                        \caption{Intensitätskurve der 1.Messung, exponentiell
                        und logarithmisch}
                            \label{fig:./Emissionsbilder/eins/Intensitatsmessung.pdf}
        \end{figure}
            
        
        Die blaue Kurve zeit den exponentiellen, die rote Kurve den
        logarithmischen Verlauf. Aus dieser wird nun der Kehrwert der Steigung
        ermittelt, welche eine Diffusionslänge von $304,2\ \mu m$ liefert.\\
        
        Die zweite Messung an der kleineren Diode im selben Die liefert
        folgende Intensitätsverläufe
        
        \begin{figure}[H]
                    \centering
                        \includegraphics[scale=0.53, trim = 1cm 6cm 1.5cm 8cm,
                        clip]{./Emissionsbilder/zwei/Intensitatsmessung.pdf}
                        \caption{Intensitätskurve der 2.Messung, exponentiell
                        und logarithmisch}
                            \label{fig:./Emissionsbilder/zwei/Intensitatsmessung.pdf}
        \end{figure}
        
        und, mit der analogen Berechnung, eine Diffusionslänge von $304,6\ \mu
        m$.\\
        
        Die dritte Messung an der großen Emissions-Diode am Rand des Wafers,
        liefert folgende Intensitätsverläufe
        
        \begin{figure}[H]
                    \centering
                        \includegraphics[scale=0.53, trim = 1cm 6cm 1.5cm 8cm,
                        clip]{./Emissionsbilder/drei/Intensitatsmessung.pdf}
                        \caption{Intensitätskurve der 3.Messung, exponentiell
                        und logarithmisch}
                            \label{fig:./Emissionsbilder/drei/Intensitatsmessung.pdf}
        \end{figure}
        
        und eine Diffusionslänge von $304,3\ \mu m$, wohingegen die kleinere
        Diode auf dem äußeren Die diese Kurven
        
        \begin{figure}[H]
                    \centering
                        \includegraphics[scale=0.53, trim = 1cm 6cm 1.5cm 8cm,
                        clip]{./Emissionsbilder/vier/Intensitatsmessung.pdf}
                        \caption{Intensitätskurve der 4.Messung, exponentiell
                        und logarithmisch}
                            \label{fig:./Emissionsbilder/vier/Intensitatsmessung.pdf}
        \end{figure}
        
        mit der Diffusionslänge von erneut $304,6\ \mu m$ wiedergibt.
        
        \vspace{1.5em}
        
        Obwohl diese Werte weit über dem Erwartungswert einer Diffusionslänge
        von $150 - 200\ \mu m$ liegen, spricht es für die Messung, dass alle
        Werte ziemlich gleich groß sind. Da alle gemessenen Dioden sich auf
        demselben Wafer befinden und somit alle gleich dotiert und behandelt wurden, ist
        es auch verständlich, dass die Diffusionslängen nicht bedeutend viel
        voneinander abweichen.\\
        Die vermutete Ursache für die dennoch zu großen Diffusionslängen könnte
        in dem Dotierungsschritt während der Wafer-Herstellung liegen. Eventuell
        wurde im Reinraum ein unbemerkter Fehler begangen, der zu größeren
        $L_n$s führte.
        
        \end{quote}
    
\end{quote} %sec Emissionsmessung

%--------------------------------------------------------------------
%--------------------------------------------------------------------


\end{document}
